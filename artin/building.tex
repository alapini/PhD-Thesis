
%%%%%%%%%%%%%%%%%%%%%%%%%%%%%%%%%%%%%%%%%%%%%%%%%%%%%%%%%%%%

\subsection{Building the tower}

This subsection introduces the basic algorithms required to build the
tower, that is, compute the required minimal polynomials $Q_i$.

\paragraph*{\bf Composition.} We give first an algorithm for
polynomial composition, to be used in the construction of the tower
defined before.  Given $P$ and $R$ in $\F_p[X]$, we want to compute
$P(R)$. For the cost analysis, it will be useful later on to consider
both the degree $k$ and the number of terms $\ell$ of $R$.

\alg{Compose} is a recursive process that cuts $P$ into $c+1$
``slices'' of degree less than $p^n$, recursively composes them with
$R$, and concludes using Horner's scheme and the linearity of the
$p$-power. At the leaves of the recursion tree, we use the following naive
algorithm.

\begin{algorithm}
  \caption{NaiveCompose}
  \begin{algorithmic}[1]
    \REQUIRE $P,R\in\F_p[X]$.
    \ENSURE $P(R)$.
    \STATE write $P=\sum_{i=0}^{\deg(P)} p_i X^{i}$, with $p_i \in \F_p$
    \STATE let $S=0$, $\rho=1$
    \STATE for $i\in [0,\dots,\deg(P)]$, let $S=S+p_i \rho$ and $\rho =\rho R$
    \STATE return $S$
  \end{algorithmic}
\end{algorithm}

\begin{lemma}
  \alg{NaiveCompose} has cost $O(\deg(P)^2k\ell)$.  
\end{lemma}
\begin{proof} At step $i$, $\rho$ and $S$ have degree at most
$ik$. Computing the sum $S + p_i \rho$ takes time $O(ik)$ and
computing the product $\rho R$ takes time $O(ik\ell)$, since $R$ has
$\ell$ terms. The total cost of step $i$ is thus $O(ik\ell)$, 
whence a total cost of $O(\deg(P)^2 k\ell)$.
\end{proof}


\begin{algorithm}
  \caption{Compose}
  \begin{algorithmic}[1]
    \REQUIRE $P,R\in\F_p[X]$.
    \ENSURE $P(R)$.
    \STATE\label{c:params} let $n=\lfloor \log_p(\deg(P)) \rfloor$ and $c=\deg(P) {\sf~div~} p^n$
    \STATE If $n=0$, return $\alg{NaiveCompose}(P,R)$
    \STATE write $P=\sum_{i=0}^{c} P_i X^{ip^n}$, with $P_i \in \F_p[X], \deg P_i<p^n$
    \STATE for $i\in [0,\dots,c]$, let $Q_i = \text{\alg{Compose}}(P_i,R)$
    \STATE let $Q=0$
    \STATE\label{c:loop} for $i\in [c,\dots,0]$, let $Q = Q R(X^{p^n})  + Q_i$
    \STATE return $Q$
  \end{algorithmic}
\end{algorithm}

%\smallskip

\begin{theorem}
  \label{theo:comp}
  If $R$ has degree $k$ and $\ell$ non-zero coefficients and if
  $\deg(P)=s$, then \alg{Compose}$(P,R)$ outputs $P(R)$ in time $O(ps
  \log_p(s)k\ell)$.
\end{theorem}
\begin{proof} Correctness is clear, since $R^{p^n}=R(X^{p^n})$. To analyze
the cost, we let $\sC(c,n)$ be the cost of {\alg{Compose}} when
$\deg(P)\le (c+1)p^n$, with $c<p$. Then $\sC(c,0) \in
  O(c^2k\ell)$.  For $n > 0$, at each pass in the loop at
step~\ref{c:loop}, $\deg(Q) < cp^n k$, so that the multiplication
(using the naive algorithm) and addition take time
$O(cp^nk\ell)$. Thus the time spent in the loop is $O(c^2p^{n}k\ell)$,
and the running time satisfies $$\sC(c,n) \le (c+1)\sC(p-1,n-1) +
O(c^2p^nk\ell).$$ Let then $\sC'(n)=\sC(p-1,n)$, so that we have
$$\sC'(0) \in O(p^2k\ell), \quad \sC'(n) \le p\sC'(n-1) +
O(p^{n+2}k\ell).$$ We deduce that $\sC'(n) \in O(p^{n+2}nk\ell)$, and
finally $\sC(c,n) \in O(cp^{n+1}nk\ell + c^2p^nk\ell)$.  The
values $c,n$ computed at step~\ref{c:params} of the top-level call to
\alg{Compose} satisfy $cp^n\le s$ and $n\le\log_p (s)$; this gives our
conclusion. \end{proof}

\paragraph* A binary divide-and-conquer
algorithm~\cite[Ex.~9.20]{vzGG} has cost $O(\Mult(sk)\log(s))$. Our
algorithm has a slightly better dependency on $s$, but adds a
polynomial cost in $p$ and $\ell$. However, we have in mind cases
  with $p$ small and $\ell=2$, where the latter solution is
advantageous.

\paragraph*{\bf Computing the minimal polynomials.} Theorem~\ref{th:cantor} shows that we have defined a primitive tower. To be
able to work with it, we explain now how to compute the minimal
polynomial $Q_i$ of $x_i$ over $\F_p$. This is done by extending
Cantor's construction~\cite{Can89}, which had $\U_0=\F_p$.

For $i=0$, we are given $Q_0\in\F_p[X_0]$ such that
$\U_0=\F_p[X_0]/Q_0(X_0)$, so there is nothing to do; we assume that
$\Tr_{\U_0/\F_p}(x_0)\ne0$ to meet the hypotheses of
Theorem~\ref{th:cantor}. Remark that if this trace was zero, assuming
$\gcd(d,p)=1$, we could replace $Q_0$ by $Q_0(X_0-1)$; this is done by
taking $R=X_0-1$ in algorithm \alg{Compose}, so by
Theorem~\ref{theo:comp} the cost is $O(pd \log_p(d))$.

For $i=1$, we know that $x_1^p-x_1=x_0$, so $x_1$ is a root of
$Q_0(X_1^p-X_1)$. Since $Q_0(X_1^p-X_1)$ is monic of degree $pd$, we
deduce that $Q_1=Q_0(X_1^p-X_1)$. To compute it, we use algorithm
\alg{Compose} with arguments $Q_0$ and $R=X_1^p-X_1$; the cost is
$O(p^2d \log_p(d))$ by Theorem~\ref{theo:comp}. The same arguments
hold for $i=2$ when $p=2$ and $d$ is odd.

To deal with other indexes $i$, we follow Cantor's construction.  Let
$\Cyclo\in \F_p[X]$ be the reduction modulo $p$ of the $(2p-1)$th
cyclotomic polynomial. Cantor implicitly works modulo an irreducible
factor of $\Cyclo$. The following shows that we can avoid
factorization, by working modulo $\Cyclo$.

\begin{lemma}
  \label{lemma:poly-cyclic}
  Let $A=\F_p[X]/\Cyclo$ and let $x = X \bmod
  \Cyclo$. For $Q\in\F_p[Y]$, define $Q^\star =
  \prod_{i=0}^{2p-2}Q(x^iY).$ Then $Q^\star$ is in $\F_p[Y]$ and there
  exists $q^\star\in\F_p[Y]$ such that $Q^\star = q^\star(Y^{2p-1})$.
\end{lemma}
\begin{proof} Let $F_1,\dots,F_e$ be the irreducible factors of $\Cyclo$
and let $f$ be their common degree. To prove that $Q^\star$ is in
$\F_p[Y]$, we prove that for $j \le e$, $Q^\star_j = Q^\star \bmod
F_j$ is in $\F_p[Y]$ and independent from $j$; the claim follows by
Chinese Remaindering.

For $j \le e$, let $a_j$ be a root of $F_j$ in the algebraic closure
of $\F_p$, so that $Q^\star_j = \prod_{i=0}^{2p-2}Q(a_j^iY).$ Since
$\gcd(p^f,2p-1)=1$, $Q^\star_j$ is invariant under ${\rm
  Gal}(\F_{p^f}/\F_p)$, and thus in $\F_p[Y]$. Besides, for $j,j'\le
e$, $a_j = a_{j'}^k$, for some $k$ coprime to $2p-1$, so that
$Q^\star_j= Q^\star_{j'}$, as needed.  

To conclude, note that for $j \le e$, $Q^\star_j(a_jY)=Q^\star_j(Y)$,
so that all coefficients of degree not a multiple of $2p-1$ are zero.
Thus, $Q^\star_j$ has the form $q^\star_j(Y^{2p-1})$; by Chinese
Remaindering, this proves the existence of the polynomial $q^\star$.
%% 12-02 %% I think you're messing up indices. You meant
%%%%%%%%%%% $i = \euler(2p-1)/e$ and $j<i$, I believe.
%% 12-31 %% Yup.
\end{proof}


\medskip

We conclude as in~\cite{Can89}: supposing that we
know the minimal polynomial $Q_i$ of $x_i$ over $\F_p$, we compute
$Q_{i+1}$ as follows. Since $x_i$ is a root of $Q_i$, it is a root of
$Q_i^\star$, so $\gamma_i=x_i^{2p-1}$ is a root of $q_i^\star$ and
$x_{i+1}$ is a root of $q_i^\star(Y^p-Y)$. Since the latter polynomial
is monic of degree $p^{i+1}d$, it is the minimal polynomial $Q_{i+1}$
of $x_{i+1}$ over $\F_p$.

\begin{theorem}
  Given $Q_i$, one can compute $Q_{i+1}$ in time
  $O( p^{i+2}d\log_p(p^id)+\Mult(p^{i+2}d)\log(p))$.
\end{theorem}
\begin{proof} Let $A=\F_p[X]/\Cyclo$. The algorithm
of~\cite{Brent93} computes $\Cyclo$ in time $O(p^2)$; then, polynomial
multiplications in degree $s$ in $A[Y]$ can be done in time
$O(\Mult(sp))$ by Kronecker substitution. The overall cost of
computing $Q_i^\star$ is $O(\Mult(p^{i+2}d)\log p)$
using~\cite[Algo.~10.3]{vzGG}. To get $Q_{i+1}$ we use algorithm
\alg{Compose} with $R=Y^p-Y$, which costs $O(p^{i+2}d\log_p(p^id))$.
\end{proof}

\smallskip

The former cost is linear in $p^{i+2}d$, up to logarithmic factors,
for an input of size $p^id$ and an output of size $p^{i+1}d$.

Some further operations will be performed when we construct the tower:
we will precompute quantities that will be of use in the algorithms of
the next sections. Details are given in the next sections, when
needed.


% Local Variables:
% mode:flyspell
% ispell-local-dictionary:"american"
% mode: TeX-PDF
% mode: reftex
% TeX-master: "../these"
% End:
%
% LocalWords:  Schreier Artin pseudotrace frobenius bivariate memoization monic
% LocalWords:  Horner Horner's cyclotomic polynomially automorphisms precompute
% LocalWords:  automorphism
