\section{Level embedding}\label{sec:level-embedding}


We discuss here change-of-basis algorithms for the tower $(\U_0,
\ldots, \U_k)$ of the previous section; these algorithms are needed
for most further operations. We detail the main case where $P_i =
X_i^p - X_i - X_{i-1}^{2p-1}$; the case $P_1= X_1^p - X_1 - X_0$ (and
$P_2=X_2^2+X_2+X_1$ for $p=2$ and $d$ odd) is easier.

By Theorem~\ref{th:cantor}, $\U_i$ equals $\F_p[X_{i-1},X_i]/I$, where
the ideal $I$ admits the following Gr{\"o}bner bases, for respectively
the lexicographic orders $X_i>X_{i-1}$ and $X_{i-1}>X_i$:
\begin{equation*}
  \left |
  \begin{array}{rl}
    X_i^p - X_i - X_{i-1}^{2p-1} \\
    Q_{i-1}(X_{i-1})         
  \end{array}
\right.
  \quad \text{and}\quad
  \left |
  \begin{array}{rl}
    X_{i-1} - R_i(X_i) \\
    Q_i(X_i),
  \end{array}
\right.
\end{equation*}
with $R_i$ in $\F_p[X_i]$. Since $\deg(Q_{i-1})=p^{i-1}d$ and
$\deg(Q_{i})=p^id$, we associate the following $\F_p$-bases of $\U_i$
to each system:
\begin{eqnarray}
{\bf D}_i&   =   &(x_i^j,\,x_{i-1}x_i^j,\,\ldots,\,x_{i-1}^{p^{i-1}d-1}x_i^j)_{0 \le j < p},\notag\\[-1mm]
\bC_i&   =   &(1,\,x_i,\,\ldots,\,x_i^{p^id-1}).  \label{eq:bases}
\end{eqnarray}
We describe an algorithm called \alg{Push-down} which takes $v$
written on the basis $\bC_i$ and returns its coordinates on the basis
${\bf D}_i$; we also describe the inverse operation, called \alg{Lift-up}. 
In other words, \alg{Push-down} inputs $v\wrt\U_i$ and outputs the
representation of $v$ as
\begin{equation}
  \label{eq:vectorspace}
  v = v_0 + v_1x_i + \cdots + v_{p-1}x_i^{p-1}, \quad\text{with all~} v_j \wrt \U_{i-1}
\end{equation}
and \alg{Lift-up} does the opposite.

Hereafter, we let $\L:\N-\{0\} \to \N$ be such that both
\alg{Push-down} and \alg{Lift-up} can be performed in time $\L(i)$; to
simplify some expressions appearing later on, we add the mild
constraints that $p\,\L(i) \le \L(i+1)$ and $p\,\Mult(p^{i}d)
\in O(\L(i))$.
To reflect the implementation's behavior, we also allow
precomputations. These precomputations are performed when we build
the tower; further details are at the end of this section.
\begin{theorem}\label{theo:L}
   One can take $\L(i)$ in $O( p^{i+1}d\log_p(p^id)^2 \ + \
\,p\,\Mult(p^{i}d))$.
\end{theorem}
Remark that the input and output have size $p^id$; using fast
multiplication, the cost is linear in $p^{i+1}d$, up to logarithmic
factors. The rest of this section is devoted to proving this
theorem.  \alg{Push-down} is a divide-and-conquer process, adapted to
the shape of our tower; \alg{Lift-up} uses classical ideas of trace
computations (as in the algorithm \alg{FindParameterization} of
Section~\ref{ssec:duality}); the values we need will be obtained using
the transposed version of \alg{Push-down}.

As said before, the algorithms of this section (and of the following
ones) use precomputed quantities. To keep the pseudo-code simple, we
do not explicitly list them in the inputs of the algorithms;
we show, later, that the precomputation is fast too.

%%%%%%%%%%%%%%%%%%%%%%%%%%%%%%%%%%%%%%%%%%%%%%%%%%%%%%%%%%%%

\subsection{Modular multiplication}\label{ssec:mulmod}

We first discuss a routine for multiplication by $X_i^{p^n}$
in $\F_p[Y,X_i]/(X_i^p-X_i-Y)$, and its transpose. We start by
remarking that $X_i^{p^n}=X_i+R_n \bmod X_i^p-X_i-Y$, with
\begin{equation}
  \label{eq:Kn}
 \begin{array}{c}R_n = \sum_{j=0}^{n-1}
  Y^{p^j}.
\end{array}
\end{equation}
Then, precisely, for $k$ in $\N$, we are interested in the operation
$\alg{MulMod}_{k,n}: A \mapsto (X_i+R_n)A \bmod X_i^p-X_i-Y$,
with $A\in \F_p[Y,X_i]$, $\deg(A,Y) < k$ and $\deg(A,X_i) <p$.

Since $R_n$ is sparse, it is advantageous to use the naive algorithm;
besides, to make transposition easy, we explicitly give the matrix of
$\alg{MulMod}_{k,n}$. Let $m_0$ be the $(k+p^{n-1})\times k$ matrix
having $1$'s on the diagonal only, and for $\ell \le p^{n-1}$, let
$m_\ell$ be the matrix obtained from $m_0$ by shifting the diagonal
down by $\ell$ places. Let finally $m'$ be the sum $\Sigma_{j=0}^{n-1}
m_{p^j}$. Then one verifies that the matrix of $\alg{MulMod}_{k,n}$
is $$\left [
\begin{matrix}
m'  &     &        &        & m_1 \\
m_0 & m'  &        &        & m_0 \\
    & m_0 & m'     &        &     \\
    &     & \ddots & \ddots &     \\
    &     &        & m_0    & m'
\end{matrix}
\right ],$$ with columns indexed by 
$(X_i^j,\dots,Y^{k-1}X_i^j)_{j < p}$ and rows by
$(X_i^j,\dots,Y^{k+p^{n-1}-1}X_i^j)_{j < p}$.  Since this matrix
has $O(pnk)$ non-zero entries,  we can compute both 
$\alg{MulMod}_{k,n}$ and its dual $\dual{\alg{MulMod}_{k,n}}$ in time $O(pnk)$.


%%%%%%%%%%%%%%%%%%%%%%%%%%%%%%%%%%%%%%%%%%%%%%%%%%%%%%%%%%%%

\subsection{Push-down}\label{sec:level-embedding:push-down}

The input of \alg{Push-down} is $v \wrt \U_i$, that is, given on the
basis $\bC_i$; we see it as a polynomial $V \in \F_p[X_i]$ of degree
less than $p^id$. The output is the normal form of $V$ modulo
$X_i^p-X_i-X_{i-1}^{2p-1}$ and $Q_{i-1}(X_{i-1})$. We first use a
divide-and-conquer subroutine to reduce $V$ modulo
$X_i^p-X_i-X_{i-1}^{2p-1}$; then, the result is reduced modulo
$Q_{i-1}(X_{i-1})$ coefficient-wise.

\smallskip

To reduce $V$ modulo $X_i^p-X_i-X_{i-1}^{2p-1}$, we first compute $W=V \bmod
X_i^p-X_i-Y$, then we replace $Y$ by $X_{i-1}^{2p-1}$ in~$W$.  Because our
algorithm will be recursive, we let $\deg(V)$ be arbitrary; then, we
have the following estimate for $W$.

\begin{lemma}
  \label{th:push-down-degree} We have $\deg(W,Y)\le \deg(V)/p$.
\end{lemma}
\begin{proof} Consider the matrix $M$ of multiplication by $X_i^p$ modulo
$X_i^p-X_i-Y$; it has entries in $\F_p[Y]$. Due to the
sparseness of the modulus, one sees that $M$ has degree at most $1$,
and so $M^k$ has coefficients of degree at most $k$. Thus, the
remainders of $X_i^{pk},\dots,X_i^{pk+p-1}$ modulo $X_i^p-X_i-Y$
have degree at most $k$ in $Y$. \end{proof}

\smallskip 

We compute $W$ by a recursive subroutine \alg{Push-down-rec}, similar
to \alg{Compose}. As before, we let $c,n$ be such that $1\le c<p$ and
$\deg(V) < (c+1)p^n$, so that we have
$$V=V_0+ V_1X_i^{p^n}+\cdots+V_c X_i^{cp^n},$$ with all $V_j$ in
$\F_p[X_i]$ of degree less than $p^n$. First, we recursively reduce
$V_0,\dots,V_c$ modulo $X_i^p-X_i-Y$, to obtain bivariate
polynomials $W_0,\dots,W_{c}$. Let $R_n$ be the polynomial defined in
Equation~\eqref{eq:Kn}. Then, we get $W$ by computing
$\Sigma_{j=0}^c W_j(X_i+R_n)^j$ modulo $X_i^p-X_i-Y$,
using Horner's scheme as in \alg{Compose}. Multiplications by
$X_i+R_n$ modulo $X_i^p-X_i-Y$ are done using \alg{MulMod}.

\begin{algorithm}
  {Push-down-rec} {$V\in \F_p[X_i]$ and $c,n\in\N$.}{$W \in\F_p[Y,X_i]$.}
\item if $n=0$ return $V$
\item write $V=\sum_{j=0}^{c} V_j X_i^{jp^n}$, with $V_j \in \F_p[X_i], \deg V_j<p^n$
\item for $j\in [0,\dots,c]$, let $W_j=\text{{\sf Push-down-rec}}(V_j,p-1,n-1)$
\item $W=0$
\item\label{pd:loop} for $j\in [c,\dots,0]$, let $W = \alg{MulMod}_{(c+1)p^{n-1},n}(W) + W_j$
\item return $W$
\end{algorithm}
\begin{algorithm}
  {Push-down}{$v\wrt \U_i$.}{ $v$ written as $v_0+\cdots+v_{p-1}x_i^{p-1}$ with $v_j \wrt \U_{i-1}$.}
\item let $V$ be the canonical preimage of $v$ in $\F_p[X_i]$
\item let $n=\lfloor \log_p(p^id-1) \rfloor$ and $c=(p^id-1){\sf~div~} p^n$
\item let $W = \text{\alg{Push-down-rec}}(V,c,n)$
\item let $Z = \text{Evaluate}(W,[X_{i-1}^{2p-1},X_i])$
\item \label{step:pd:mod} let $Z = Z {\sf~mod~} Q_{i-1}$
\item \label{step:pd:return} return the residue class of $Z$ mod $(X_i^p - X_i - X_{i-1}^{2p-1},Q_{i-1})$
\end{algorithm}

\begin{proposition}\label{prop:pd}
  Algorithm \alg{Push-down} is correct and takes time $O(p^{i+1}d
  \log_p(p^id)^2 + p\,\Mult(p^id))$.
\end{proposition}
\begin{proof} Correctness is straightforward; note that at
step~\ref{pd:loop} of \alg{Push-down-rec}, $\deg(W,Y) <
(c+1)p^{n-1}$, so our call to $\alg{MulMod}_{(c+1)p^{n-1},n}$ is
justified. By the claim of Subsection~\ref{ssec:mulmod} on the cost of
$\alg{MulMod}$, the total time spent in that loop is $O(nc^2p^n)$. As
in Theorem~\ref{theo:comp}, we deduce that the time spent in
\alg{Push-down-rec} is $O(n^2c^2p^n)$.

In \alg{Push-down}, we have $cp^n< p^id$ and $n<\log_p (p^id)$, so the
previous cost is seen to be $O(p^{i+1}d \log_p(p^id)^2)$. Reducing one
coefficient of $Z$ modulo $Q_{i-1}$ takes time $O(\Mult(p^id))$, so
step~\ref{step:pd:mod} has cost
$O(p\,\Mult(p^id))$. Step~\ref{step:pd:return} is free, since at this
stage $Z$ is already reduced. \end{proof}

%%%%%%%%%%%%%%%%%%%%%%%%%%%%%%%%%%%%%%%%%%%%%%%%%%%%%%%%%%%%
%%%%%%%%%%%%%%%%%%%%%%%%%%%%%%%%%%%%%%%%%%%%%%%%%%%%%%%%%%%%
%%%%%%%%%%%%%%%%%%%%%%%%%%%%%%%%%%%%%%%%%%%%%%%%%%%%%%%%%%%%

\subsection{Transposed push-down}

Before giving the details for \alg{Lift-up}, we discuss here the
transpose of \alg{Push-down}.  \alg{Push-down} is the $\F_p$-linear
change-of-basis from the basis $\bC_i$ to $\bD_i$, so its transpose
takes an $\F_p$-linear form $\ell \in \dual{\U_i}$ given by its values
on $\bD_i$, and outputs its values on $\bC_i$. The input is the
(finite) generating series $L=\Sigma_{a < p^{i-1}d,\, b < p}\,
\ell(x_{i-1}^ax_i^b)X_{i-1}^aX_i^b$; the output is $M=\Sigma_{a <
  p^id}\, \ell(x_i^a)X_i^a$.

As in~\cite{BoLeSc03}, the transposed algorithm is obtained by
reversing the initial algorithm step by step, and replacing
subroutines by their transposes. The overall cost remains the same; we
review here the main transformations.

In \alg{Push-down-rec}, the initial loop at step~\ref{pd:loop} is a
Horner scheme; the transposed loop is run backward, and its core
becomes $L_j=L\bmod Y^{n-1}$ and
$L=\dual{\alg{MulMod}_{(c+1)p^{n-1},n}}(L)$; a small simplification
yields the pseudo-code we give.  In \alg{Push-down}, after calling
\alg{Push-down-rec}, we evaluate $W$ at $[X_{i-1}^{2p-1},X_i]$: the
transposed operation $\dual{{\sf Evaluate}}$ maps the series
$\Sigma_{a,b}\, \ell_{a,b} X_{i-1}^a X_i^b$ to $\Sigma_{a,b}\,
\ell_{(2p-1)a,b}\, Y^a X_i^b$. Then, originally, we perform a
Euclidean division by $Q_{i-1}$ on $Z$. The transposed algorithm
$\dual{\sf mod}$ is in~\cite[Sect.~5.2]{BoLeSc03}: the transposed
Euclidean division amounts to compute the values of a sequence
linearly generated by the polynomial $Q_{i-1}$ from its first
$p^{i-1}d$ values.

\begin{algorithm}
  {$\dual{\alg{\text{Push-down-rec}}}$}
  {$L\in\F_p[Y,X_i]$ and $c,n\in\N$.}
  {$M\in \F_p[X_i]$}
\item If $n=0$ return $L$
\item\label{pdt:loop} for $j\in [c,\dots,0]$,
  \begin{itemize*}
  \item let $L_j = L \bmod Y^{n-1}$
  \item let $M_j=\dual{\text{\sf Push-down-rec}}(L_j,p-1,n-1)$
  \item let $L = \dual{\alg{MulMod}_{(c+1)p^{n-1},n}}(L)$
  \end{itemize*}
\item return $\sum_{j=0}^{c} M_j X_i^{jp^n}$
\end{algorithm}
\begin{algorithm}
  {$\dual{\alg{\text{Push-down}}}$} {$L\in \F_p[X_{i-1},X_i]$} {$M \in \F_p[X_i]$}
\item let $n=\lfloor \log_p(p^id-1) \rfloor$ and $c=(p^id-1) {\sf~div~} p^n$
\item let $P=\dual{{\sf mod}}(L,Q_{i-1})$
\item let $M = \dual{\text{Evaluate}}(P,[X_{i-1}^{2p-1},X_i])$
\item return $\dual{\text{\alg{Push-down-rec}}}(M,c,n)$
\end{algorithm}

%%%%%%%%%%%%%%%%%%%%%%%%%%%%%%%%%%%%%%%%%%%%%%%%%%%%%%%%%%%%
%%%%%%%%%%%%%%%%%%%%%%%%%%%%%%%%%%%%%%%%%%%%%%%%%%%%%%%%%%%%
%%%%%%%%%%%%%%%%%%%%%%%%%%%%%%%%%%%%%%%%%%%%%%%%%%%%%%%%%%%%

\subsection{Lift-up}
\label{sec:level-embedding:lift-up}

Let $v$ be given on the basis $\bD_i$ and let $W$ be its canonical
preimage in $\F_p[X_{i-1},X_i]$.  The lift-up algorithm finds $V$ in
$\F_p[X_i]$ such that $W=V \bmod (X_i^p-X_i-X_{i-1}^{2p-1},Q_{i-1})$
and outputs the residue class of $V$ modulo $Q_i$. Hereafter, we
assume that both $Q_i'^{-1} \bmod Q_i$ and the values of the trace
$\Tr_{\U_i/\F_p}$ on the basis $\bD_i$ are known.  The latter will be
given under the form of the (finite) generating series
$$\begin{array}{c}
  S_i=\sum_{a < p^{i-1}d,\, b < p} \Tr_{\U_i/\F_p}(x_{i-1}^ax_i^b)X_{i-1}^a X_i^b,
\end{array}
$$ 
see the discussion below.

Then, as in Subsection~\ref{ssec:duality}, we use trace formulas to
write $v$ as a polynomial in $x_i$: we see $\U_i$ as a separable
extension over $\F_p$ and we look for a parameterization
$v=A(x_i)$. To do this, we compute the values of
$L=v\cdot\Tr_{\U_i/\F_p}$ on the basis $\bD_i$ via transposed
multiplication (see Subsection~\ref{ssec:duality}) and rewrite
equations~\eqref{eq:MN} as
\begin{equation}
  \label{eq:MNliftup}
  M = \sum_{j < p^id}L(x_i^j)X_i^j,\quad N = M\rev_{p^id}(Q_i) \bmod X_i^{p^id}.
\end{equation}
To compute the values of $M$ we could use~\cite[Th.~4]{Sho94} as we
did in step~\ref{alg:para:trmodcomp} of \alg{FindParameterization}; it
is however more efficient to use \alg{Push-down$^\ast$} as it was
shown in the previous subsection. The rest of the computation goes as
in steps~\ref{alg:para:multrunc} and~\ref{alg:para:mulmod} of
\alg{FindParametrization}.

\begin{algorithm}{Lift-up}
  { $v$ written as $v_0+\cdots+v_{p-1}x_i^{p-1}$ with $v_j \wrt \U_{i-1}$.}{$v\wrt \U_i$.}
\item let $W$ be the canonical preimage of $v$ in $\F_p[X_{i-1},X_i]$
\item \label{alg:lift-up:transmul} let $L = \alg{TransposedMul}(W,\,S_i)$
\item \label{alg:lift-up:pow} let $M=\dual{{\text{\sf Push-down}}}(L)$
\item \label{alg:lift-up:mult} let $N = M \rev_{p^id}(Q_i) {\sf ~mod~} X_i^{p^id}$
\item \label{alg:lift-up:mulmod} let $V=\rev_{p^id-1}(N) {Q_i'}^{-1} {\sf ~mod~} Q_i$
\item \label{alg:lift-up:ret} return the residue class of $V$ modulo $Q_i$
\end{algorithm}

\begin{proposition}\label{prop:lu}
 Algorithm \alg{Lift-up} is correct and takes time $O(p^{i+1}d\log_p(p^id)^2+p\,\Mult(p^{i}d))$.
\end{proposition}
\begin{proof} Correctness is clear by the discussion
above. \alg{TransposedMul} implements the transposed multiplication;
an algorithm of cost $O(\Mult(p^id))$ for this is
in~\cite[Coro.~2]{PS06}.  The last subsection showed that
step~\ref{alg:lift-up:pow} has the same cost as \alg{Push-down}. Then,
the costs of steps~\ref{alg:lift-up:mult} and~\ref{alg:lift-up:mulmod}
are $O(\Mult(p^id))$ and step~\ref{alg:lift-up:ret} is free since $V$
is reduced.\end{proof}


\smallskip

Propositions~\ref{prop:pd} and~\ref{prop:lu} prove
Theorem~\ref{theo:L}. The precomputations, that are done at the
construction of $\U_i$, are as follows. First, we need the values of
the trace on the basis ${\bf D}_i$; they are obtained in time $O(\Mult(p^id))$
by~\cite[Prop.~8]{PS06}. Then, we need ${Q_i'}^{-1} \bmod Q_i$; this
takes time $O(\Mult(p^id) \log(p^id))$ by fast extended GCD
computation.  These precomputations save logarithmic factors at best,
but are useful in practice.

%\eject 

% Local Variables:
% mode:flyspell
% ispell-local-dictionary:"american"
% mode: TeX-PDF
% mode: reftex
% TeX-master: "../these"
% End:
%
% LocalWords:  Schreier Artin pseudotrace frobenius bivariate memoization
% LocalWords:  precomputations precomputation
