\section{Level embedding}
\label{sec:level-embedding}

We discuss here change-of-basis algorithms for the tower $(\U_0,
\ldots, \U_k)$ of the previous section; these algorithms are needed
for most further operations. We detail the main case where $P_i =
X_i^p - X_i - X_{i-1}^{2p-1}$; the case $P_1= X_1^p - X_1 - X_0$ (and
$P_2=X_2^2+X_2+X_1$ for $p=2$ and $d$ odd) is easier.

Recall the two families of $\F_p$-bases we have defined so far:
\begin{align}
  \basis{B}_i &=
  \{x_0^{e_0} \cdots x_i^{e_i} \;|\; 0 \le e_0 < d,\; 0\le e_j < p 
  \text{ for $j>0$}\}
  \text{,}\\
  \basis{C}_i &=(1,x_i,\dots,x_i^{p^id-1})
  \text{.}  
\end{align}
The first one arises naturally when constructing the tower as a
succession of Artin-Schreier extensions, and we expect our inputs to
be given in such basis. Furthermore, lifting in $\basis{B}_j$ an
element written on $\basis{B}_i$ for $i<j$ is immediate in this basis,
and so is the inverse operation. The basis $\basis{C}_i$, on the
other hand, is practical for multiplication, inversion, etc., but it
is not evident how to lift elements.

We shall thus need algorithms to change between these two bases. Since
$x_i$ is clearly a separating element for the variety $V(K_i)$ (see
Chapter~\ref{cha:trace-computations}), we will use
Proposition~\ref{th:uni-multi-uni} to go from $\basis{B}_i$ to
$\basis{C}_i$, but we shall need an algorithm for the inverse map first.

Instead of converting from $\basis{C}_i$ to $\basis{B}_i$ in one shot,
we will pass through some intermediate bivariate bases to keep the
complexity low. By Theorem~\ref{th:cantor}, $\U_i$ equals
$\F_p[X_{i-1},X_i]/I$, where the ideal $I$ admits the following
Gr{\"o}bner bases, for respectively the lexicographic orders
$X_i>X_{i-1}$ and $X_{i-1}>X_i$:
\begin{equation}
  \left |
  \begin{array}{rl}
    X_i^p - X_i - X_{i-1}^{2p-1} \\
    Q_{i-1}(X_{i-1})         
  \end{array}
\right.
  \quad \text{and}\quad
  \left |
  \begin{array}{rl}
    X_{i-1} - R_i(X_i) \\
    Q_i(X_i),
  \end{array}
\right.
\end{equation}
with $R_i$ in $\F_p[X_i]$. Both Gröbner bases are triangular and
bivariate, one can go from one to the other using the algorithms of
\cite{pascal+schost06}, in fact most of the ideas of this section are
inspired by this paper.

Since $\deg(Q_{i-1})=p^{i-1}d$ and $\deg(Q_{i})=p^id$, we associate
the following $\F_p$-bases of $\U_i$ to each system:
\begin{align}
  \basis{D}_i &= (x_i^j,\,x_{i-1}x_i^j,\,\ldots,\,x_{i-1}^{p^{i-1}d-1}x_i^j)_{0 \le j < p}\text{,}\\
  \label{eq:bases}
  \basis{C}_i &= (1,\,x_i,\,\ldots,\,x_i^{p^id-1})\text{.}  
\end{align}
We describe an algorithm called \hyperref[alg:push-down]{\alg{Push-down}} which takes $v$
written on the basis $\basis{C}_i$ and returns its coordinates on the
basis $\basis{D}_i$. Then, using Proposition~\ref{th:uni-multi-uni},
we will be able to describe the inverse operation, called
\hyperref[alg:liftup]{\alg{Lift-up}}.  In other words, \hyperref[alg:push-down]{\alg{Push-down}} inputs $v\wrt\U_i$ and
outputs the representation of $v$ as
\begin{equation}
  \label{eq:vectorspace}
  v = v_0 + v_1x_i + \cdots + v_{p-1}x_i^{p-1}, \quad\text{with all~} v_j \wrt \U_{i-1}
\end{equation}
and \hyperref[alg:liftup]{\alg{Lift-up}} does the opposite.

Then, the change from $\basis{C}_i$ to $\basis{B}_i$ is done by
repeatedly applying \hyperref[alg:push-down]{\alg{Push-down}}, and the opposite is obtained by
repeatedly applying \hyperref[alg:liftup]{\alg{Lift-up}}.

Hereafter, we let $\Lift:\N-\{0\} \to \N$ be such that both
\hyperref[alg:push-down]{\alg{Push-down}} and \hyperref[alg:liftup]{\alg{Lift-up}} can be performed in time $\Lift(i)$; to
simplify some expressions appearing later on, we add the mild
constraints that $p\,\Lift(i) \le \Lift(i+1)$ and $p\,\Mult(p^{i}d)
\in O(\Lift(i))$.
To reflect the implementation's behavior, we also allow
precomputations. These precomputations are performed when we build
the tower; further details are at the end of this section.
\begin{theorem}\label{theo:L}
  One can take $\Lift(i)$ in $O( p^{i+1}d\log_p(p^id)^2 \ + \
\,p\,\Mult(p^{i}d))$.
\end{theorem}
Remark that the input and output have size $p^id$; using fast
multiplication, the cost is linear in $p^{i+1}d$, up to logarithmic
factors. The rest of this section is devoted to proving this theorem.
\hyperref[alg:push-down]{\alg{Push-down}} is a divide-and-conquer process, adapted to the shape
of our tower; \hyperref[alg:liftup]{\alg{Lift-up}} is a special case of
Proposition~\ref{th:uni-multi-uni}, the power projection will be
obtained using the transposed version of \hyperref[alg:push-down]{\alg{Push-down}}.

As said before, the algorithms of this section (and of the following
ones) use precomputed quantities. To keep the pseudo-code simple, we
do not explicitly list them in the inputs of the algorithms;
we show, later, that the precomputation is fast too.

%%%%%%%%%%%%%%%%%%%%%%%%%%%%%%%%%%%%%%%%%%%%%%%%%%%%%%%%%%%%

\subsection{Modular multiplication}
\label{ssec:mulmod}

We first discuss a routine for multiplication by $X_i^{p^n}$
in $\F_p[Y,X_i]/(X_i^p-X_i-Y)$, and its transpose. We start by
remarking that 
\begin{equation}
  \label{eq:Kn}
  X_i^{p^n}=X_i+R_n \bmod X_i^p-X_i-Y \qquad\text{with }
  R_n = \sum_{j=0}^{n-1} Y^{p^j}
  \text{.}
\end{equation}
Then, precisely, for $k$ in $\N$, we are interested in the operation
\begin{equation}
  \label{eq:89}
  \alg{MulMod}_{k,n}: A \mapsto (X_i+R_n)A \bmod X_i^p-X_i-Y
  \text{,}  
\end{equation}
with $A\in \F_p[Y,X_i]$, $\deg_Y(A) < k$ and $\deg_{X_i}(A) <p$.

Since $R_n$ is sparse, it is advantageous to use the naive algorithm;
besides, to make transposition easy, we explicitly give the matrix of
$\hyperref[eq:89]{\alg{MulMod}_{k,n}}$. Let $m_0$ be the
$(k+p^{n-1})\times k$ matrix having $1$'s on the diagonal only, and
for $\ell \le p^{n-1}$, let $m_\ell$ be the matrix obtained from $m_0$
by shifting the diagonal down by $\ell$ places. Let finally $m'$ be
the sum $\Sigma_{j=0}^{n-1} m_{p^j}$. Then one verifies that the
matrix of $\hyperref[eq:89]{\alg{MulMod}_{k,n}}$ is $$\left [
\begin{matrix}
m'  &     &        &        & m_1 \\
m_0 & m'  &        &        & m_0 \\
    & m_0 & m'     &        &     \\
    &     & \ddots & \ddots &     \\
    &     &        & m_0    & m'
\end{matrix}
\right ],$$ 
with columns and rows indexed by 
\begin{equation}
  \label{eq:90}
  (X_i^j,\dots,Y^{k-1}X_i^j)_{j < p}
  \quad\text{and}\quad
  (X_i^j,\dots,Y^{k+p^{n-1}-1}X_i^j)_{j < p}  
\end{equation}
respectively.  Since this matrix has $O(pnk)$ non-zero entries, we can
compute both $\hyperref[eq:89]{\alg{MulMod}_{k,n}}$ and its dual
$\dual{\hyperref[eq:89]{\alg{MulMod}_{k,n}}}$ in time $O(pnk)$.


%%%%%%%%%%%%%%%%%%%%%%%%%%%%%%%%%%%%%%%%%%%%%%%%%%%%%%%%%%%% 

\subsection{Push-down}\label{sec:level-embedding:push-down}

The input of \hyperref[alg:push-down]{\alg{Push-down}} is $v \wrt \U_i$, that is, given on the
basis $\basis{C}_i$; we see it as a polynomial $V \in \F_p[X_i]$ of degree
less than $p^id$. The output is the normal form of $V$ modulo
$X_i^p-X_i-X_{i-1}^{2p-1}$ and $Q_{i-1}(X_{i-1})$. We first use a
divide-and-conquer subroutine to reduce $V$ modulo
$X_i^p-X_i-X_{i-1}^{2p-1}$; then, the result is reduced modulo
$Q_{i-1}(X_{i-1})$ coefficient-wise.

To reduce $V$ modulo $X_i^p-X_i-X_{i-1}^{2p-1}$, we first compute 
\begin{equation}
  \label{eq:91}
  W=V \bmod X_i^p-X_i-Y
  \text{,} 
\end{equation}
then we replace $Y$ by $X_{i-1}^{2p-1}$ in $W$.  Because our algorithm
will be recursive, we let $\deg(V)$ be arbitrary; then, we have the
following estimate for $W$.

\begin{lemma}
  \label{th:push-down-degree} We have $\deg_Y(W)\le \deg(V)/p$.
\end{lemma}
\begin{proof}
  Consider the matrix $M$ of multiplication by $X_i^p$ modulo
  $X_i^p-X_i-Y$; it has entries in $\F_p[Y]$. Due to the sparseness of
  the modulus, one sees that $M$ has degree at most $1$, and so $M^k$
  has coefficients of degree at most $k$. Thus, the remainders of
  $X_i^{pk},\dots,X_i^{pk+p-1}$ modulo $X_i^p-X_i-Y$ have degree at
  most $k$ in $Y$.
\end{proof}


We compute $W$ by a recursive subroutine \hyperref[alg:push-down-rec]{\alg{Push-down-rec}}, similar
to \hyperref[alg:compose]{\alg{Compose}}. As before, we let $c,n$ be such that $1\le c<p$ and
$\deg(V) < (c+1)p^n$, so that we have
$$V=V_0+ V_1X_i^{p^n}+\cdots+V_c X_i^{cp^n},$$ with all $V_j$ in
$\F_p[X_i]$ of degree less than $p^n$. First, we recursively reduce
$V_0,\dots,V_c$ modulo $X_i^p-X_i-Y$, to obtain bivariate
polynomials $W_0,\dots,W_{c}$. Let $R_n$ be the polynomial defined in
Equation~\eqref{eq:Kn}. Then, we get $W$ by computing
$\Sigma_{j=0}^c W_j(X_i+R_n)^j$ modulo $X_i^p-X_i-Y$,
using Horner's scheme as in \hyperref[alg:compose]{\alg{Compose}}. Multiplications by
$X_i+R_n$ modulo $X_i^p-X_i-Y$ are done using \hyperref[eq:89]{\alg{MulMod}}.

\begin{algorithm}
  \caption{\label{alg:push-down-rec}\alg{Push-down-rec}}
  \begin{algorithmic}[1]
    \REQUIRE $V\in \F_p[X_i]$ and $c,n\in\N$.
    \ENSURE $W \in\F_p[Y,X_i]$.
    \STATE if $n=0$ return $V$;
    \STATE write $V=\sum_{j=0}^{c} V_j X_i^{jp^n}$, with $V_j \in \F_p[X_i], \deg V_j<p^n$;
    \STATE for $j\in [0,\dots,c]$, let $W_j=\text{\alg{Push-down-rec}}(V_j,p-1,n-1)$;
    \STATE $W=0$;
    \STATE\label{pd:loop} for $j\in [c,\dots,0]$, let $W = \hyperref[eq:89]{\alg{MulMod}_{(c+1)p^{n-1},n}}(W) + W_j$;
    \STATE return $W$.
  \end{algorithmic}
\end{algorithm}

\begin{algorithm}
  \caption{\label{alg:push-down}\alg{Push-down}}
  \begin{algorithmic}[1]
    \REQUIRE $v\wrt \U_i$.
    \ENSURE $v$ written as $v_0+\cdots+v_{p-1}x_i^{p-1}$ with $v_j \wrt \U_{i-1}$.
    \STATE let $V$ be the canonical preimage of $v$ in $\F_p[X_i]$;
    \STATE let $n=\lfloor \log_p(p^id-1) \rfloor$ and $c=(p^id-1)\text{ div } p^n$;
    \STATE let $W = \text{\hyperref[alg:push-down-rec]{\alg{Push-down-rec}}}(V,c,n)$;
    \STATE let $Z = \alg{Evaluate}(W,[X_{i-1}^{2p-1},X_i])$;
    \STATE \label{step:pd:mod} let $Z = Z \bmod Q_{i-1}$;
    \STATE \label{step:pd:return} return the residue class of $Z \bmod (X_i^p - X_i - X_{i-1}^{2p-1},Q_{i-1})$.
  \end{algorithmic}
\end{algorithm}

\begin{proposition}\label{prop:pd}
  Algorithm \hyperref[alg:push-down]{\alg{Push-down}} is correct and takes time 
  \begin{equation}
    \label{eq:92}
    O(p^{i+1}d
    \log_p(p^id)^2 + p\,\Mult(p^id))
    \text{.}
  \end{equation}
\end{proposition}
\begin{proof}
  Correctness is straightforward; note that at step~\ref{pd:loop} of
  \hyperref[alg:push-down-rec]{\alg{Push-down-rec}}, $\deg_Y(W) < (c+1)p^{n-1}$, so our call to
  $\hyperref[eq:89]{\alg{MulMod}_{(c+1)p^{n-1},n}}$ is justified. By
  the claim of Subsection~\ref{ssec:mulmod} on the cost of
  $\hyperref[eq:89]{\alg{MulMod}}$, the total time spent in that loop is
  $O(nc^2p^n)$. As in Theorem~\ref{theo:comp}, we deduce that the time
  spent in \hyperref[alg:push-down-rec]{\alg{Push-down-rec}} is $O(n^2c^2p^n)$.

  In \hyperref[alg:push-down]{\alg{Push-down}}, we have $cp^n< p^id$ and $n<\log_p (p^id)$, so
  the previous cost is seen to be $O(p^{i+1}d
  \log_p(p^id)^2)$. Reducing one coefficient of $Z$ modulo $Q_{i-1}$
  takes time $O(\Mult(p^id))$, so step~\ref{step:pd:mod} has cost
  $O(p\,\Mult(p^id))$. Step~\ref{step:pd:return} is free, since at
  this stage $Z$ is already reduced. 
\end{proof}

%%%%%%%%%%%%%%%%%%%%%%%%%%%%%%%%%%%%%%%%%%%%%%%%%%%%%%%%%%%%
%%%%%%%%%%%%%%%%%%%%%%%%%%%%%%%%%%%%%%%%%%%%%%%%%%%%%%%%%%%%
%%%%%%%%%%%%%%%%%%%%%%%%%%%%%%%%%%%%%%%%%%%%%%%%%%%%%%%%%%%%

\subsection{Transposed push-down}

Before giving the details for \hyperref[alg:liftup]{\alg{Lift-up}}, we discuss here the
transpose of \hyperref[alg:push-down]{\alg{Push-down}}.  As in
Section~\ref{sec:from-univ-bivar}, \hyperref[alg:push-down]{\alg{Push-down}} is the same thing
as the map
\begin{equation}
  \label{eq:81}
  \begin{aligned}
    \ev_{x_i}:\F_p[T] &\ra \U_i^{\basis{D}_i}\text{,}\\ 
    g&\mapsto g(x_i)\text{.}
  \end{aligned}
\end{equation}
So its transpose is the map
\begin{equation}
  \label{eq:82}
  \begin{aligned}
    \proj_{x_i}:(\dual{\U_i})^{\dual{\basis{D_i}}}&\ra\F_p[[1/T]]\text{,}\\
    \ell&\mapsto\sum_{j\ge0}\frac{\ell(x_i^j)}{T^j}\text{.}
  \end{aligned}
\end{equation}

\hyperref[alg:push-down]{\alg{Push-down}} is an \hyperref[sec:r-algebraic-transforms]{algebraic
  transform}, thus, applying Theorem~\ref{th:tellegen-R-algeb}, the
transposed algorithm is obtained by reversing the initial algorithm
step by step, and replacing subroutines by their transposes. The
overall cost remains the same; we review here the main
transformations.

As usual, we identify the dual of the space $\F_p[Y,X_i]$ to
$\F_p[[1/Y,1/X_i]]$. Thus linear forms given as input to the algorithm
are written as series
\begin{equation}
  \label{eq:84}
  L=\sum_{a,b\ge0}\frac{\ell_{a,b}}{Y^aX_i^b}
  \text{.}
\end{equation}
We do the same for $\F_p[X_i]$ and $\F_p[X_{i-1},X_i]$.

The initial loop at step~\ref{pd:loop} is a Horner scheme; the
transposed loop is run backward, and its core becomes $L_j=L\bmod
Y^{1-n}$ and
$L=\dual{\hyperref[eq:89]{\alg{MulMod}_{(c+1)p^{n-1},n}}}(L)$; a small
simplification yields the pseudo-code we give.  In \hyperref[alg:push-down]{\alg{Push-down}},
after calling \hyperref[alg:push-down-rec]{\alg{Push-down-rec}}, we evaluate $W$ at
$[X_{i-1}^{2p-1},X_i]$: the transposed operation $\dual{{\alg{Evaluate}}}$ is the map
\begin{equation}
  \label{eq:83}
  \sum_{a,b} \frac{\ell_{a,b}}{X_{i-1}^a X_i^b} \mapsto
  \sum_{a,b} \frac{\ell_{(2p-1)a,b}}{Y^a X_i^b}
  \text{.} 
\end{equation}
Then, originally, we perform a Euclidean division by $Q_{i-1}$ on
$Z$. The \index{transposed~modular~reduction}transposed algorithm
$\dual{\bmod}$ amounts to compute the values of a sequence linearly
generated by the polynomial $Q_{i-1}$ from its first $p^{i-1}d$ values
(see Section~\ref{sec:transp-eucl-divis}).

\begin{algorithm}
  \caption{\label{alg:push-down-rec-star}$\dual{\text{\alg{Push-down-rec}}}$}
  \begin{algorithmic}[1]
    \REQUIRE $L\in\F_p[[1/Y,1/X_i]]$ and $c,n\in\N$.
    \ENSURE $M\in \F_p[[1/X_i]]$.
    \STATE If $n=0$ return $L$;
    \FORALL{\label{pdt:loop} $j\in [c,\dots,0]$}
    \STATE let $L_j = L \bmod Y^{1-n}$;
    \STATE let $M_j=\dual{\text{\alg{Push-down-rec}}}(L_j,p-1,n-1)$;
    \STATE let $L = \dual{\hyperref[eq:89]{\alg{MulMod}_{(c+1)p^{n-1},n}}}(L)$;
    \ENDFOR
    \STATE return $\sum_{j=0}^{c} \frac{M_j}{X_i^{jp^n}}$.
  \end{algorithmic}
\end{algorithm}

\begin{algorithm}
  \caption{\label{alg:push-down-star}$\dual{\text{\alg{Push-down}}}$}
  \begin{algorithmic}[1]
    \REQUIRE $L\in \F_p[[1/X_{i-1},1/X_i]]$.
    \ENSURE $M \in \F_p[[1/T]]$.
    \STATE let $n=\lfloor \log_p(p^id-1) \rfloor$ and $c=(p^id-1) \text{ div } p^n$;
    \STATE let $P=\dual{\bmod}(L,Q_{i-1})$;
    \STATE let $M = \dual{\alg{Evaluate}}(P,[X_{i-1}^{2p-1},X_i])$;
    \STATE return $\dual{\text{\hyperref[alg:push-down-rec-star]{\alg{Push-down-rec}}}}(M,c,n)$;
  \end{algorithmic}
\end{algorithm}


%%%%%%%%%%%%%%%%%%%%%%%%%%%%%%%%%%%%%%%%%%%%%%%%%%%%%%%%%%%%
%%%%%%%%%%%%%%%%%%%%%%%%%%%%%%%%%%%%%%%%%%%%%%%%%%%%%%%%%%%%
%%%%%%%%%%%%%%%%%%%%%%%%%%%%%%%%%%%%%%%%%%%%%%%%%%%%%%%%%%%%

\subsection{Lift-up}
\label{sec:level-embedding:lift-up}

Let $v$ be given on the basis $\basis{D}_i$ and $W$ its canonical
preimage in $\F_p[X_{i-1},X_i]$.  The lift-up algorithm finds $V$ in
$\F_p[X_i]$ such that
\begin{equation}
  \label{eq:93}
  W=V \bmod
  (X_i^p-X_i-X_{i-1}^{2p-1},Q_{i-1})
\end{equation}
and outputs the residue class of $V$ modulo $Q_i$. Hereafter, we
assume that both $Q_i'^{-1} \bmod Q_i$ and the values $\rho_i(1)$ of
the trace $\Tr_{\U_i/\F_p}$ on the basis $\basis{D}_i$ are known.  See
the discussion below.

\paragraph{Lift-up}
We use Proposition~\ref{th:uni-multi-uni} to write $v$ as a
polynomial in $x_i$. To do this we proceed as in steps~\ref{alg:rur:4}
and~\ref{alg:rur:5} of \hyperref[alg:rur]{\alg{RUR}}.  To compute the power projection we
could use transposed bivariate modular composition as
in~\cite{shoup99}; it is however more efficient to use
\hyperref[alg:push-down-star]{$\dual{\alg{Push-down}}$}.

\begin{algorithm}
  \caption{\label{alg:liftup}\alg{Lift-up}}
  \begin{algorithmic}[1]
    \REQUIRE $v$ written as $v_0+\cdots+v_{p-1}x_i^{p-1}$ with $v_j \wrt \U_{i-1}$.
    \ENSURE $v\wrt \U_i$.
    \STATE \label{alg:lift-up:transmul} let $\ell = \alg{TransposedMul}(v,\,\rho_i(1))$;
    \STATE \label{alg:lift-up:pow} let $M=\frac{1}{T}\hyperref[alg:push-down-star]{\dual{{\text{\alg{Push-down}}}}}(\ell)$;
    \STATE \label{alg:lift-up:mult} let $V = Q_iM \bmod T^{p^id}$;
    \STATE \label{alg:lift-up:mulmod} return $v=V(x_i)Q_i(x_i)^{-1} = V {Q_i'}^{-1} \bmod Q_i$.
  \end{algorithmic}
\end{algorithm}

\begin{proposition}\label{prop:lu}
  Algorithm \hyperref[alg:liftup]{\alg{Lift-up}} is correct and takes time
  \begin{equation}
    \label{eq:85}
    O(p^{i+1}d\log_p(p^id)^2+p\,\Mult(p^{i}d))
    \text{.}    
  \end{equation}
\end{proposition}
\begin{proof}
  Correctness is a consequence of Theorem~\ref{th:rur} and of the
  algorithm given in Section~\ref{sec:from-univ-bivar}.

  \alg{TransposedMul} implements the
  \index{transposed~modular~multiplication}transposed bivariate
  modular multiplication; an algorithm of cost $O(\Mult(p^id))$ for
  this is in~\cite[Corollary~2]{pascal+schost06} (see also
  Section~\ref{sec:transp-algor}).  The last subsection showed that
  step~\ref{alg:lift-up:pow} has the same cost as
  \hyperref[alg:push-down]{\alg{Push-down}}. Then, the costs of steps~\ref{alg:lift-up:mult}
  and~\ref{alg:lift-up:mulmod} are $O(\Mult(p^id))$.
\end{proof}

Propositions~\ref{prop:pd} and~\ref{prop:lu} prove
Theorem~\ref{theo:L}.


\paragraph{Precomputations}
The precomputations, that are done at the construction of $\U_i$, are
as follows.  First, we need the values of the trace on the basis
$\basis{D}_i$. By~\eqref{eq:trcomp} we know that
\begin{equation}
  \label{eq:86}
  \Tr_{\U_i/\F_p}(x_{i-1}^ax_i^b) = 
  \Tr_{\U_{i-1}/\F_p}\circ\Tr_{\U_i/\U_{i-1}}(x_{i-1}^ax_i^b)
  \text{,}
\end{equation}
then, by \eqref{eq:pd}
\begin{equation}
  \label{eq:87}
  \Tr_{\U_i/\U_{i-1}}(x_{i-1}^ax_i^b) =
  \begin{cases}
    0 &\text{for $0\le b < p-1$,}\\
    -x_{i-1}^a &\text{for $b=p-1$.}
  \end{cases}
\end{equation}
Thus the values of $\Tr_{\U_i/\F_p}$ on the basis $\basis{D}_i$ are 
\begin{equation}
  \label{eq:88}
  0, \ldots, 0, -\Tr_{\U_{i-1}/\F_p}(1), -\Tr_{\U_{i-1}/\F_p}(x_{i-1}), \ldots, -\Tr_{\U_{i-1}/\F_p}(x_{i-1}^{p^{i-1}d-1})
  \text{.}
\end{equation}
They can be computed in time $O(\Mult(p^{i-1}d))$ using
Lemma~\ref{th:multi-newton-sums}.

Then, we need ${Q_i'}^{-1} \bmod Q_i$; this takes time $O(\Mult(p^id)
\log(p^id))$ by fast extended GCD computation.  These precomputations
save logarithmic factors at best, but are useful in practice.


% Local Variables:
% mode:flyspell
% ispell-local-dictionary:"american"
% mode: TeX-PDF
% mode: reftex
% TeX-master: "../these"
% End:
%
% LocalWords:  Schreier Artin pseudotrace frobenius bivariate memoization
% LocalWords:  precomputations precomputation
