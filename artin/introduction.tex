In this chapter we give fast algorithms for arithmetic operations in
Artin-Schreier towers. Prior results for this task are due to
Cantor~\cite{cantor89} and Couveignes~\cite{couveignes00}. However,
the algorithms of~\cite{couveignes00} need as a prerequisite a fast
multiplication algorithm in some towers of a special kind, called
``Cantor towers'' in~\cite{couveignes00}. Such an algorithm is
unfortunately not in the literature, making the results
of~\cite{couveignes00} non practical. This chapter fills this gap.

To our knowledge, no previous work provided the missing ingredients to
put Couveignes' algorithms to practice. Part of Cantor's results were
independently discovered by Wang and Zhu~\cite{wang+zhu88} and have
been extended in another direction (fast polynomial multiplication
over arbitrary finite fields) by von zur Gathen and
Gerhard~\cite{vzgatehn+gerhard02} and Mateer~\cite{mateer08}.

Technically, the main algorithmic contribution of this chapter is a
fast change-of-basis algorithm based on the techniques of
Chapter~\ref{cha:trace-computations}. Building on this, it is possible
to obtain fast multiplication routines, and by extension completely
explicit versions of all algorithms of~\cite{couveignes00}. Along the
way, we also extend constructions of Cantor to the case of a general
finite base field $\U_0$, where Cantor had $\U_0=\F_p$.

The algorithms presented in this chapter have been integrated in a
\texttt{C++} library called \texttt{FAAST}, based on Shoup's
\texttt{NTL}~\cite{NTL} library. This chapter is joint work with
Schost~\cite{df+schost09}.

We start by giving the relevant definitions, and preliminaries. In
Section~\ref{sec:fast-tower}, we define a specific Artin-Schreier
tower, where arithmetic operations will be fast. Our key
change-of-basis algorithm for this tower is in
Section~\ref{sec:level-embedding}. In
Sections~\ref{sec:pseudotrace-frobenius}
and~\ref{sec:couveignes-algorithm}, we revisit Couveignes' algorithm
for isomorphism between Artin-Schreier towers~\cite{couveignes00} in
our context, which yields fast arithmetics for {\em any}
Artin-Schreier tower. Finally, Section~\ref{sec:benchmarks} presents
our implementation of the \texttt{FAAST} library and gives
experimental results.



\section{Artin-Schreier extensions}
\label{sec:artin-schr-extens}

If $\U$ is a field of characteristic $p$, polynomials of the form
\begin{equation}
  \label{eq:36}
  P=X^p - X - \alpha
  \text{,}  
\end{equation}
with $\alpha \in \U$ are called {\em Artin-Schreier polynomials}; a
field extension $\U'/\U$ is {\em Artin-Schreier} if it is of the form
$\U' = \U[X]/P$, with $P$ an Artin-Schreier polynomial.

\begin{theorem}
  \label{th:artin-schreier}
  The Artin-Schreier polynomial $X^p-X-\alpha$ is irreducible in $\U$
  if and only if
  \begin{equation}
    \label{eq:62}
    \Tr_{\U/\F_p}(\alpha) \ne 0
    \text{.}
  \end{equation}
  If it is reducible, then it is split and its roots are 
  \begin{equation}
    \label{eq:63}
    \eta, \eta+1, \ldots, \eta + p -1
    \text{.}
  \end{equation}
\end{theorem}
\begin{proof}
  See \cite[Theorem~2.25]{lidl+niederreiter:2}.
\end{proof}

An {\em Artin-Schreier tower} of height $k$ is a sequence of
Artin-Schreier extensions $\U_i / \U_{i-1}$, for $1\le i \le k$; it is
denoted by $(\U_0, \ldots, \U_k)$. In what follows, we only consider
extensions of finite degree over $\F_p$. Thus, $\U_i$ is of degree
$p^i$ over $\U_0$, and of degree $p^id$ over $\F_p$, with
$d=[\U_0:\F_p]$.

The importance of this concept comes from the fact that all Galois
extensions of degree $p$ are Artin-Schreier. As such, they arise
frequently, e.g., in number theory (for instance, when computing
$p^k$-torsion groups of Abelian varieties over $\F_p$). The need for
fast arithmetics in these towers is motivated in particular by
applications to isogeny computation and point-counting in cryptology,
as we will see in Chapter~\ref{cha:algor-small-char}.

We count all time complexities in number of operations in
$\F_p$. Then, notation being as before, optimal algorithms in $\U_k$
would have complexity $O(p^kd)$; most of our results are (up to
logarithmic factors) of the form $O(p^{k+\alpha} d^{1+\beta})$, for
small constants $\alpha,\beta$ such as $0,1,2$ or $3$.

For several operations, different algorithms will be available, and
their relative efficiencies can depend on the values of $p$, $d$ and
$k$. In these situations, we always give details for the case where
$p$ is small, since cases such as $p=2$ or $p=3$ are especially useful
in practice. Some of our algorithms could be slightly
improved, but we usually prefer giving the simpler solutions.




% Local Variables:
% mode:flyspell
% ispell-local-dictionary:"american"
% mode: TeX-PDF
% mode: reftex
% TeX-master: "../these"
% End:
%
% LocalWords:  Schreier Artin pseudotrace frobenius bivariate memoization
% LocalWords:  isogeny Couveignes
