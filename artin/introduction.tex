\section{Introduction}
\label{sec:introduction}
Artin-Schreier towers are important.

\begin{definition}[Artin-Schreier polynomial]
  \label{def:aspolynomial}
  Let $\K$ be a field of characteristic $p$ and $\alpha\in\K$. The polynomial
  \[X^p - X - \alpha\]
  is called an Artin-Schreier polynomial.
\end{definition}

\begin{proposition}
  \label{th:asfundamental}
  Let $\F_q$ be a finite field of characteristic $p$, the polynomial
  \[X^p - X - \alpha\]
  is irreducible if and only if $\Tr_{\F_q/\F_p}(\alpha) \ne 0$. If
  the polynomial has a root $\eta\in\F_q$, then it splits completely
  over $\F_q$ and its roots are $\eta, \eta + 1, \ldots, \eta +
  (p-1)$.
\end{proposition}
\begin{proof}
  \cite[Theorem 2.25]{LN}.
\end{proof}

\begin{definition}[Artin-Schreier extension]
  \label{def:asextension}
  Let $\K$ be a field and let $\AS{P}$ be an irreducible
  Artin-Schreier polynomial. The field extension $\LK/\K$ where $\LK =
  \K[X]/\AS{P}(X)$ is called an Artin-Schreier extension.
\end{definition}

We are interested in towers of Artin-Schreier extensions over the
field $\U_0=\F_{p^d}$, with $p\ndiv d$. For $k$ a non-negative
integer, an {\em Artin-Schreier tower} is a sequence of Artin-Schreier
extensions $\U_i / \U_{i-1}$, for $1\le i \le k$; it is denoted by
$(\U_0, \U_1, \ldots, \U_k)$. Remark that $\U_i$ is of degree $p^i$
over $\U_0$, and thus of degree $dp^i$ over $\F_p$.

We look for efficient algorithms for arithmetic operations in such a
tower. To measure them, we use the notion of {\em multiplication
  function}; we count all time complexities of our algorithms in
number of operations in $\F_p$ and all space complexities in number of
elements of $\F_p$.

\begin{definition}[Multiplication function]
  \label{def:mult}
  A function $\Mult : \N \rightarrow \N$ is a multiplication function
  if
  \begin{itemize}
  \item polynomials in $\F_p[X]$ of degree less than $n$ can be
    multiplied in $\Mult(n)$ operations in $\F_p$;
  \item $\Mult(n + n') \ge \Mult(n) + \Mult(n')$ holds for all $n,n'$.
  \end{itemize}
\end{definition}
Typical orders of magnitude for $\Mult(n)$ are $O(n^{\log_2 3})$ for
Karatsuba multiplication or $O(n\log n \log\log n)$ for FFT
multiplication.

Using Newton iteration and the Cook-Sieveking-Kung algorithm, one can
compute the quotient and remainder in the Euclidean division of two
polynomials in $\F_p[X]$ of degree at most $n$ in $O(\Mult(n))$
$\F_p$-operations. The extended GCD of two polynomials of degree at
most $n$ can be computed in time $O(n^2)$ by a naive algorithm;
half-GCD techniques yield algorithms of complexity $O(\Mult(n)\log
n)$.

We let $\ModComp(m)$ be the cost of the modular composition of
polynomials in $\F_p[X]$ of degree at most $m$: given $a,b,c$ in
$\F_p[X]$ of degree at most $m$, this operation returns $a(b) \bmod
c$.  A typical value for $\ModComp(m)$ is $O\left(\Mult(m)\sqrt{m} +
  m^\frac{\omega+1}{2}\right)$, where $\omega$ is the exponent of
linear algebra; this is the actual algorithm implemented in NTL.
Umans-Kedlaya!!

In all that follows, a multiplication function is fixed. Then, for
most operations, we achieve algorithms with complexity
$\tildO_{p,k,d}\left(\Mult(p^{k+a}k^{b}d^{c})\right)$ where $a,b,c$
are some (small) integer constants. We use the soft-$O$ notation
$\tildO_{x_1,\ldots,x_m}$ to ignore logarithmic factors in the
variables $x_1,\ldots,x_m$; we omit the variables names when they are
evident from the context.

% Local Variables:
% mode:flyspell
% ispell-local-dictionary:"british"
% End:
%
% LocalWords:  Schreier Artin pseudotrace frobenius bivariate memoization
