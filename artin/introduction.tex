\section{Introduction}

\paragraph*{\bf Definitions.} If $\U$ is a field of characteristic $p$,
polynomials of the form $P=X^p - X - \alpha$, with $\alpha \in \U$,
are called {\em Artin-Schreier polynomials}; a field extension
$\U'/\U$ is {\em Artin-Schreier} if it is of the form $\U' = \U[X]/P$,
with $P$ an Artin-Schreier polynomial.

An {\em Artin-Schreier tower} of height $k$ is a sequence of
Artin-Schreier extensions $\U_i / \U_{i-1}$, for $1\le i \le k$; it is
denoted by $(\U_0, \ldots, \U_k)$. In what follows, we only consider
extensions of finite degree over $\F_p$. Thus, $\U_i$ is of degree
$p^i$ over $\U_0$, and of degree $p^id$ over $\F_p$, with
$d=[\U_0:\F_p]$.

The importance of this concept comes from the fact that all Galois
extensions of degree $p$ are Artin-Schreier. As such, they arise
frequently, e.g., in number theory (for instance, when computing
$p^k$-torsion groups of Abelian varieties over $\F_p$). The need for
fast arithmetics in these towers is motivated in particular by
applications to isogeny computation and point-counting in cryptology,
as in~\cite{Couveignes96}.

\paragraph*{\bf Our contribution.} The purpose of this paper is to
give fast algorithms for arithmetic operations in Artin-Schreier
towers. Prior results for this task are due to Cantor~\cite{Can89} and
Couveignes~\cite{Couveignes00}. However, the algorithms
of~\cite{Couveignes00} need as a prerequisite a fast multiplication
algorithm in some towers of a special kind, called ``Cantor towers''
in~\cite{Couveignes00}. Such an algorithm is unfortunately not in the
literature, making the results of~\cite{Couveignes00} non practical.

This paper fills the gap. Technically, our main algorithmic
contribution is a fast change-of-basis algorithm; it makes it possible
to obtain fast multiplication routines, and by extension completely
explicit versions of all algorithms of~\cite{Couveignes00}. Along the
way, we also extend constructions of Cantor to the case of a general
finite base field $\U_0$, where Cantor had $\U_0=\F_p$.  We present
our implementation, in a library called \texttt{FAAST}, based on
Shoup's \texttt{NTL}~\cite{NTL}. As an application, we put to practice
Couveignes' isogeny computation algorithm~\cite{Couveignes96} (or,
more precisely, its refined version presented in~\cite{DeFeo10}).

\paragraph*{\bf Complexity notation.} We count time complexity
in number of operations in $\F_p$. Then, notation being as before,
optimal algorithms in $\U_k$ would have complexity $O(p^kd)$; most of
our results are (up to logarithmic factors) of the form
$O(p^{k+\alpha} d^{1+\beta})$, for small constants $\alpha,\beta$ such as
$0,1,2$ or $3$.

Many algorithms below rely on fast multiplication; thus, we let $\Mult
: \N \rightarrow \N$ be a {\em multiplication function}, such that
polynomials in $\F_p[X]$ of degree less than $n$ can be multiplied in
$\Mult(n)$ operations, under the conditions of~\cite[Ch.~8.3]{vzGG}.
Typical orders of magnitude for $\Mult(n)$ are $O(n^{\log_2(3)})$ for
Karatsuba multiplication or $O(n\log (n) \log\log (n))$ for FFT
multiplication. Using fast multiplication, fast algorithms are
available for Euclidean division or extended GCD~\cite[Ch.~9 \&
11]{vzGG}.

The cost of {\em modular composition}, that is, of computing $F(G)
\bmod H$, for $F,G,H\in\F_p[X]$ of degrees at most $n$, will be
written $\ModComp(n)$. We refer to~\cite[Ch.~12]{vzGG} for a
presentation of known results in an algebraic computational model: the
best known algorithms have subquadratic (but superlinear) cost in
$n$. Note that in a boolean RAM model, the algorithm of~\cite{KeUm08}
takes quasi-linear time.

For several operations, different algorithms will be available, and
their relative efficiencies can depend on the values of $p$, $d$ and
$k$. In these situations, we always give details for the case where
$p$ is small, since cases such as $p=2$ or $p=3$ are especially useful
in practice. Some of our algorithms could be slightly
improved, but we usually prefer giving the simpler solutions.

\paragraph*{\bf Previous work.} As said above, this paper
builds on former results of Cantor~\cite{Can89} and
Couveignes~\cite{Couveignes00,Couveignes96}; to our knowledge, prior
to this paper, no previous work provided the missing ingredients to
put Couveignes' algorithms to practice. Part of Cantor's
  results were independently discovered by Wang and Zhu~\cite{WaZh88}
and have been extended in another direction (fast polynomial
multiplication over arbitrary finite fields) by von zur Gathen and
Gerhard~\cite{GaGe96} and Mateer~\cite{GaMa08}.

This paper is an expanded version of the conference
paper~\cite{DeSc09}. We provide a more thorough description of the
properties of Cantor towers (Section~\ref{sec:fast-tower}),
improvements to some algorithms (e.g. the Frobenius or pseudo-trace
computations) and a more extensive experimental section.

\paragraph*{\bf Organization of the paper.}
Section~\ref{sec:arithmetics} consists in preliminaries: trace
computations, duality, basics on Artin-Schreier extensions. In
Section~\ref{sec:fast-tower}, we define a specific Artin-Schreier
tower, where arithmetic operations will be fast. Our key
change-of-basis algorithm for this tower is in
Section~\ref{sec:level-embedding}. In
Sections~\ref{sec:pseudotrace-frobenius}
and~\ref{sec:couveignes-algorithm}, we revisit Couveignes' algorithm
for isomorphism between Artin-Schreier towers~\cite{Couveignes00} in
our context, which yields fast arithmetics for {\em any}
Artin-Schreier tower. Finally, Section~\ref{sec:benchmarks} presents
our implementation of the \texttt{FAAST} library and gives
experimental results obtained by applying our algorithms to
Couveignes' isogeny algorithm~\cite{Couveignes96} for elliptic curves.




% Local Variables:
% mode:flyspell
% ispell-local-dictionary:"american"
% mode: TeX-PDF
% mode: reftex
% TeX-master: "../these"
% End:
%
% LocalWords:  Schreier Artin pseudotrace frobenius bivariate memoization
% LocalWords:  isogeny Couveignes
