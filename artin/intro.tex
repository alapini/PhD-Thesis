The goal of this part of the document is to develop efficient
algorithms to compute in some finite dimensional algebras over a field
$\K$. We start by reviewing the generic techniques to compute modulo
$0$-dimensional ideals in this chapter.

Let $\K$ be a field, and let $x_1,\ldots,x_n$ be indeterminates. We
denote by $\K[\lst{x}]$ the algebra $\K[x_1,\ldots,x_n]$. Any finite
dimensional $\K$-algebra $\algeb{A}$ is isomorphic to a quotient
$\K[\lst{x}]/I$ for some $0$-dimensional ideal $I$.

Residue classes of $\K[\lst{x}]$ modulo an ideal $I$ are indeed a very
good representation of the elements of $\algeb{A}$. However, different
choices for $I$ can have different impacts on the efficiency of the
algorithms. Consider, for example, the ideal of $\Q[x,y]$
\begin{equation}
  \label{eq:example-x<y}
  (x^2 + x + 1, y^3 - x)
  \text{,}
\end{equation}
another set of generators for the same ideal is
\begin{equation}
  \label{eq:example-y<x}
  (y^6 + y^3 + 1, x - y^3)
  \text{.}
\end{equation}
Both sets of generators are Gröbner bases of $I$ and identify
$\Q[x,y]/I$ to $\Q(\zeta_9)$. However, while \eqref{eq:example-x<y}
naturally identifies $\Q[x]/(x^2+x+1)$ to the subfield
$\Q(\zeta_3)\subset\Q(\zeta_9)$, this information is lost by
\eqref{eq:example-y<x}, making it harder to test for appartenence to
$\Q(\zeta_3)$ in this case.



% Local Variables:
% mode:flyspell
% ispell-local-dictionary:"american"
% mode:TeX-PDF
% mode:reftex
% TeX-master: "../these"
% End:

% LocalWords:  indeterminates
