\section{A \emph{nice} tower}
\label{sec:fast-tower}

This section is devoted to the proof of the following theorem which
extends a result by Cantor~\cite{Can89}.

\begin{theorem}
  \label{th:cantor}
  Let $d=[\U_0 :\F_p]$ be prime to $p$ and let $x_0$ be a generator of
  $\U_0$ over $\F_p$ such that $\Tr_{\U_0/\F_p}(x_0)\ne0$. Define
  $\gamma_i$, $\AS{P}_i$ and $x_i$ as
  \begin{eqnarray*}
    \gamma_0 &=& x_0\\
    \gamma_1 &=& 
    \begin{cases}
      x_1        &\text{if $p=2$,}\\
      x_1^{2p-1}      &\text{if $p>2$,}
    \end{cases} \\
    \gamma_i &=&  x_i^{2p-1} \quad i \ge 2\\
    \\
    \AS{P}_{i}(X) &=& X^p - X - \gamma_i\\
    x_{i+1} &&\text{ a root of } \AS{P}_i
  \end{eqnarray*}
  Then, for $i=0,\dots,k-1$, the polynomial $\AS{P}_i$ is irreducible
  over $\U_i$ and $x_{i+1}$ generates $\U_{i+1}$ over $\F_p$.
\end{theorem}

In \cite[Theorem 1.2]{Can89} Cantor proves the case $\U_0 = \F_p$ and
$x_0=1$. His statement is more generic than ours because it asks for
less restrictive conditions on the $\gamma_i$'s, but such greater
generality has to be paid off with a more complicated proof. We would
like to discuss the relationship between his proof and ours.

In order to prove his theorem, Cantor introduces the linear operator
$S:x\mapsto x^p-x$, remarks that its kernel is $\F_p$ and studies the effects
of its application on the $x_i$'s. In order to extend such a technique
to the case $\U_0\ne\F_p$, we need consider a similar operator and we
want its kernel to be $\U_0$. A natural choice is then
\[S:x\mapsto x^{p^d}-x\text{.}\]
As in Cantor's proof, in order to study the action of $S$ one needs to
see $\U_k$ as a vector space over $\U_0$ and consider some linear
subspaces that are not necessarily fields. There is a way of getting
rid of such an abundance of linear subspaces by considering only some
iterated applications of the operator $S$. Indeed
\[S^{p^i-p^j} = \PTr_{(p^i,dp^j)}\]
where $\PTr$ is the operator we will introduce in definition
\ref{def:pseudotrace}. This way, the only subspaces of $\U_k$ one
needs to consider are the $\U_i$'s and the proof is made extremely
simple by the particular conditions we asked on the $\gamma_i$'s.

We now give the proof of the theorem. We start with a classic lemma.

\begin{lemma}
  \label{Lemma:trace-AS}
  Let $\LK/\K$ be an Artin-Schreier extension and let $x$ be a
  generator of $\LK$ over $\K$ whose minimal polynomial is
  Artin-Schreier, then
  \begin{equation*}
    \Tr_{\LK/\K}(x^i) = \begin{cases}
      0  & \text{if $i=0,\ldots,p-2$,}\\
      -1 & \text{if $i=p-1$.}
    \end{cases}
  \end{equation*}
\end{lemma}
\begin{proof}
  Consider the formal power series of $\Tr_{\LK/\K}(x^i)$
  \begin{equation*}
    \sum_{i\ge0}\Tr(x^i)T^i = \sum_{i\ge0}\sum_{\zeta\in B}\zeta^iT^i
    \text{,}
  \end{equation*}
  where $B$ is the set of the conjugates of $x$ over $\K$ and the
  equality follows from the fact that the conjugation distributes over
  the product. By swapping sums
  \begin{equation*}
    \sum_{i\ge0}\Tr(x^i)T^i = \sum_{\zeta\in B}\frac{1}{1-\zeta T} =
    \frac{\sum_{\zeta\in B}\prod_{\zeta'\ne\zeta}(1-\zeta'T)}
         {\prod_{\zeta\in B}(1-\zeta T)} \text{.}
  \end{equation*}
  
  Let $\AS{P} = X^p-X-\alpha$ be the minimal polynomial of $x$ over
  $\K$, remark the following equalities
  \begin{equation*}
    \AS{P}(T) = \prod_{\zeta\in B}(T-\zeta) \;\text{,} \qquad
    \frac{\partial\AS{P}}{\partial X}(T) =
    \sum_{\zeta\in B}\prod_{\zeta'\ne\zeta}(T-\zeta') = -1\text{.}
  \end{equation*}
  Notice that the two polynomials have degrees in $T$ respectively $p$
  and $p-1$. Then, by taking reverse polynomials (see also section
  \ref{sec:level-embedding}, definition \ref{def:rev}), we have
  \begin{equation*}
    \sum_{i\ge0}\Tr(x^i)T^i =
    \frac{T^{p-1}\frac{\partial\AS{P}}{\partial X}(\frac{1}{T})}
      {T^p\AS{P}(1/T)} = \frac{-T^{p-1}}{1 - T^{p-1} - \alpha T^p}
      \text{.}
  \end{equation*}
  
  Hence we see that the first non-null coefficient of the formal power
  series is the $(p-1)$-th one and its value is exactly $-1$.
\end{proof}

We first address the irreducibility of $\AS{P}_i$.

\begin{lemma}
  \label{Lemma:irreducibility}
  For $i=0,\dots,k-1$ the polynomial $\AS{P}_i$ is
  irreducible and $x_{i+1}$ generates $\U_{i+1}$ over $\U_i$.
\end{lemma}
\begin{proof}
  We will show by induction on $i$ that $\Tr_{\U_i/\F_p}\ne0$ for
  every $i$, then, by theorem \ref{th:asfundamental}, $\AS{P}_i$ is
  irreducible and $x_{i+1}$ generates $\U_{i+1}$ over $\U_i$.
  
  For $i=0$ this is true by hypothesis. For $i \ge 1$, $x_i$ generates
  $\U_i$ over $\U_{i-1}$ by induction hypothesis and its minimal
  polynomial $\AS{P}_{i-1}$ is evidently Artin-Schreier. Then $x_i$
  meets the hypotheses of lemma \ref{Lemma:trace-AS}.

  Suppose that $p\ge3$, then
  \begin{equation*}
    \gamma_i=x_i^{2p-1}  = x_i^px_i^{p-1} =
    \gamma_{i-1} + x_i + x_i^{p-1} \gamma_{i-1}\text{.}
  \end{equation*}
  By the linearity of the trace and by lemma \ref{Lemma:trace-AS}
  \begin{equation}
    \label{eq:trace>2}
    \Tr_{\U_i/\U_{i-1}}(\gamma_i) = -\gamma_{i-1}
  \end{equation}
  and by the composition of traces $\Tr_{\U_i/\F_p}(\gamma_i) =
  -\Tr_{\U_{i-1}/\F_p}(\gamma_{i-1})$ is different from $0$ by induction
  hypothesis.

  When $p=2$, and $i=1$, using the same technique yields
  \begin{equation}
    \label{eq:trace=2,1}
    \Tr_{\U_1/\F_2}(\gamma_1)= \Tr_{\U_1/\F_2}(x_1)=
    \Tr_{\U_0/\F_2}(\Tr_{\U_1/\U_0}(x_1))= \Tr_{\U_0/\F_2}(1)\text{,}
  \end{equation}
  which is different from $0$ because $[\U_0:\F_2]$ is odd.

  For $p=2$ and $i > 1$, we get $x_i^{3} = \gamma_{i-1} + x_i (1+
  \gamma_{i-1})$ and as a consequence
  \begin{equation}
    \label{eq:trace=2}
    \Tr_{\U_i/\U_{i-1}}(\gamma_i)=\Tr(x_i^3)=1+\gamma_{i-1}
    \text{.}
  \end{equation}
  Since $\U_{i-1}$ has even degree over $\F_2$, we deduce
  $\Tr_{\U_{i-1}/\F_2}(1)=0$, and thus
  \[\Tr_{\U_{i}/\F_2}(\gamma_i)=\Tr_{\U_{i-1}/\F_2}(\gamma_{i-1})\]
  and we conclude by induction.
\end{proof}

As a consequence of the proof of the lemma, we have the following

\begin{corollary}
  \label{coro:trace}
  For $0\le j < i \le k-1$ we have
  \begin{equation*}
    \Tr_{\U_i/\U_{j}}(\gamma_i) = \begin{cases}
      (-1)^{i-j}\gamma_{j}  &\text{if $p\ge3$,}\\
      1+\gamma_{j}         &\text{if $p=2$ and $j\ge1$,}\\
      1                    &\text{if $p=2$ and $j=0$.}
    \end{cases}
  \end{equation*}
\end{corollary}
\begin{proof}
  The case $p\ge3$ follows from equation \eqref{eq:trace>2} by
  induction on $j$. The case $p=2$ follows from equations
  \eqref{eq:trace=2,1} and \eqref{eq:trace=2} and the fact that
  $\Tr_{\U_i/\F_p}(1)=0$ for $i\ge1$.
\end{proof}

We have just showed that $\U_{i+1} = \F_p[x_0,\ldots,x_i]$; the next
step is to show $\U_{i+1}=\F_p[x_i]$. To do so we introduce some more
notation.

\begin{definition}[Pseudotrace]
  \label{def:pseudotrace}
  The $i$-th pseudotrace of order $j$ is the linear operator
  defined by
  \begin{equation*}
    \PTr_{(i,j)} \;:\; x \mapsto \sum_{l=0}^{i-1}x^{p^{jl}}\text{.}
  \end{equation*}
  When $j=1$ we simply call it the $i$-th pseudotrace and we note
  $\PTr_i$.
\end{definition}

Notice that the pseudotrace is defined by the polynomial
$\PTr_{(i,j)}(X)$ in $\F_p[X]$. It is a ``linearized'' polynomial and
it has many special properties, the most important of which is being
$\F_{p^j}$-linear in its argument. See \cite[3.4]{LN} for more
details. It is also important to notice that $\PTr_{(p^i,dp^j)}$
coincides with $\Tr_{\U_{i+j}/\U_j}$ over elements of $\U_{i+j}$. For
the sake of convenience, we will note $\PTr_{\U_{i+j}/\U_{j}}$ instead
of $\PTr_{(p^i,dp^j)}$.

\begin{lemma}
  \label{Lemma:primitive>2}
  If $p\ge3$, $x_i$ generates $\U_i$ over $\F_p$ for $i=0,\ldots,k$.
\end{lemma}
\begin{proof}
  Induction on $i$. For $i=0$ this is true by hypothesis.

  For $i\ge1$, $x_i$ generates $\U_i$ over $\U_{i-1}$ by lemma
  \ref{Lemma:irreducibility} and $x_{i-1}$ generates $\U_{i-1}$ over
  $\F_p$ by induction hypothesis, so that $\U_i =
  \F_p[x_{i-1},x_i]$. Then it suffices to show that there is a
  polynomial $P\in\F_p[X]$ such that $P(x_i) = x_{i-1}$.

  For $i=1$ we know by definition that $x_1^p-x_1=x_0$, so that the
  polynomial $X^p-X$ does the trick.

  For $i\ge2$, consider the following equality, given by the linearity
  of the pseudotrace
  \begin{equation*}
    \PTr_{\U_{i-1}/\U_{i-2}}(x_i)^p - \PTr_{\U_{i-1}/\U_{i-2}}(x_i) =
    \PTr_{\U_{i-1}/\U_{i-2}}(x_i^p-x_i)\text{.}
  \end{equation*}
  We know by definition that $x_i^p-x_i=\gamma_{i-1}$, but
  $\gamma_{i-1}$ is in $\U_{i-1}$, so that the pseudotrace coincides
  with the trace as remarked above. We have
  \begin{equation*}
    \PTr_{\U_{i-1}/\U_{i-2}}(x_i)^p - \PTr_{\U_{i-1}/\U_{i-2}}(x_i) =
    \Tr_{\U_{i-1}/\U_{i-2}}(\gamma_{i-1}) = -\gamma_{i-2} \text{,}
  \end{equation*}
  where the last equality comes from corollary \ref{coro:trace}.

  We have shown that $\PTr_{\U_{i-1}/\U_{i-2}}(x_i)$ is a root of
  ${X^p-X+\gamma_{i-2}}$. But ${x_{i-1}^p-x_i=\gamma_{i-2}}$, hence we
  have the equality
  \begin{equation*}
    \PTr_{\U_{i-1}/\U_{i-2}}(x_i) = -x_{i-1}+\beta_i
    \qquad\text{with $\beta_i\in\F_p$.}
  \end{equation*}
  We conclude by observing that the polynomial
  $-\PTr_{\U_{i-1}/\U_{i-2}}(X)+\beta_i$ is in $\F_p[X]$ and sends
  $x_i$ over $x_{i-1}$.
\end{proof}

The case $p=2$ uses the same technique, but requires a bit more
work. Recall that $\F_4$ is the field $\{0,1,\omega,\omega+1\}$, where
$\omega$ and $\omega+1$ are two primitive cubic roots of the unit and
their minimal polynomial is $X^2-X-1$. Notice that $\F_4$ is not a
subfield of $\U_0$ because $d$ is odd, but it is a subfield of $\U_1$.

\begin{lemma}
  \label{Lemma:primitive=2-F4}
  If $p=2$, $x_i$ generates $\U_i$ over $\F_4$ for $i=1,\ldots,k$.
\end{lemma}
\begin{proof}
  Induction on $i$. For $i=1$, $\U_1 = \F_2[x_0,x_1]$, but
  $x_1^2-x_1=x_0$, so $\U_1 = \F_2[x_1]$. Since $\F_4\subset\U_1$,
  $x_1$ generates $\U_1$ over $\F_4$.
  
  For $i=2$, $\U_2=\F_4[x_1,x_2]$ by lemma \ref{Lemma:irreducibility}
  and induction hypothesis. But $x_2^2-x_2=x_1$, so that
  $\U_2=\F_4[x_2]$.

  For $i\ge3$, like in lemma \ref{Lemma:primitive>2},
  \begin{equation*}
    \PTr_{\U_{i-1}/\U_{i-2}}(x_i)^2 - \PTr_{\U_{i-1}/\U_{i-2}}(x_i) =
    \Tr_{\U_{i-1}/\U_{i-2}}(\gamma_{i-1}) = 1+\gamma_{i-2}
  \end{equation*}
  so that $\PTr_{\U_{i-1}/\U_{i-2}}(x_i)$ is a root of
  $X^2-X-(1+\gamma_{i-2})$. We deduce
  \[\PTr_{\U_{i-1}/\U_{i-2}}(x_i) = x_{i-1} + \omega_i\text{,}\]
  where $\omega_i$ is one of the two primitive roots of the unity,
  hence in $\F_4$. Then $\PTr_{\U_{i-1}/\U_{i-2}}(X) - \omega_i$ is in
  $\F_4[X]$ and sends $x_i$ over $x_{i-1}$.
\end{proof}

\begin{lemma}
  \label{Lemma:primitive=2}
  If $p=2$, $x_i$ generates $\U_i$ over $\F_2$ for $i=0,\ldots,k$.
\end{lemma}
\begin{proof}
  For $i=0$ this is true by hypothesis and for $i=1$ we already showed
  in the proof of lemma \ref{Lemma:primitive=2-F4} that
  $\U_1=\F_2[x_1]$.

  For $i\ge2$ we know by lemma \ref{Lemma:primitive=2-F4} that
  $\U_i=\F_2[x_i,\omega]$ where $\omega$ is a primitive root of
  unity. We then look for a polynomial in $\F_2[X]$ that sends $x_i$
  over $\omega$.

  By the same technique as in lemmas \ref{Lemma:primitive>2} and
  \ref{Lemma:primitive=2-F4}
  \begin{equation*}
    \PTr_{\U_{i-1}/\U_0}(x_i)^2-\PTr_{\U_{i-1}/\U_0}(x_i) = 
    \Tr_{\U_{i-1}/\U_0}(\gamma_{i-1}) = 1 \text{,}
  \end{equation*}
  where the last equality comes from corollary \ref{coro:trace}.
  
  Then $\PTr_{\U_{i-1}/\U_0}(x_i)$ is a root of $X^2-X-1$, hence
  \begin{equation*}
    \PTr_{\U_{i-1}/\U_0}(x_i) = \omega + \beta_i
    \qquad\text{with $\beta_i\in\F_2$.}
  \end{equation*}
  Then the polynomial $\PTr_{\U_{i-1}/\U_0}(X) - \beta_i$ is in
  $\F_2[X]$ and sends $x_i$ over $\omega$.
\end{proof}


% Local Variables:
% mode:flyspell
% ispell-local-dictionary:"british"
% End:
%
% LocalWords:  Schreier Artin pseudotrace frobenius bivariate memoization
