\section{A primitive tower}
\label{sec:fast-tower}

Our first task in this section is to describe a specific
Artin-Schreier tower where arithmetics will be fast; then, we explain
how to construct this tower. 

%%%%%%%%%%%%%%%%%%%%%%%%%%%%%%%%%%%%%%%%%%%%%%%%%%%%%%%%%%%%

\subsection{Definition}

The following theorem extends results by
Cantor~\cite[Theorem~1.2]{cantor89}, who dealt with the case
$\U_0=\F_p$.

\begin{theorem}
  \label{th:cantor}
  Let $\U_0=\F_p[X_0]/Q_0$, with $Q_0$ irreducible of
  degree $d$, let $x_0 = X_0 \bmod Q_0$ and assume that
  $\Tr_{\U_0/\F_p}(x_0)\ne0$. Let $(G_i)_{0 \le i <k}$ be defined by
$$ \begin{cases}
G_0 = ~X_0\\
G_1 = ~X_1        &\text{if $p=2$ and $d$ is odd,}\\
G_i = ~X_i^{2p-1} &\text{in any other case.}
\end{cases}$$
Then, $(G_i)_{0 \le i <k}$ defines a primitive tower $(\U_0,\dots,\U_k)$.
\end{theorem}

As before, for $i \ge 1$, let $P_i = X_i^p - X_i - G_{i-1}$ and for $i
\ge 0$, let $K_i$ be the ideal $\langle Q_0,P_1,\dots,P_i\rangle$ in
$\F_p[X_0,\dots,X_i]$.  Then the theorem says that for $i\ge 0$,
$\U_i=\F_p[X_0,\dots,X_i]/K_i$ is a field, and that $x_i=X_i \bmod
K_i$ generates it over $\F_p$.  We prove it as a consequence of a more
general statement.

\begin{lemma}
  Let $\U$ be the finite field with $p^n$ elements, and let $\U'/\U$
  be an extension field with $[\U':\U]=p^i$. Let $\alpha\in\U'$ be
  such that
  \begin{equation}
    \label{eq:ASgen}
    \Tr_{\U'/\U}(\alpha) = \beta \ne 0
    \text{,}
  \end{equation}
  then $\F_p[\beta]\subset\F_p[\alpha]$ and $p^i$ divides
  $\left[\F_p[\alpha]:\F_p[\beta]\right]$.
\end{lemma}
\begin{proof}
  \pdfmctwo{Recalled what Eq. P2 is.}
  Equation~\eqref{eq:ASgen} can be written as $\beta = \sum_j
  \alpha^{p^{jn}}$, thus $\F_p[\beta] \subset \F_p[\alpha]$.  The rest
  of the proof follows by induction on $i$. If $[\U':\U]=1$, then
  $\alpha=\beta$ and there is nothing to prove. If $i\ge1$, let $\U''$
  be the intermediate extension such that $[\U':\U'']=p$ and let
  $\alpha'=\Tr_{\U'/\U''}(\alpha)$, then, by composition of traces
  (Eq.~\ref{eq:trcomp}), $\Tr_{\U''/\U}(\alpha') = \beta$ and by
  induction hypothesis $p^{i-1}$ divides
  $[\F_p[\alpha']:\F_p[\beta]]$.

  Now, suppose that $p$ does not divide
  $[\F_p[\alpha]:\F_p[\alpha']]$.  Since $\F_p[\alpha']\subset\U''$,
  this implies that $p$ does not divide $[\U''[\alpha]:\U'']$; but
  $\alpha\in\U'$ and $[\U':\U'']=p$ by construction, so necessarily
  $[\U''[\alpha]:\U''] = 1$ and $\alpha\in\U''$. This implies
  $\Tr_{\U'/\U''}(\alpha) = p\alpha = 0$ and, by~\ref{eq:trcomp},
  $\beta=0$. Thus, we have a contradiction and $p$ must divide
  $[\F_p[\alpha]:\F_p[\alpha']]$. The claim follows.\end{proof}

\begin{corollary}
  \label{coro:gen}
  With the same notation as above, if $\Tr_{\U'/\U}(\alpha)$ generates
  $\U$ over $\F_p$, then $\F_p[\alpha] = \U'$.
\end{corollary}

Hereafter, recall that we write $\gamma_i=G_i \bmod K_i$. We prove
that the $\gamma_i$'s meet the conditions of the corollary.

\begin{lemma}
  \label{coro:trace}
  If $p\ne2$, for $i \ge 0$, $\U_i$ is a field and, for $i\ge1$,
  \begin{equation}
    \label{eq:79}
    \Tr_{\U_i/\U_{i-1}}(\gamma_i) = -\gamma_{i-1}
    \text{.}    
  \end{equation}
\end{lemma}
\begin{proof} Induction on $i$: for $i=0$, this is true by
  hypothesis. For $i \ge 1$, by induction hypothesis
  $\U_0,\ldots,\U_{i-1}$ are fields; we then set $i'=i-1$ and prove by
  nested induction that $\Tr_{\U_{i'}/\F_p}(\gamma_{i'})\ne 0$ under
  the hypothesis that $\U_0,\ldots,\U_{i'}$ are fields. This,
  by~\ref{th:artin-schreier}, implies that
  $X_i^p-X_i-\gamma_{i-1}$ is irreducible in $\U_{i-1}[X_{i+1}]$ and
  $\U_i$ is a field.

  For $i'=0$, $\Tr_{\U_0/\F_p}(\gamma_0)=\Tr_{\U_0/\F_p}(x_0)$ is
  non-zero and we are done.  For $i' \ge 1$, we know that
  $\gamma_{i'}=x_{i'}^{2p-1}=x_{i'}^px_{i'}^{p-1}$, which rewrites
  \begin{equation}
    (x_{i'}+\gamma_{i'-1})x_{i'}^{p-1} = x_{i'}^p +\gamma_{i'-1} x_{i'}^{p-1}
    = \gamma_{i'-1} + x_{i'} +\gamma_{i'-1} x_{i'}^{p-1}.
  \end{equation}
  By~\ref{eq:pd} we get Eq.~\eqref{eq:79}, and by~\ref{eq:trcomp} we
  deduce the equality
  \begin{equation}
    \label{eq:64}
    \Tr_{\U_{i'}/\F_p}(\gamma_{i'})=-\Tr_{\U_{i'-1}/\F_p}(\gamma_{i'-1})\text{.}    
  \end{equation}
  The induction assumption implies that this is non-zero, and the
  claim follows.
\end{proof}

\begin{lemma}
  If $p=2$, for $i \ge 0$, $\U_i$ is a field. For $i\ge2$,
  \begin{align}
    \Tr_{\U_i/\U_{i-1}}(\gamma_i) &= 1+\gamma_{i-1}
    \text{,}\\
    \Tr_{\U_1/\U_0}(\gamma_1) &= \begin{cases}
      1+\gamma_0 &\text{if $d$ even,}\\
      1          &\text{if $d$ odd.}
    \end{cases}
  \end{align}
\end{lemma}
\begin{proof}
  The proof closely follows the previous one. For $i'=0$,
  $\Tr_{\U_0/\F_p}(\gamma_0)=\Tr_{\U_0/\F_p}(x_0)$ is non-zero.  For
  $i'=1$ and $d$ odd, 
  \begin{equation}
    \label{eq:66}
    \Tr_{\U_1/\U_0}(\gamma_1)=\Tr_{\U_1/\U_0}(x_1)  = 1    
  \end{equation}
  by~\ref{eq:pd}, and
  \begin{equation}
    \label{eq:67}
    \Tr_{\U_0/\F_p}(1) = d\bmod 2\ne0
    \text{.}
  \end{equation}
  For all the other cases 
  \begin{equation}
    \label{eq:68}
    \gamma_{i'}=x_{i'}^2x_{i'}=\gamma_{i'-1} +
    (1+\gamma_{i'-1})x_{i'}\text{,}    
  \end{equation}
  thus
  \begin{equation}
    \Tr_{\U_{i'}/\U_{i'-1}}(\gamma_{i'})=1+\gamma_{i'-1}    
  \end{equation}
  by~\ref{eq:pd} and $\Tr_{\U_{i'-1}/\F_p}(1) = 0$. In any case, using
  the induction hypothesis and~\ref{eq:trcomp}, we deduce
  $\Tr_{\U_{i'}/\F_p}(\gamma_{i'}) = 1$ and this concludes the proof.
\end{proof}


\begin{proof}[Proof of Theorem~\ref{th:cantor}]
  We prove that $\U_i=\F_p[\gamma_i]$, then the theorem follows since
  clearly $\F_p[\gamma_i]\subset\F_p[x_i]$.

  If $p\ne2$, by Lemma~\ref{coro:trace} and~\ref{eq:trcomp},
  \begin{equation}
    \label{eq:69}
    \Tr_{\U_i/\U_0}(\gamma_i) = (-1)^i\gamma_0
    \text{,}    
  \end{equation}
  thus $\U_i=\F_p[\gamma_i]$ by Corollary~\ref{coro:gen} and the fact
  that $\gamma_0 = x_0$ generates $\U_0$ over $\F_p$.

  If $p=2$, we first prove that $\U_1=\F_p[\gamma_1]$.  If $d$ is odd,
  $\gamma_1^p + \gamma_1 = x_0$ implies $\U_0\subset\F_p[\gamma_1]$,
  but $\gamma_1\not\in\U_0$, thus necessarily $\U_1=\F_p[\gamma_1]$.
  If $d$ is even, $\Tr_{\U_1/\U_0}(\gamma_1)=1+\gamma_0$ clearly
  generates $\U_0$ over $\F_p$, thus $\U_1=\F_p[\gamma_1]$ by
  Corollary~\ref{coro:gen}.
  
  Now we proceed like in the $p\ne2$ case by observing that
  $\Tr_{\U_i/\U_1}(\gamma_i)=1+\gamma_1$ generates $\U_1$ over $\F_p$.
\end{proof}

\begin{remark}
  The choice of the tower of Theorem~\ref{th:cantor} is in some sense
  \emph{optimal} between the choices given by
  Corollary~\ref{coro:gen}. In fact, each of the $G_i$'s is the
  ``simplest'' polynomial in $\F_p[X_i]$ such that
  $\Tr_{\U_i/\F_p}(\gamma_i)\ne0$, in terms of lowest degree and least
  number of monomials, as shown by Proposition~\ref{th:p3} and
  Eq.~\eqref{eq:62}.
\end{remark}

We also remark that the construction we made in this section gives us
a family of normal elements for free. In fact, recall the following
proposition from~\cite[Section~5]{hachenberger}.
\begin{proposition}
  Let $\U'/\U$ be an extension of finite fields with $[\U':\U]=kp^i$
  where $k$ is prime to $p$ and let $\U''$ be the intermediate field
  of degree $k$ over $\U$. Then $x\in\U'$ is normal over $\U$ if and
  only if $\Tr_{\U'/\U''}(x)$ is normal over $\U$. In particular, if
  $[\U':\U]=p^i$, then $x\in\U'$ is normal over $\U$ if and only if
  $\Tr_{\U'/\U}(x)\ne0$.
\end{proposition}
Then we easily deduce the following corollary.
\begin{corollary}
  Let $(\U_0,\ldots,\U_k)$ be an Artin-Schreier tower defined by some
  $(G_i)_{0\le i<k}$. Then, every $\gamma_i$ is normal over $\U_0$;
  furthermore $\gamma_i$ is normal over $\F_p$ if and only if
  $\Tr_{\U_i/\U_0}(\gamma_i)$ is normal over $\F_p$.
\end{corollary}

In the construction of Theorem~\ref{th:cantor}, if we furthermore
suppose that $\gamma_0$ is normal over $\F_p$, using
Lemma~\ref{coro:trace} we easily see that the conditions of the
corollary are met for $p\ne2$.  For $p=2$, this is the case only if
$[\U_0:\F_p]$ is even (we omit the proofs that if $\gamma_0$ is normal
then so are $-\gamma_0$ and $1+\gamma_0$).

\begin{remark}
  Observe however that this does not imply the normality of the
  $x_i$'s. In fact, they can {\em never} be normal because
  $\Tr_{\U_i/\U_{i-1}}(x_i) = 0$ by~\ref{eq:pd}.  Granted that
  $\gamma_0$ is normal over $\F_p$, it would be interesting to have an
  efficient algorithm to switch representations from the univariate
  $\F_p$-basis in $x_i$ to the $\F_p$-normal basis generated by
  $\gamma_i$. 

  \pdfmctwo{The corollary and the remark answer a question Kaltofen
    asked me at ISSAC 2009. I just added this paragraph, so to cool
    down the excitement of the reader about the interest of using such
    normal basis.} In particular, having such a change of
  representations would allow efficient computations of Frobenius
  automorphisms.  However, in Section~\ref{sec:pseudotrace-frobenius},
  we give a quasi-optimal algorithm to compute Frobenius
  automorphisms, making no use of this remark.
\end{remark}


% Local Variables:
% mode:flyspell
% ispell-local-dictionary:"american"
% mode:TeX-PDF
% mode: reftex
% TeX-master: "../these"
% End:
%
% LocalWords:  Schreier Artin pseudotrace frobenius bivariate memoization
