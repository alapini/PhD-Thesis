\appendix

\section{Strengthenings of the main theorem}

We discuss here some improvements to theorem \ref{th:cantor}. We first
investigate more deeply the existence of primitive towers.

\begin{definition}
  Let $\F_q$ be a finite field of characteristic $p$, we note by
  $\AST^i(\F_q)$ the unique (up to isomorphism) extension field such
  that $[\AST^i(\F_q):\F_q]=p^i$.
\end{definition}

\begin{proposition}
  Let $\U_i = \AST^i(\F_q)$ be a field of characteristic $p$ and let
  $\alpha\in\U_i$ be such that $\Tr_{\U_i/\F_q}(\alpha) = \beta \ne 0$. Then
  \[\F_p[\alpha] \supset \AST^i(\F_p[\beta]) \text{.}\]
\end{proposition}
\begin{proof}
  First observe that trivially $\F_p[\beta]\subset\F_p[\alpha]$ since
  $\beta$ can be written as a sum of powers of $\alpha$.
  
  If $i=0$ there is nothing left to prove. Suppose then that $i\ge1$
  and let $\U_{i-1}=\AST^{i-1}(\F_q)$, we start by proving that
  $\alpha\notin\U_{i-1}$. In fact, suppose this is not the case, then
  \begin{equation*}
    \beta=\Tr_{\U_i/\F_q}(\alpha) =
    \Tr_{\U_{i-1}/\F_q}\circ\Tr_{\U_i/\U_{i-1}}(\alpha) =
    \Tr_{\U_{i-1}/\F_q}(p\alpha) = 0 \text{,}
  \end{equation*}
  leading to a contradiction.

  Observe now that $\F_p[\beta]\subset\F_q$ and let
  $[\F_p[\alpha]:\F_p[\beta]]=d$, then $p^i|d$. In fact if this were
  not the case, $\alpha\in\F_p[\alpha]\subset\U_{i-1}$, leading to a
  contradiction. The claim follows.
\end{proof}

\begin{corollary}
  Let $\U_i=\AST^i(\F_q)$ and $\alpha\in\U_i$. If
  $\Tr_{\U_i/\F_q}(\alpha)$ generates $\F_q$ over $\F_p$, then
  $\alpha$ generates $\U_i$ over $\F_p$.
\end{corollary}

Then, from corollary \ref{coro:trace} it is easy to see that the
$\gamma_i$'s of Theorem~\ref{th:cantor} meet the conditions of this
corollary. To prove the primitivity of the tower, it then suffices to
notice that $\U_i=\F_p[\gamma_i] \subset \F_p[x_i]$ because $\gamma_i$
is a power of $x_i$. By generalizing this argument we have the
following

\begin{corollary}
  Let the tower $(\U_0,\ldots,\U_k)$ be generated by
  $(\alpha_0,\ldots,\alpha_{k-1})$. If for every $i$,
  $\Tr_{\U_i/\U_0}(\alpha_i)$ generates $\U_0$, then the tower is
  primitive.
\end{corollary}

We now discuss the issue of finding normal elements in Artin-Schreier
towers. First notice that because of Lemma~\ref{Lemma:trace-AS}, a
root of an Artin-Schreier polynomial cannot be a normal element, thus
there is no hope of having such a thing as a \emph{normal} tower.

Finding normal elements in generic extensions has been studied by many
authors. We address to~\cite{Hach} for a complete treatment. The case
of Artin-Schreier towers is much simpler and resumes to the following
to statements.

\begin{proposition}
  Let $\LK/\K$ be an extension of finite fields with $[\LK:\K]=kp^i$
  where $k$ is prime to $p$ and let $\K'$ be the intermediate field of
  degree $k$ over $\K$. Then $x\in\LK$ is normal over $\K$ if and only
  if $\Tr_{\LK/\K'}(x)$ is normal over $\K$.
\end{proposition}

\begin{corollary}
  Let $\LK/\K$ be an extension of finite fields with $[\LK:\K]=p^i$,
  then $x\in\LK$ is normal over $\K$ if and only if
  $\Tr_{\LK/\K}(x)\ne0$.
\end{corollary}

For a proof, see~\cite[Section 5]{Hach}. Then we easily deduce the
following corollary.

\begin{corollary}
  Let the tower $(\U_0,\ldots,\U_k)$ be generated by
  $(\alpha_0,\ldots,\alpha_{k-1})$. Every $\alpha_i$ is normal over
  $\U_0$. Furthermore $\alpha_i$ is normal over $\F_p$ if and only if
  $\Tr_{\U_i/\U_0}(\alpha_i)$ is normal over $\F_p$.
\end{corollary}

Then, by observing that any scalar multiple of a normal element is
normal again, one sees that, if $p=3$, the $\gamma_i$'s of
Theorem~\ref{th:cantor} are normal over $\F_p$ if and only if
$\gamma_0$ is normal over $\F_p$. For $p=2$ this is the case only if
$[\U_0:\F_p]$ is even.

We finally are interested in the multiplicative order of the
$x_i$'s. Conjecture : $\langle x_i\rangle = \langle
x_{i+1}\rangle\cap\U_i^\ast$.


% Local Variables:
% mode:flyspell
% ispell-local-dictionary:"british"
% End:
%
% LocalWords:  Schreier Artin
