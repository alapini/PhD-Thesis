%% these.tex
%% Copyright 2010 Luca De Feo
%% All rights reserved


\section{\tALpy{}}
\label{sec:texttttransalpyne}

\lstset{language=python}

% Overfull in a4ps
We present here a ``Python implementation of a transposable Algebraic
Language'', in short \index{transalpyne@\tALpy{}}\tALpy{}.

\tALpy{} is a limited functional language incorporating all the
features that we have discussed until now, in particular:
\begin{itemize}
\item it supports algebraic code (i.e., not restricted to native types
  such as integers or floats);
\item it has a static type checker that only gives types to algebraic
  variables and performs linearity inference (see
  Section~\ref{sec:inference});
\item it is able to transpose any linearization of an algebraic
  program (see Section~\ref{sec:r-algebr-algor});
\item its code can be compiled to Python code, or interpreted inside
  the Python interpreter.
\end{itemize}

At the moment we write, the first stable release of \tALpy{} is not
ready yet. We plan to start distribute it by the beginning of 2011. It
will be available from \url{http://transalpyne.gforge.inria.fr/}.


\subsection{Concepts}
\tALpy{} has been conceived as a scripting language to be used on top
of computer algebra systems. We made an effort to give syntax and
semantics as close as possible to the Python programming language.

In \tALpy{} there is no such concept as an executable program: only
functions can be defined in \tALpy{}.  \tALpy{} programs can be
compiled to Python code, so that their functions can be called by
Python programs; we plan to support compilation to other languages in
the future. \tALpy{} can also be interpreted via the Python
interpreter. A \tALpy{} library contained in a file
\texttt{my-library.yp} can be imported in a Python program via the
statement
\begin{lstlisting}
  import my-library
\end{lstlisting}
The Python interpreter recognizes the \texttt{.yp} extension and
launches the \tALpy{} interpreter on the file; the functions of the
library are interpreted and transposed by the \tALpy{} interpreter and
their names are exported to the Python namespace.

\tALpy{} is mostly dynamically typed, with the only exception of
\emph{algebraic types}. In order to transpose a function, \tALpy{}
must know at \emph{transpose time} which variables contain algebraic
elements and which variables contain other data (such as booleans,
strings, ints, etc.); this can be done by explicitly specifying the
type of the input and output parameters of a function, while all the
other variables can be left untyped. \tALpy{} supports two sorts of
algebraic types: ring elements and module elements; we plan to support
more complex algebraic types, such as algebras, in the
future. \tALpy{} relies on Python operator overloading to represent
ring operations.

\tALpy{}'s type checker also performs a linearity inference as
described in Section~\ref{sec:inference}. The base ring is supposed to
be commutative, we plan to add support for non commutative rings in
the future. When a function admits more than one linearization,
\tALpy{} computes and transposes all of them; if this leads to an
ambiguity in a function call, it raises a compile time error.


\subsection{Syntax}
\label{sec:syntax}
We only describe \tALpy{} syntax informally. Indentation has a
syntactic value (it delimits blocks) and keywords are pretty much the
same as in Python. A \tALpy{} file contains a \emph{type declaration}
section followed by a \emph{name definition} section.

\subsubsection{Type declarations}
\index{algebraic~type} Type declarations let the user declare which
are the algebraic types.  \tALpy{} supports two type constructors: a
ring constructor and a free module constructor.

\begin{lstlisting}
  type Ring R
  type Module(R) M
\end{lstlisting}

This example declares \lstinline+R+ as a ring type and \lstinline+M+
as a free module type over the ring \lstinline+R+. The typechecker
ensures that modules are consistently declared.


\subsubsection{Name declarations}
Name declarations take three forms: \emph{imports}, \emph{function
  definitions} and \emph{aliases}. Imports are declared as in Python
and have the same semantics. Note however that the type checker does
not enter imported modules to infer linearity: any imported function
is considered as a scalar function.

There is no \lstinline+return+ statement in \tALpy{}, function
definitions are declared as follows
\begin{lstlisting}
  def (a, b)my-function(c, d):
\end{lstlisting}
where input arguments are given on the right and output arguments on
the left.

Inside function definitions, there are four types of statements:
\lstinline+pass+ statements (the statement that does nothing),
assignments (including augmented assignments), \lstinline+for+ loops
and \lstinline+if+s. The syntax is identical to Python.

On the left hand side of assignments, may only appear variable names
and subscripts. On the right hand side of assignments, the following
types of expressions may appear:
\begin{itemize}
\item String, numeric and boolean constants;
\item Binary and unary operators \lstinline-+-, \lstinline+-+,
  \lstinline+*+, \lstinline+/+,
  \lstinline+%+, \lstinline+div+, \lstinline+mod+, \lstinline+<+,
  \lstinline+>+, \lstinline+<=+, \lstinline+<=+, \lstinline+==+,
  \lstinline+!=+, \lstinline+and+, \lstinline+or+, \lstinline+not+,
  \lstinline+in+;
\item Parenthesized expressions;
\item Subscripts and slices;
\item List constructors, including comprehensions;
\item Variable evaluations;
\item Function calls.
\end{itemize}
The syntax for all of these is identical to Python. The only notable
exception are function calls where a keyword \lstinline+trans+ is
added to let the user call a transposition of a function. In case a
function has more than one linearization (and thus more than one
transposition), \emph{signature specifiers} enclosed in braces
\lstinline+{+, \lstinline+}+ let the user specify which
linearization/transposition is wanted.

Finally, aliases let the user export specific
linearization/transpositions of functions with names that can be used
inside a Python program.

Figure \ref{fig:prog} gives a complete \tALpy{} example. It defines a
\lstinline+product+ function and two aliases (with transposition and
signature specifiers).

\begin{figure}[!ht]
  \centering
\begin{lstlisting}
type Ring R

def (R c)product(R a, R b):
    c = a * b
  
l_product = trans {linear R}product{linear R, const R}
r_product = trans {linear R}product{const R, linear R}
\end{lstlisting}
  
  \caption{A \tALpy{} program}
  \label{fig:prog}
\end{figure}

\ifbfive\pagebreak\fi
\subsection{Semantics}
We only give here the points where \tALpy{} semantics differ from
Python.

\subsubsection{Types}
\tALpy{} is statically typed for algebraic types. The type of each
input and output parameter of a function must be specified in the
definition as in figure \ref{fig:prog}. When the type of an argument
is omitted, it is assumed to have non-algebraic type. Variables inside
the body of a function cannot be explicitly typed, the type checker
deduces their types from the types of the input parameters.

\subsubsection{Side effects}
There is no side effect in \tALpy{}. In particular, there is no global
variable and assignment itself is a let-binding. After having
transposed the functions, the \tALpy{} compiler/interpreter leaves to
the target language the task of executing them, thus it cannot enforce
the no-side-effect policy at runtime. It is the responsibility of the
user to insure that no side effect happens inside a \tALpy{} function.

\subsubsection{Conditionals, loops}
\label{sec:conditionals-loops}
In order for the type inference to work, we must work around a feature
of Python. The following is correct Python code, even if a is not
defined before the if statement:
\begin{lstlisting}
  if x:
      a = 0
  b = a
\end{lstlisting}
In case the if-block is not executed, Python simply issues a runtime
error. If we want to do a type inference, we must enforce a stricter
policy.

In \tALpy{} one cannot use outside of an if-block a variable that has
been first assigned in one of the branches but not in all of them.
If-blocks have a return type: it is the product of the types of all
the variables modified inside one of its branches, and that appear in
each branch or were defined before the if; this type must be the same
at the issue of any branch. For example, in the following code
\begin{lstlisting}
  a = R.zero()
  if True:
      a = R.one()
  else:
      b = R.zero()
\end{lstlisting}
the return type of the if statement is the type of \lstinline{a},
because it has been modified in the if branch, and was defined before
the if, and is the same in both branches (namely,
\lstinline{R}). Using the value of \lstinline{b} after the block
generates a compile time error.

For loops also have a return type. To this extent, they are treated as
if they were if-blocks with an empty else-block.



\subsubsection{Algebraic variables}
\label{sec:algebraicvar}
\index{algebraic~type} Type declarations merely say that some
variables belong to a type, but do not specify any particular
implementation of the type. The implementation of rings and modules is
left to the user and must be given in an external module written in
Python (or in whatever the target language is). The user is only
required to implement them as objects and to expose a few methods.

Ring objects must:
\begin{itemize}
\item Overload \lstinline-+- and \lstinline+*+ with the obvious semantic;
\item Implement a method \lstinline-zero- that returns the zero of the ring;
\item Optionally, implement a method \lstinline-one- that returns the
  one of the ring;
\item Optionally, implement a method \lstinline-Z- that takes an
  integer $n$ and returns the element $n\cdot 1$ of the ring;
\item Optionally, implement methods \lstinline+div+ and
  \lstinline+mod+ that perform Euclidean division with remainder;
\item Optionally, overload \lstinline+/+, thus making the ring into a
  field.
\end{itemize}

Module objects must:
\begin{itemize}
\item Overload \lstinline-+- and \lstinline+*+ with the obvious semantic;
\item Implement a method \lstinline-zero- that returns the zero of the
  module;
\item Overload the subscript operator \lstinline+[]+ so that it
  implements some arbitrary projections on the underlying ring. Most
  often, a module will be implemented as an array of ring objects and
  \lstinline+[i]+ will just be projection onto the \lstinline+i+-th
  coordinate.
\item Overload the assignment-to-subscript operator in the obvious
  way.
\end{itemize}

Algebraic output parameters of a function are implicitly initialized
to zero via their \lstinline+zero+ method. This insures that
non-assigned algebraic output parameters are always linear in the
inputs of the function.

Algebraic elements cannot be combined through the use of lists: lists
of algebraic objects are non-algebraic objects and extraction from a
list always yields a non-algebraic object.


\subsubsection{Function calls}
\tALpy{} does not have tuples; the return type of a function with many
output parameters is not a tuple, as a consequence its return value
cannot be assigned to a variable: it must be assigned to as many
variables as there are output parameters.  Another consequence of this
is that functions with many outputs cannot be used inside expressions:
their outputs can only be assigned to variables.

Function names not declared in the library are simply regarded as
external functions. They are assumed to have one return parameter,
thus a multi-assignment will return an error. External functions have
no algebraic input or output parameters. This is useful to call
built-in Python functions from inside a \tALpy{} program (one common
example is the function \lstinline+range+, needed to iterate in for
loops).

\subsubsection{Recursion and Higher order}
\tALpy{} allows recursion and even calling its own transpose. It does
not allow to pass functions as arguments to functions, although the
transposition algorithm internally uses this technique to transpose
for loops. A higher order transposable language is theoretically
possible and we consider adding this feature to \tALpy{} in the
future.

\subsection{Linearization}
\label{sec:linearization}

% Overfull in b5
After a first type checking to determine which variables are
algebraic, \tALpy{}'s compiler/interpreter runs the linearization
inference algorithm of Section~\ref{sec:inference}.

Since a function may have more than one linearization, \tALpy{} allows
the user to annotate the types of the algebraic arguments so that they
can be constrained to be linears or scalars (non-algebraic arguments
are by default scalars). The two keywords for this are
\lstinline{linear} and \lstinline{const}, the following code shows an
example of use:
\begin{lstlisting}
  def (linear R c)product(linear R a, const R b):
      c = a * b
\end{lstlisting}

The user is free to leave some arguments unspecified. For example, the
previous code could have been written
\begin{lstlisting}
  def (R c)product(linear R a, R b):
      c = a * b
\end{lstlisting}
The linearity inference looks for all the linearizations compatible
with the specified modifiers, thus in this case both codes yield the
same linearization. If more than a linearization is acceptable,
\tALpy{} computes all of them.

\ifafourps\enlargethispage{\baselineskip}\fi
We call \index{signature}\emph{signature} a list of linear/const
modifiers inferred for the arguments of a function. Signature
specifiers (see Section~\ref{sec:syntax}) can be used to distinguish
between different linearizations when calling the transposed
function. They are written as
\begin{lstlisting}
  {const R}product{linear R, const R}
\end{lstlisting}

Thus, in the example we gave in figure \ref{fig:prog},
\lstinline+l_product+ is an alias for the transposed left-linear
product, while \lstinline+r_product+ is an alias for the transposed
right-linear one. Aliases are extremely useful since they are the only
way to export the transposed functions to the namespace of the target
language.




\subsection{Partial evaluation}
\label{sec:partial-evaluation-1}
After the type checking and linearity inference phase, any discovered
linearization of any function is partially evaluated and transposed as
in Section~\ref{sec:r-algebr-algor}.

\begin{figure}[b]
\begin{lstlisting}
def (R a, R b)f(R c, R d):
  if d > R.zero():
    x, y = f(c, d - R.one())
    a, b = x * y, y + R.one()
  else:
    a, b = c, d
\end{lstlisting}
  \caption{\label{fig:bad}A \tALpy{} program that does nothing
    interesting, but is very hard to transpose (with respect to the
    signature \lstinline+\{linear R, const R\}f\{linear R, const
    R\}+).}
\end{figure}


The first step is to translate any for loop into a tail-recursive
function, as this simplifies greatly the partial evaluation. The
partial evaluation is then done in two steps: first we evaluate all
the statements depending exclusively from the scalars, then we strip
those statements off the partially evaluated program. Let us explain
this through an example. Consider the program in Figure~\ref{fig:bad},
we want to transpose it with respect to the signature $\ell\times
s\ra\ell\times s$.

First we generate the program that evaluates at \lstinline+d+ the
statements that only depend on \lstinline+d+, we call it
\lstinline+fS+.
\begin{lstlisting}
def (R b)fS(R d):
  if d > R.zero():
    y = fS(d - R.one())
    b = y + R.one()
  else:
    b = d
\end{lstlisting}

Now we generate the partial evaluation of \lstinline+f+ at
\lstinline+d+ by stripping off all the values that solely depend on
\lstinline+d+.
\begin{lstlisting}
def (R a)f(R c):
  if d > R.zero():
    x = f(c, d - R.one())
    a = x * y
  else:
    a = c
\end{lstlisting}
Notice that this program needs the values of \lstinline+d+ and
\lstinline+y+ that are computed in \lstinline+fS+. Depending on
whether the code is compiled or interpreted, we use a different
strategy.

In case we generate code, we simply concatenate the bodies of
\lstinline+fS+ and \lstinline+f+, eventually performing
$\alpha$-conversion to avoid name clashes (notice that
$\alpha$-conversion is possible because we have eliminated for
loops). Thus, the refactored code for \lstinline+f+ would look like in
Figure~\ref{fig:bad-ref}.

\begin{figure}[t]
  \centering
\begin{lstlisting}
def (R a, R b)f(R c, R d):
  # Scalar part
  if d > R.zero():
    y = fS(d - R.one())
    b = y + R.one()
  else:
    b = c
  # Linear part
  if d > R.zero():
    x, _ = f(c, d - R.one())
    a = x * y
  else:
    a = c
\end{lstlisting}
  \caption{\label{fig:bad-ref}Refactoring of the code of
    Figure~\ref{fig:bad}, separating the scalar from the linear part.}
\end{figure}


Notice, however, that this is inefficient because it generates two
recursive calls: one in the scalar part and one in the linear part. A
better solution would be to evaluate \lstinline+fS(d)+ only once and
save all its stack, so that all the scalar values needed in the
partial evaluation can be retrieved from it. When we interpret code we
choose an intermediate solution: we save the return value of any call
to \lstinline+fS+ in a memoization table, then retrieve the values
from the table when they are needed in the linear part. This is a sort
of lazy evaluation of functions in the scalar part.

\begin{remark}
  Regardless of whether the code is interpreted or compiled, scalar
  parts of real world algorithms tend to be very short and simple. In
  particular they seldom contain a recursive call having both scalar
  and linear return values as the one in Figure~\ref{fig:bad}. Thus we
  can reasonably assume that the generated code is as efficient as the
  original one.
\end{remark}

\ifbfive\enlargethispage{\baselineskip}\fi
\subsection{Transposition}
\label{sec:transposition}
Finally, the linear part of each linearized function is transposed as
in Section~\ref{sec:stra-line-progr}. In doing this we read the code
from bottom to top and transpose each instruction, i.e.\ we swap input
and output algebraic arguments of any function, and substitute each
function call with a call to the transposed function. The example of
Figure~\ref{fig:bad} is transposed in Figure~\ref{fig:mechant}. We
borrow the names \index{forward~sweep}\emph{forward sweep} and
\index{reverse~sweep}\emph{reverse sweep} from the theory of automatic
differentiation~\cite{griewank2008evaluating}, where a similar
technique is applied in the reverse mode.

\begin{figure}[t]
  \centering
\begin{lstlisting}
def (R c, R b)fT(R a, R d):
  # Forward sweep
  if (d > R.zero()):
    y = fS(d - R.one())
    b = y + R.one()
  else:
    b = d
  # Reverse sweep
  if (d > R.zero()):
    x = a * y
    c, _ = trans f(x, d - R.one())
  else:
    c = a
\end{lstlisting}
  \caption{Transposition of Figure~\ref{fig:bad}.}
  \label{fig:mechant}
\end{figure}




%%% Local Variables: 
%%% mode:flyspell
%%% ispell-local-dictionary:"american"
%%% mode: TeX-PDF
%%% mode: reftex
%%% TeX-master: "../these"
%%% End: 
