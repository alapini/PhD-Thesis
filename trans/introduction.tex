The complexity of algebraic algorithms is often more easily described
in a non-Turing model where one assumes that any algebraic operation
can be done in a unit of time and any other operation is
free. \emph{Algebraic complexity} studies precisely the computational
models that behave this way.

\pdfmargincomment{I do not agree with Éric: if we consider the BSS
model as being part of algebraic complexity, then the Turing model is
its binary counterpart.}  For algorithms over finite rings, the
algebraic complexity gives a precise estimate for the complexity in
the Turing or RAM model (also called \emph{binary models}). For other
rings, the algebraic estimate may be way off target, but it can
nevertheless give useful information.

In this chapter we study models that allow one to study the algebraic
complexity of linear operators. We first present the \emph{arithmetic
  circuit}, then the \emph{straight line program}. Because of their
algebraic structure, these models support some algebraic
manipulations.  Our principal interest will be the \emph{transposition
  theorem}, stating that it is possible to apply classical duality (in
the sense of Section~\ref{sec:linear-algebra:duality}) to programs,
while preserving some complexity invariants. The interest for the
transposition theorem comes from the applications we have seen in
Section~\ref{sec:transp-algor} and other more advanced applications
that we will see in the next chapters.

Finally, in Section~\ref{sec:autom-diff}, we study the
relationship between the transposition theorem and the classical
theory of \emph{automatic differentiation}.



% Local Variables:
% mode:flyspell
% ispell-local-dictionary:"american"
% mode:reftex
% mode:TeX-PDF
% TeX-master: "../these"
% End:
%
