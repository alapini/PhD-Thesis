\section{\tAL}
\label{sec:transAL}

In the previous Section we saw how the scalar and linear parts of a
computation can be automatically inferred through type checking, this
is however not enough to generate transposed code for a generic
program. In this section we define \tAL{}, a very narrow sublanguage
of purely functional languages, and give an algorithm to transpose
programs written in it. Our approach is similar to \cite{BoLeSc03},
but \tAL{} is more powerful than the language defined there.

The main obstacle to transposition of $\lambda$-terms is the
possibility to form terms whose types contain an arbitrary number of
arrows; transposition, on the other hand, only deals with types
containing one unique arrow. To avoid this problem \tAL{} only permits
to declare function types with one arrow, it is defined by the
following EBNF grammar
\begin{align}
  S &::= F^\ast\\
  \tau &::= \ell \;|\; s \;|\; \ldots\\
  A &::= x::\tau\\
  \label{eq:tAL-exp}
  E &::= A^\ast\\
  \label{eq:tAL-if}
  E &::= A_b \;?\; E_i : E_e\\
  \label{eq:tAL-apply}
  E &::= f \al A^\ast\\
  \label{eq:tAL-let}
  E &::= A^\ast\la E_1.E_2\\
  \label{eq:tAL-fun}
  F &::= f = \proc A_i^\ast \ra E
\end{align}
where $x$ are identifiers and $f$ function names (the two namespaces
are separate). Its semantics are given by the following translation to
a functional language (using Haskell syntax)
\begin{align}
  [x_1::\tau_1\cdots x_n::\tau_n]_\delta &\quad=_\delta\quad (x_1::\tau_1,\ldots,x_n::\tau_n)
  \quad\text{,}\\
  [A_b \;?\; E_i : E_e]_\delta &\quad=_\delta\quad
  \text{if }A_b\text{ then }[E_i]_\delta\text{ else }[E_e]_\delta
  \quad\text{,}\\
  [f \al A_i^\ast]_\delta &\quad=_\delta\quad
  f [A_i^\ast]_\delta \quad\text{,}\\
  [A^\ast\la E_1.E_2]_\delta &\quad=_\delta\quad
  \text{let } [A^\ast]_\delta = [E_1]_\delta \text{ in } [E_2]_\delta
  \quad\text{,}\\
  [f = \proc A_i^\ast \ra E]_\delta &\quad=_\delta\quad
  f = \lambda [A_i^\ast]_\delta.[E]_\delta\quad\text{.}
\end{align}
For simplicity, we assume a strict evaluation strategy.

Some functions and identifiers are supposed predefined in \tAL{}:
besides the usual arithmetic and logic constants, there are the
constants \lstinline{zero}, \lstinline{(<+>)}, etc., defined in the
previous Section. Constants of function type are exported to \tAL{} as
predefined functions, constants of non-function type as identifiers,
with the only exception of \lstinline{zero} that is exported as a
function of type $()\ra\tau$.

\tAL{} programs must be valid programs after translation to the
functional language and they must type correctly according to the
rules given in the previous Section; in practice the types of
identifiers could be omitted since they can automatically be computed
by the generalized Hindley-Milner inference we sketched there, but we
explicitly list them in \tAL{} for simplicity. In our grammar we only
listed the types $\ell$ and $s$, since they are the only significant
ones for transposition; in our examples we will note $\ast$ for any
type different from $\ell$ and $s$, or simply omit it.

A \tAL{} program is then just a collection of function definitions,
here's an example that multiplies an element $x$ by a scalar $n\in\Z$:
\begin{equation}
  \label{eq:tALprog}
  \begin{aligned}
    &f = \proc x::\ell \;\; n::\ast \ra\\[-1ex]
    &\qquad b::\ast \la (==) \al n::\ast \;\; 0::\ast\\[-1ex]
    &\qquad b::\ast \;?\\[-1ex]
    &\qquad\qquad \text{zero} \al\\[-1ex]
    &\qquad :\\[-1ex]
    &\qquad\qquad n::\ast \la (-) \al n::\ast \;\; 1::\ast.\\[-1ex]
    &\qquad\qquad y::\ell \la f \al x::\ell \;\; n::\ast.\\[-1ex]
    &\qquad\qquad (\langle+\rangle) \al x::\ell \;\; y::\ell\\[-1ex]
  \end{aligned}
\end{equation}

\tAL{} programs are very similar to straight line programs, in fact,
avoiding the production \eqref{eq:tAL-if} and recursive calls, the two
are strictly equivalent. Just like straight line programs are
isomorphic to circuits \cite[Lemma 13.17]{BuClSh}, terminating \tAL{}
programs are isomorphic to circuit families. To prove this, we
\emph{fix} the values of the variables appearing in the tests of
conditional statements, then a \tAL{} function is just a straight line
program. Furthermore, if we also \emph{fix} all the variables of type
$*$, then we obtain circuit families over $(R,\Sbasis)$; if we
\emph{fix} variables of type $s$ too, we obtain families over
$(R,\Tbasis)$.

We only detail the last isomorphism. First we need to distinguish the
scalar and the linear part of the computation.

\begin{definition}[Scalar restriction]
  The \emph{scalar restriction} of a predefined function $f$, noted
  $\s{f}$, is $f$ itself if its type does not involve any $\ell$, the
  zero function $()\mapsto()$ otherwise. 

  The \emph{scalar restriction} of a \tAL{} function $f$ is defined
  together with the \emph{scalar restriction} $\s{t}$ of a \tAL{} term
  $t$ as follows:

  \begin{align}
    \s{x::\ell\;A^\ast} &\quad=\quad \s{A^\ast} \text{ ,}\\
    \s{x::\tau\;A^\ast} &\quad=\quad x::\tau\;\s{A^\ast}
    \qquad\text{if $\tau\ne\ell$,}\\
    \s{A_b\,?\,E_i:E_e} &\quad=\quad A_b\,?\,\s{E_i}:\s{E_e} \text{ ,}\\
    \notag
    \label{eq:scalar-let}
    \s{f \al A^\ast} &\quad=\quad \s{f} \al \s{A^\ast}
    \text{ ,}\\
    \s{A^\ast\la E_1.E_2} &\quad=\quad \s{A_o^\ast}\la\s{E_1}.\s{E_2}\text{ ,}\\
    \label{eq:scalar-fun}
    \s{f = \proc A_i^\ast\ra E} &\quad=\quad \s{f} = \proc \s{A_i^\ast}\ra\s{E}
    \text{ .}
  \end{align}
\end{definition}

In practice, scalar restriction removes all the names having type
$\ell$ from a function definition. Notice that if $t$ is a valid term,
then $\s{t}$ is valid too: it can be easily shown, in fact, that no
identifier with type $\tau\ne\ell$ depends on any identifier with type
$\ell$.

In the following definition we make use of an environment $\Gamma$,
that is a partial function from names to values.  Let $\Gamma$ be an
environment such that it is defined on all the free variables of a
\tAL{} expression $E$, then the evaluation of $E$ in $\Gamma$, noted
$\Gamma(E)$, is obtained by assigning the values in $\Gamma$ to its
free variables and then evaluating as defined by the semantics.  The
environment $\Gamma:A^\ast\la E$ is the environment $\Gamma$ augmented
by matching the names in $A^\ast$ to the value of $\Gamma(E)$.

\begin{definition}[Linearization of predefined functions]
  Let $v$ be a value of type $s$, the \emph{linearization} by $s$ of
  \lstinline{<*>}$::\ell\ra s\ra\ell$, noted \lstinline{<*>}$_v$, is
  the section $($\lstinline{<*>}$v)$; the linearization of
  \lstinline{<*>}$::s\ra\ell\ra\ell$ is also noted\lstinline{<*>}$_v$
  and is defined analogously. Since the type is always specified by
  the context, we never need to precise which of the two
  linearizations we are talking about.
  
  The functions \lstinline{zero}, \lstinline{<+>} and \lstinline{neg}
  are their own linearization if and only if their type contains
  $\ell$'s. The linearization of any other predefined function is the
  zero function $()\mapsto()$.
\end{definition}

\begin{definition}[Linearization]
  Let $f=\proc x_1::\tau_1\cdots x_n::\tau_n\ra E$ be the definition
  of a \tAL{} function. Up to permutation of the indices, we can
  suppose that $\tau_i=\ell$ if and only if $i\le m$. Let $v_i$ be a
  value of type $\tau_i$ for any $m<i\le n$, the \emph{linearization}
  by $(v_i)_i$ of $f$ is the function defined by
  \begin{equation}
    f_{(v_i)_i} = \proc x_1::\tau_1\cdots x_m::\tau_m\ra [E]_\Gamma
    \text{ ,}
  \end{equation}
  where $\Gamma$ is the environment defined by
  \begin{equation}
    \Gamma(x) =
    \begin{cases}
      v_i \quad\text{if $x=x_i$ for some $i\in[m+1,\ldots,n]$,}\\
      \text{undefined otherwise,}
    \end{cases}
  \end{equation}
  and $[E]_\Gamma$ is defined by the following rules:
  \begin{align}
    \l{x::\ell\;A^\ast} &\quad=\quad x::\ell\;\l{A^\ast} \text{ ,}\\
    %%
    \l{x::\tau\;A^\ast} &\quad=\quad \l{A^\ast}
    \qquad\text{if $\tau\ne\ell$,}\\
    %%
    \label{eq:linear-if}
    [x::*\,?\,E_i:E_e]_\Gamma &\quad=\quad 
    [E_i]_\Gamma \qquad\text{if $\Gamma(x)=$ true,}\\
    %%
    \label{eq:linear-else}
    [x::*\,?\,E_i:E_e]_\Gamma &\quad=\quad 
    [E_e]_\Gamma \qquad\text{if $\Gamma(x)=$ false,}\\
    %%
    \label{eq:linear-apply}
    [f \al A^\ast]_\Gamma &\quad=\quad f_{\Gamma(\s{A^\ast})} \al  \l{A^\ast} \text{ ,}\\
    \label{eq:linear-let}
    [A^\ast\la E_1.E_2]_\Gamma &\quad=\quad
    \l{A^\ast}\la[E_1]_\Gamma.[E_2]_{\Gamma:\s{A^\ast}\la\s{E_1}}
    \text{ .}
  \end{align}
\end{definition}

Using the same notation as in the definition, set
$\pspace_f=\prod_{m<i\le n}\tau_i$. Then for any $v\in\pspace_f$,
$f_v$ has no branchings thanks to rules \eqref{eq:linear-if} and
\eqref{eq:linear-else}. If $f$ terminates on some input $u$, then it
terminates for any other input such that its projection on $\pspace_f$
is equal to the projection of $u$, because linear inputs cannot
influence boolean values, thus they cannot influence branchings. This
implies that either $f_v$ terminates on any input or it always
loops.

Suppose that $f_v$ terminates, if we recursively substitute each
expression $g_{v'}\al A^\ast$ with the definition of $g$, then it is
easily seen by induction that this process eventually terminates and
one ends up with a straight-line program over the basis
\lstinline{zero}, \lstinline{<+>}, \lstinline{neg},
\lstinline{<*>}$_a$. Hence $f_v$ is equivalent to an arithmetic
circuit. To summarize, the mapping $v\mapsto f_v$ from $\pspace_f$ to
\tAL{} programs defines an uniform circuit family if and only if $f$
terminates on any input. Here's the linearization of the example
program \eqref{eq:tALprog}

\begin{align}
  &\begin{aligned}
    &f_0 = \proc x::\ell \ra {\tt zero}\al
  \end{aligned}\\[0.5ex]
  &\begin{aligned}
    &f_{n>0} = \proc x::\ell \ra\\[-1ex]
    &\qquad y::\ell \la f_{n-1} \al x::\ell.\\[-1ex]
    &\qquad (\langle+\rangle) \al x::\ell \;\; y::\ell\\[-1ex]
  \end{aligned}
\end{align}


Once a function has been linearized, it is straightforward to write
down its dual: one simply reads it from top to bottom and swaps input
and output arguments to functions as in \cite{BoLeSc03}. The exact
transformation is made slightly technical by rule \eqref{eq:tAL-let};
instead of presenting it we will rather discuss in the next sections
how we practically transpose code in our implementation.

Before discussing practice, we make some remarks on our linearization
technique. First observe that in order to derive a linearization, rule
\eqref{eq:linear-let} forces to evaluate the scalar part first. The
simplest way to achieve this is to evaluate $\s{f} v$ and to store all
the pairs name-value before computing $f_v$. This technique is similar
to what is called \emph{forward sweep} in automatic differentiation
\cite{GVM91,Gri92} and it may seem that its cost is at most a constant
factor of the cost of evaluating the whole $f$, with an additional
penalty in space because all the values of the scalar part must be
hold in memory. There are, however, some important differences:
\begin{itemize}
\item The \emph{forward sweep} is only needed on the scalar part,
  while in automatic differentiation one needs to evaluate the whole
  $f$ before differentiating. In many practical cases, as in our
  example, the scalar part is negligible compared to the rest.
\item In presence of recursive calls, the forward sweep breaks head
  recursion optimization (this is the case in our example). Dually, it
  breaks tail recursion optimization in the transposed function.
\item The use of recursion may make the forward sweep far more
  expensive than just a constant factor. Consider the function
  \begin{equation}
    \begin{aligned}
      &f = \proc c::\ell \;\; d::s \ra\\[-1ex]
      &\qquad b::* \la (>) \al d::s \;\; \text{\tt zero}::s.\\[-1ex]
      &\qquad b::*\;?\\[-1ex]
      &\qquad\qquad t::s \la (\langle-\rangle) \al d::s \;\; \text{\tt one}::s.\\[-1ex]
      &\qquad\qquad x::\ell \;\; y::s \la f \al c::\ell \;\; t::s.\\[-1ex]
      &\qquad\qquad a::\ell \la (\langle*\rangle) \al x::\ell \;\; y::s.\\[-1ex]
      &\qquad\qquad b::s \la (\langle+\rangle) \al y::s\;\;\text{\tt one}::s.\\[-1ex]
      &\qquad\qquad a::\ell\;\;b::s\\[-1ex]
      &\qquad :\\[-1ex]
      &\qquad\qquad c::\ell\;\;d::s
    \end{aligned}
  \end{equation}
  Its linearization is 
  \begin{align}
    &f_0 = \proc c::\ell \ra  c::\ell\\[0.5ex]
    \label{eq:mechant-lin}
    &\begin{aligned}
      &f_d = \proc c::\ell \ra\\[-1ex]
      &\qquad x::\ell \la f_{d-1} \al c::\ell.\\[-1ex]
      &\qquad a::\ell \la (\langle*\rangle)_{\s{f}(d-1)} \al x::\ell.\\[-1ex]
      &\qquad a::\ell\\[-1ex]
    \end{aligned}
  \end{align}
  and its scalar restriction is
  \begin{equation}
    \begin{aligned}
      &\s{f} = \proc d::s \ra\\[-1ex]
      &\qquad b::* \la (>) \al d::s \;\; \text{\tt zero}::s.\\[-1ex]
      &\qquad b::*\;?\\[-1ex]
      &\qquad\qquad t::s \la (\langle-\rangle) \al d::s \;\; \text{\tt one}::s.\\[-1ex]
      &\qquad\qquad y::s \la \s{f} \al t::s.\\[-1ex]
      &\qquad\qquad b::s \la (\langle+\rangle) \al y::s\;\;\text{\tt one}::s.\\[-1ex]
      &\qquad\qquad b::s\\[-1ex]
      &\qquad :\\[-1ex]
      &\qquad\qquad d::s
    \end{aligned}
  \end{equation}
  Thus, in order to compute $f_d$ one has to compute $\s{f}(d-1)$ in
  the forward sweep, and that requires a recursive call to
  $\s{f}(d-2)$. Then one has to compute $f_{d-1}$ and the value of
  $\s{f}(d-2)$ is needed again for this. In conclusion, any recursive
  call is potentially doubled and this can make the overall cost of
  evaluating $f_v$ (or its transpose) much worse than the cost of
  evaluating $f$. The workaround is to realize that all the recursive
  calls to $\s{f}$ always happen with the same arguments, thus it is
  enough to apply a lazy evaluation strategy on the scalar parts by
  memoizing them.
\end{itemize}
Applying all these ideas, we can guarantee that the
linearization/transposition only yields a constant increase in time
and an increase in space that is at most the size of the scalar part

Finally, we remark that \tAL{} is a very limited language as it
doesn't even have support for lists. Functions that act on lists are
usually written using pattern-matching, but this interacts very poorly
with transpostion because the information on which pattern matches the
output is lost. A good compromise is to add two types $M(\ell)$ and
$M(s)$ representing infinite dimensional modules over $R$ and to
expose an interface as in the previous Section. However, this adds
some issues that will be discussed in the next section.


% Local Variables:
% mode:flyspell
% ispell-local-dictionary:"american"
% mode:TeX-PDF
% TeX-master: "transAL"
% End:
%
