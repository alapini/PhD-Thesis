%% these.tex
%% Copyright 2010 Luca De Feo
%% All rights reserved


In this chapter we describe the implementations we made of the
algorithms of the previous chapter and some experimental results.

\section{Implementation of Couveignes' algorithm}
\label{sec:implementation}

We implemented \ctwoasfimc{} as \texttt{C++} programs using the
libraries \texttt{NTL} \cite{shoup2003ntl} for finite field
arithmetic, \texttt{gf2x}~\cite{gf2x} for fast arithmetic in
characteristic $2$ and \texttt{FAAST} (see
Section~\ref{sec:artin-benchmarks}) for fast arithmetic in
Artin-Schreier towers.  We also have a Magma~\cite{MAGMA} prototype of
the same algorithm, not making use of the fast algorithms of
Chapter~\ref{cha:artin-schr-towers}.  This section mainly deals with
some tricks we implemented in order to speed up the computation.

\subsection{Building \texorpdfstring{$E[p^k]$}{E[pk]} and \texorpdfstring{$E'[p^k]$}{E[pk]}}
\label{sec:impl:torsion}

\paragraph{$p$-Torsion}
For $p\ne2$, \ctwo{} and its variants require to build the extension
$\F_q[c]$ where $c$ is a $(p-1)$-th root of $H_E$. In order to deal with
the lowest possible extension degree, it is a good idea to modify the
curve so that $[\F_q[c]:\F_q]$ is the smallest possible.

$[\F_q[c]:\F_q]$ is invariant under isomorphism, but taking a twist
can save us a quadratic extension. Let $u=c^{-2}$, the curve
\begin{equation*}
  \bar{E} : y^2 = x^3 + a_2ux^2 + a_4u^2x + a_6u^3
\end{equation*}
is defined over $\F_q[c^2]$ and is isomorphic to $E$ over $\F_q[c]$
via $(x,y)\mapsto(\sqrt{u}^2x,\sqrt{u}^3y)$. Its Hasse invariant is
$H_{\bar{E}} = (u)^{\frac{p-1}{2}}H_E = 1$, thus its $p$-torsion
points are defined over $\F_q[c^2]$.

In order to compute the $p^k$-torsion points of $E$ we build
$\F_q[c^2]$, we compute $\bar{P}$ a $p^k$-torsion points of $\bar{E}$
using $p$-descent, then we invert the isomorphism to compute the
abscissa of $P\in E[p^k]$. Since the Cauchy interpolation only needs
the abscissas of $E[p^k]$, this is enough to complete the
algorithm. Scalar multiples of $P$ can be computed without knowledge
of $y(P)$ using \hyperref[rk:montgomery]{Montgomery formulas}.

\pdfmcone{"Kummer surface" -> "Kummer variety".}  Note that for $p=2$
we use the same construction in an implicit way since we do a
$p$-descent on the Kummer variety.


\paragraph{$p^k$-Torsion points}
For $p\ne2$ we use Voloch's $p$-descent to compute the $p^k$-torsion
points iteratively as described in Section \ref{sec:C2}. To factor the
Artin-Schreier polynomial \eqref{th:voloch:cover}, we use the
algorithms from Section~\ref{sec:couveignes-algorithm} implemented in
\texttt{FAAST}.

To solve system \eqref{th:voloch:isom} we first compute
\begin{equation*}
  V(x,y) = \left(\frac{g(x)}{h^2(x)}, 
    sy\left(\frac{g(x)}{h^2(x)}\right)'\right)
\end{equation*}
through \titleref{sec:velu-formulas}.\footnote{Vélu formulas compute
  this isogeny up to an indeterminacy on the sign of the ordinate, the
  actual value of $s$ must be determined by composing $V$ with
  $\frobisog$ and verifying that it corresponds to $[p]$ by trying
  some random points.} Recall that we work on a curve having Hasse
invariant $1$, system \eqref{th:voloch:isom} can then be rewritten
\begin{equation*}
  \left\{
    \begin{aligned}
      \tilde{x}(P) &= \frac{g(x)}{h^2(x)}\\
      \tilde{y}(P) &= sy\left(\frac{g(x)}{h^2(x)}\right)'\\
      \tilde{z}(P) &= -2y\frac{h'(x)}{h(x)}
    \end{aligned}
  \right.
\end{equation*}
where $P$ is the point on the cover $C$ that we want to pull back
($\tilde{x}(P)$, $\tilde{y}(P)$ and $\tilde{z}(P)$ are just its
coordinates). After some substitutions this is equivalent to
\begin{equation*}
  \left\{
    \begin{aligned}
      \tilde{x}(P)h^2(x) - g(x) &= 0\\
      \left(\tilde{x}(P)h^2(x) - g(x) - \frac{\tilde{y}(P)}{s\tilde{x}(P)}h^2(x)\right)' &= 0
    \end{aligned}
  \right.
\end{equation*}
Then a solution in $x$ to this system is given by the GCD of the two
equations. Note that proposition \ref{th:voloch} ensures there is one
unique solution. These formulas are slightly more efficient than the
ones in~\cite[$\S$6.2]{lercier-algorithmique}.

For $p=2$ we use the library \texttt{FAAST} (for solving
Artin-Schreier equations) on top of \texttt{gf2x} (for better
performance). There is nothing special to remark about the
$2$-descent.


\subsection{Cauchy interpolation and loop}
\label{sec:impl:cauchy}
The polynomial interpolation step is done as described in Section
\ref{sec:C2-AS-FI}. As a result of this implementation, the polynomial
interpolation algorithm was added to the library \texttt{FAAST}.

The rational fraction reconstruction is implemented using a fast XGCD
algorithm on top of \texttt{NTL} and \texttt{gf2x}. This algorithm was
added to \texttt{FAAST} too.

The loop uses modular composition as in Section~\ref{sec:C2-AS-FI-MC}
in order to minimise the number of interpolations. The timings in the
next section clearly show that this non-asymptotically-optimal variant
performs much faster in practice.

To check that the rational fractions are isogenies we test their
degrees, that their denominator is a square and that they act as group
morphisms on a fixed number of random points. All these checks take a
negligible amount of time compared to the rest of the algorithm.


\subsection{Parallelisation of the loop}
\label{parallel}

The most expensive step of \ctwoasfimc{}, in theory as well as in
practice, is the final loop over the points of $E'[p^k]$. Fortunately,
this phase is very easy to parallelise with very little overhead.

Let $n$ be the number of processors we wish to parallelise on, suppose
that $[\U_k:\F_q]$ is maximal, then we make only one interpolation
followed by $\euler(p^k)/2$ modular compositions.\footnote{If
  $[\U_k:\F_q]$ is not maximal, the parallelisation is
  straightforward: we simply send one interpolation to each processor
  in turn.} We set $m=\left\lfloor\frac{\euler(p^k)}{2n}\right\rfloor$
and we compute the action of $\frobisog^{m}$ on $E[p^k]$ as in
Section~\ref{sec:C2-AS-FI-MC}:
\begin{equation*}
  F^{(m)}(X) = F(X) \circ \cdots \circ F(X) \bmod T(X)
  \;\text{,}
\end{equation*}
this can be done with $\Theta(\log m)$ modular compositions via a
binary square-and-multiply approach as in
Section~\ref{sec:modular-composition}.

Then we compute the $n$ polynomials
\begin{equation*}
  A_{mi}(X) = A_{m(i-1)}(X) \circ F^{(m)}(X) \bmod T(X)
\end{equation*}
and distribute them to the $n$ processors so that they each work on a
separate slice of the $A_i$'s. The only overhead is $\Theta(\log
(\ell/n))$ modular compositions with coefficients in $\F_q$, this is
acceptable in most cases.

\section{Implementation of \texorpdfstring{\ctwoud{}}{C2-UD}}
\label{sec:implementation-c2-ud}
We modified our \texttt{C++} implementation of \ctwoasfimc{} to obtain
two variants of \ctwoud{}.

The first one takes an integer $k$ and looks for all isogenies of
degree $p^c\ell$ with $\ell$ prime to $p$, $\ell<\euler(p^k)/4$
and $c$ arbitrary. This is done by slightly modifying the modular
composition step of Section~\ref{sec:C2-AS-FI-MC}. Suppose we know an
interpolating polynomial $A_0$, that we view as a morphisms $E[p^k]\ra
E'[p^k]$ such that
\begin{equation}
  \label{eq:184}
  A_0\circ[n](P) = [n](P')
  \quad\text{for any $n$.}
\end{equation}
Then we compose with the Frobenius isogeny $\frobisog$
\begin{equation}
  \label{eq:185}
  A_1 \eqdef \frobisog\circ A_0:E[p^k]\ra {E'}^{(p)}[p^k]
  \text{,}
\end{equation}
where ${E'}^{(p)}$ is the curve
\begin{equation}
  \label{eq:186}
  E^{(p)}\;:\; y^2 = x^3+{a'}^px + {b}'^p
  \text{.}
\end{equation}
So $A_1$ is one of the polynomials that Couveignes algorithm computes
when looking for an isogeny between $E$ and ${E'}^{(p)}$. If $q=p^d$,
iterating $d$ times this construction, we fall back on $E'$, as we
would have if we had directly applied the
\hyperref[sec:curves-over-finite]{Frobenius automorphism} as in
Section~\ref{sec:C2-AS-FI-MC}. Thus, paying an additional factor of
$\log_p q$, we can compute any isogeny of degree $p^c\ell$ with
arbitrary $c$ and $\ell$ bounded as before.

When $\log_pq$ is large, the previous variant becomes unpractical. We
implemented a second variant using the algorithm described in
Section~\ref{sec:C2:non-prime}, this allows to compute any isogeny of
degree $\ell\le\euler(p^k)/4$, even if $p$ divides $\ell$. The
asymptotic cost of this variant is the same as one run of \ctwoasfimc{},
because the search for $\ell$ prime to $p$ dominates.



\section{Implementation of Lercier-Sirvent}
We implemented a Magma prototype of \titleref{alg:bmss}
and \titleref{alg:le-si}. In both cases we did
not take Remark~\ref{rk:bmss} into account, and only implemented the
variant using rational fraction reconstruction. We used Magma native
support for $p$-adics to construct the field $\Q_q$.

\pdfmctwo{Atkin's MODULAR polynomial. Not "canonical".}
Instead of the classical modular polynomials $\Modpol_\ell$ we used
Atkin's modular polynomials $\Modpol^\ast_\ell$ since they have
smaller coefficients and degree; this does not change the other steps
of the algorithm.

The modular polynomials were not computed on the fly as suggested in
Section~\ref{sec:lercier-sirvent}, instead they were taken from the
tables precomputed in Magma, however this is not expected to
significantly affect the running time for the sizes we treat in the
next section (compare with \cite[Table~3]{sutherland10:modpol}).


% Local Variables:
% mode:flyspell
% ispell-local-dictionary:"american"
% mode:TeX-PDF
% TeX-master: "../these"
% mode:reftex
% End:
%
% LocalWords:  Schreier Artin pseudotrace Frobenius bivariate Joux Sirvent FFT
% LocalWords:  Couveignes isogenies Schoof isogeny cryptosystems Lercier Hasse
% LocalWords:  precomputation arithmetics polylogarithmic Karatsuba precomputes
% LocalWords:  endomorphisms 
