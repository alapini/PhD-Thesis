%% these.tex
%% Copyright 2010 Luca De Feo
%% All rights reserved


Let $E$ and $E'$ be two elliptic curves defined over $\K$, by finding
an \emph{explicit isogeny} we mean to find a $\K$-rational function
from $E(\clot{\K})$ to $E'(\clot{\K})$ such that the map it defines is
an isogeny.

In this chapter we are interested in finding explicit isogenies of
ordinary elliptic curves over finite fields. In what follows $\F_q$
will be a finite field of characteristic $p$, and $d$ the positive
integer such that $q=p^d$.

\pdfmctwo{Given more details on "what's changed".}
Parts of this chapter and of the following have been published
in \cite{df10}. However, the complexity analysis we give in
Proposition~\ref{th:lercier-sirvent} benefits from recent advances on
the computation of modular polynomials~\cite{sutherland10:modpol};
this in turn changes the relative ranking of the algorithms of this
chapter in terms of complexity. We also present a new algorithm in
Section~\ref{sec:bounded}.


\section{Overview}
\label{sec:history}

The problem of computing an explicit degree $\ell$ isogeny between two
given elliptic curves over $\F_q$ was originally motivated by point
counting methods based on Schoof's
algorithm~\cite{atkin88,elkies98,schoof95}. A review of the most
efficient algorithms to solve this problem is given
in~\cite{bostan+morain+salvy+schost08}, together with a new
quasi-optimal algorithm that we will review in Section~\ref{sec:bmss}.

All the algorithms of~\cite{bostan+morain+salvy+schost08} only work
when $\ell\ll p$. The case where $p$ is small compared to $\ell$ was
first treated by Couveignes in~\cite{couveignes94}, making use of
formal groups. The complexity of his method is $\tildO(\ell^3\log q)$ operations in
$\F_p$ assuming $p$ is constant, however it has an exponential
complexity in $\log p$.

Later, Lercier~\cite{lercier96} gave an algorithm specific to
characteristic $2$, that uses some linear properties of the problem to
build a linear system from whose solution the isogeny can be deduced.
Its complexity is conjectured to be $\tildO(\ell^3\log q)$ operations
in $\F_p$, but it has a much better constant factor
than~\cite{couveignes94}. At the moment we write, this is by many
orders of magnitude the fastest algorithm to solve practical instances
of the problem when $p=2$, thus being the \emph{de facto} standard for
cryptographic use.

Couveignes, again, proposed an algorithm in~\cite{couveignes96}
working for any $p$, based on the structure of the $p^k$ torsion of
ordinary elliptic curves. Using improvements
from~\cite{couveignes00,df+schost09,df10}, this algorithm has a
quadratic complexity in $\ell$. We review the original algorithm as
well as its improved variants in Sections~\ref{sec:C2}
to~\ref{sec:bounded}.

\pdfmcone{A little more crypto here.}
After the discovery of $p$-adic alternatives to Schoof's
algorithm~\cite{satoh00,fouquet+gaudry+harley00}, interest in computing
isogenies in small characteristic was lost for some time. However,
other cryptographic applications of isogenies soon appeared.  The
\index{GHS~attack}GHS attack uses Weil descent to reduce the
\index{discrete~logarihm~problem}discrete logarithm problem
(\index{DLP|see{discrete logarithm problem}}DLP) of an elliptic curve
over a binary field of composite degree to the DLP of an hyperelliptic
curve over a smaller field~\cite{gaudry+hess+smart02,GHS,hess03}. A
similar application is the reduction of the DLP of some genus $3$
hyperelliptic curves to the DLP of genus $3$ non-hyperelliptic
curves~\cite{smith09}. Isogeny graphs have been used to construct hash
functions~\cite{charles+lauter+goren09} and to compute Hilbert class
polynomials and modular
polynomials~\cite{sutherland10:hilbert,sutherland10:modpol}. New
cryptographic protocols based on isogenies have also been proposed:
Rostovtsef and Stolbunov~\cite{rostovtsev+stolbunov06} construct a
Diffie-Hellman key exchange based on a DLP-like problem in a cycle of
isogenous curves; Teske~\cite{mauer+menezes+teske01,teske06}
constructs a trapdoor cryptosystem by hiding a GHS-vulnerable curve
behind a random path of isogenies.

On the wave of the renewed interest for isogenies, two $p$-adic
algorithms were recently proposed by Joux and
Lercier~\cite{joux+lercier06} and Lercier and
Sirvent~\cite{lercier+sirvent08} to compute isogenies in arbitrary
characteristic. The former method has complexity $\tildO(\ell^2(1 +
\ell/p)\log q)$ operations in $\F_p$, which makes it well adapted to
the case where $p\sim\log q$.  The latter has binary complexity
$\tildO(\ell^3)$, it is the best algorithm to compute isogenies when
the characteristic is not fixed. We review the second algorithm in
Section~\ref{sec:lercier-sirvent}.


\section{Vélu formulas}
\label{sec:velu-formulas}


\begin{figure}
  \centering
  \[\xymatrix{
    E \ar[r]^{[m]}\ar@/_1pc/[rrr]_{\I'} & E \ar[r]^\I & E' \ar[r]^{\frobisog^n} & E'^{(p^n)}
  }\]
  \label{fig:fact}
  \caption{Factorization of an isogeny. $\I'$ has kernel $E[m]\oplus\ker\I$.}
\end{figure}

Since an isogeny can be uniquely factored in the product of a
separable and a purely inseparable isogeny, we focus on the problem of
computing explicit separable isogenies. Furthermore one can factor out
multiplication-by-$m$ maps, thus reducing the problem to compute
explicit separable isogenies with cyclic kernel (see figure
\ref{fig:fact}).

In the rest of this chapter, unless otherwise stated, by
$\ell$-isogeny we mean a separable isogeny with kernel isomorphic to
$\Z/\ell\Z$.


For any finite subgroup $G \subset E(\clot{\K})$, Vélu
formulas~\cite{velu71} give in a canonical way an elliptic curve
$\bar{E}$ and an explicit separable isogeny $\I:E\rightarrow \bar{E}$
such that $\ker\I=G$. The isogeny is $\K$-rational if and only if the
polynomial vanishing on the abscissas of $G$ belongs to $\K[X]$.

The isogeny computed by Vélu formulas is the map
\ifafourps
\begin{equation}
  \label{eq:155}
  \I(P) = \left(x(P) + \sum_{Q\in G\diffset\{\0\}}x(P+Q) - x(Q),\right.
    \left.y(P) + \sum_{Q\in G\diffset\{\0\}}y(P+Q) - y(Q)\right)
  \text{.}
\end{equation}
\else
\begin{multline}
  \label{eq:155}
  \I(P) = \left(x(P) + \sum_{Q\in G\diffset\{\0\}}x(P+Q) - x(Q),\right.\\
    \left.y(P) + \sum_{Q\in G\diffset\{\0\}}y(P+Q) - y(Q)\right)
  \text{.}
\end{multline}
\fi
Using the addition formulas it is straightforward to obtain the
coefficients of the curve $\bar{E}$ and the explicit isogeny.  For
simplicity, we do so only for the case $p\ge3$ and $E$ in the form
\begin{equation}
  \label{eq:160}
  E \;:\; y^2 =  x^3 + a_2x^2 + a_4x + a_6
\end{equation}
(note that this is always possible by
Proposition~\ref{th:simplified-weierstrass}). 

We set $G^\ast=G\diffset\{\0\}$. We denote by $f,f'$ the two
functions in $\K(E)$
\begin{equation}
  \label{eq:162}
  \begin{aligned}
    f(P) &= x(P)^3 + a_2x(P)^2 + a_4x(P) + a_6
    \text{,}\\
    f'(P) &= 3x(P)^2 + 2a_2x(P) + a_4
    \text{.}
  \end{aligned}
\end{equation}
From the \hyperref[eq:121]{addition formulas}, after some calculations
(see Appendix~\ref{cha:proof-velus-formulas} for an automatic proof of
this calculation), Eq.~\eqref{eq:155} is equivalent to
\ifafourps
\begin{equation}
  \label{eq:161}
  \I(x,y) = \left(x + \sum_{Q\in G^\ast} \frac{f'(Q)}{x-x(Q)} + \frac{2f(Q)}{(x-x(Q))^2}\text{,}\right.
  \left.y + \sum_{Q\in G^\ast} -\frac{yf'(Q)}{(x-x(Q))^2} - \frac{4yf(Q)}{(x-x(Q))^3}\right)
  \text{.}
\end{equation}
\else
\begin{multline}
  \label{eq:161}
  \I(x,y) = \left(x + \sum_{Q\in G^\ast} \frac{f'(Q)}{x-x(Q)} + \frac{2f(Q)}{(x-x(Q))^2}\text{,}\right.\\
  \left.y + \sum_{Q\in G^\ast} -\frac{yf'(Q)}{(x-x(Q))^2} - \frac{4yf(Q)}{(x-x(Q))^3}\right)
  \text{.}
\end{multline}
\fi

Observe that if $Q\in G^\ast$ is a $2$-torsion point, then $f(Q)=0$;
while if $Q$ is not a $2$-torsion point, $x(Q)$ is counted twice in
the sum of the previous equation. Then, the denominator of $\I_x$ is
\begin{equation}
  \label{eq:158}
  h(x) \eqdef \prod_{Q\in G^\ast}(x - x(Q))
  \text{,}
\end{equation}
and by Eq.~\eqref{eq:161} it is evident that
\begin{equation}
  \label{eq:159}
  \I(x,y) = \left(\frac{g(x)}{h(x)}, y\left(\frac{g(x)}{h(x)}\right)'\right)
\end{equation}
for some polynomial $g$.  Finally, if we set \pdfmcfour{There was a
  huge mistake in this equation! This whole page has changed to
  reflect the corrections.}
\begin{equation}
  \label{eq:164}
  t = \sum_{Q\in G^\ast} f'(Q)\text{,}
  \qquad
  u = \sum_{Q\in G^\ast} 2f(Q)\text{,}
  \qquad
  w = u + \sum_{Q\in G^\ast} x(Q)f'(Q)\text{,}
\end{equation}
the isogenous curve has equation
\begin{equation}
  \label{eq:163}
  \bar{E}\;:\;y^2 = x^3 + a_2x^2 + (a_4-5t)x + a_6 - 4a_2t - 7w
  \text{.}
\end{equation}

\begin{remark}
  Eqs.~\eqref{eq:161} and~\eqref{eq:163} can be used to deduce the
  isogeny and the image curve from the knowledge of the kernel.
  \pdfmctwo{The sentence "When the isogenous curve is of no interest"
    wasn't really useful. I stripped it.}  It is often more convenient
  to use the reformulation given by Elkies~\cite{elkies98}
  \begin{equation}
    \label{eq:157}
    \frac{g(x)}{h(x)} = x + \sum_{Q\in G^\ast}x - x(Q) - \frac{f'(x)}{x-x(Q)} + \frac{2f(x)}{(x-x(Q))^2}
  \end{equation}
  (this equality is shown in Appendix~\ref{cha:proof-velus-formulas},
  too), which implies
  \begin{equation}
    \label{eq:165}
    \frac{g(x)}{h(x)} = \ell x - p_1 - f'(x)\frac{h'(x)}{h(x)} -
    2f(x)\left(\frac{h'(x)}{h(x)}\right)'
    \text{,}
  \end{equation}
  where $p_1$ is the first power sum of $h$.  

  Thus, from the knowledge of $h(x)$, one can deduce the isogeny and
  the isogenous curve in $O(\Mult(\deg\I))$ operations in $\K$.
\end{remark}

Given two curves $E$ and $E'$, Vélu formulas reduce the problem of
finding an explicit isogeny between $E$ and $E'$ to that of finding
the kernel of an isogeny between them. Once the polynomial $h(X)$
vanishing on $\ker\I$ is found, the explicit isogeny is computed
composing Vélu formulas with the isomorphism between $\bar{E}$ and
$E'$ as in figure \ref{fig:velu}.

\begin{figure}
  \centering
  \[\xymatrix{
    E \ar[r]^{\bar{\I}} \ar[rd]^\I & \bar{E} \ar[d]^{\simeq}\\
    & E'
  }\]
  \caption{Using Vélu formulas to compute an explicit isogeny.}
  \label{fig:velu}
\end{figure}




% Local Variables:
% mode:flyspell
% ispell-local-dictionary:"american"
% mode:TeX-PDF
% mode:reftex
% TeX-master: "../these"
% End:
%
% LocalWords:  Schreier Artin pseudotrace frobenius bivariate Joux Sirvent FFT
% LocalWords:  Couveignes isogenies Schoof isogeny cryptosystems Lercier
% LocalWords:  precomputation arithmetics polylogarithmic Karatsuba
