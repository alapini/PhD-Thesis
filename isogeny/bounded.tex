\section{Smallest degree isogeny}
\label{sec:bounded}

We now present an extension to Couveignes algorithm that could be
useful in cryptographic application. It is well known that two curves
having the same number of points over a finite field are isogenous,
however this doesn't say anything on the degree of the isogeny
connecting them. Given two elliptic curves $E$ and $E'$ defined over
$\F_q$ and having the same number of points, we want to find the
smallest degree isogeny between them.

The simplest solution is to take any algorithm computing a fixed
degree isogeny and try all the degrees until an isogeny is found. If
$\ell$ is the degree of the smallest isogeny, this of course adds a
factor $\ell$ to the complexity of any polynomial time algorithm.

Couveignes' algorithm can be easily adapted to solve this problem at
no additional cost. We call this algorithm C2SD and we will only
discuss its most efficient variant C2SD-AS-FI.

Observe that, apart for the choice of $k$, the computation of $E[p^k]$
and the polynomial interpolation step do not depend at all on
$\ell$. The degree of the isogeny only comes into play in the last
part of the Cauchy interpolation, that is the rational function
reconstruction. We study more in detail this last step.


\paragraph{Rational Function Reconstruction}
Rational function reconstruction takes as input a degree $n$
polynomial $T$, a polynomial $A$ of degree less than $n$ and a target
degree $m\le n$ and outputs the unique rational function such that
\begin{equation*}
  A \equiv \frac{R}{V} \bmod T
\end{equation*}
and $\deg R < m$, $\deg V \le n-m$. This is done by computing a Bezout
relation $AV + TU = R$ with the expected degrees via an XGCD
algorithm. If a classical XGCD algorithm is used, one simply computes
all the lines
\begin{equation}
  \label{eq:XGCD}
  \begin{aligned}
    R_0 &= T, & U_0 &= 1, & V_0 &= 0,\\
    R_1 &= A, & U_1 &= 0, & V_1 &= 1,\\
    R_{i-1} &= Q_iR_i + R_{i+1}, & U_{i+1} &= U_{i-1}-Q_iU_i, & V_{i+1} &= V_{i-1}-Q_iV_i
  \end{aligned}
\end{equation}
and stops as soon as a remainder $R_{i+1}$ with $\deg R_{i+1}<m$ is
found. If a fast XGCD algorithm as \cite[Algo. 11.4]{vzGG} is used,
one directly aims at the two lines
\begin{equation}
  \label{eq:FastGCD}
  \begin{aligned}
    R_{h-2} &= Q_{h-1}R_{j-1} + R_h\\
    R_{h-1} &= Q_hR_h + R_{h+1}
  \end{aligned}
\end{equation}
such that $\deg R_{h+1} < m \le \deg R_h$ without computing the
intermediate lines.

When looking for an $\ell$-isogeny, one simply sets
$m=\ell+1$. Observe that if the algorithm doesn't return a rational
fraction $\frac{R}{V}$ with $\deg R = \ell$ and $\deg V = \ell -1 $,
then no such fraction congruent to $A$ modulo $T$ exists.

If $\ell$ is not \emph{a priori} known, we can still use the fact that
a separable isogeny with cyclic kernel must have $\deg R = \deg V +
1$. In fact if we suppose $R = R_i$ and $V = V_i$, then
\begin{align*}
  \deg T &= \deg V_{i+1} + \deg R_i,\\
  \deg R_i - \deg V_i &= \deg R_{i-1} - \deg V_{i+1}
\end{align*}
implies
\begin{equation*}
  \deg T + 1 = \deg R_{i-1} + \deg R_i  
  \;\text{.}
\end{equation*}
Hence, if $A$ is congruent to an $\ell$-isogeny with $\ell =
\left\lfloor\frac{\deg T}{2}\right\rfloor - t$ for some $t\ge0$, then
\begin{equation}
  \label{eq:degseq}
  \deg R_{i-1} =
  \left\lceil\frac{\deg T}{2}\right\rceil + t + 1 >
  \left\lfloor\frac{\deg T}{2}\right\rfloor - t = \deg R_i
  \;\text{.}
\end{equation}
Thus we can recover any isogeny having degree less than
$\left\lfloor\frac{\deg T}{2}\right\rfloor$ using either a classical
or a fast XGCD algorithm setting $m = \left\lceil\frac{\deg
    T}{2}\right\rceil + 1$.


\paragraph{Recognising an isogeny}
Once we have a rational fraction with the required degree, we have to
test if it really is an isogeny. In order to understand how often we
have to make this test, we introduce some more terminology. Let $n_i =
\deg R_i$, we call $(n_0,\ldots,n_r)$ the \emph{degree sequence} of
$A$ and $T$; a degree sequence is said \emph{normal} if $n_i = n_{i+1}
+ 1$ for any $i$.

\begin{proposition}
  \label{th:normseq}
  Let $f,g\in\F_q[X]$ be uniformly chosen random polynomials of
  respective degrees $n_0>n_1>0$ and let $(n_0, n_1, \ldots, n_r)$ be
  their degree sequence. For $0\le i < n_1$ define the binary random
  variables $X_i = 1 \Leftrightarrow i\in(n_0,n_1,\ldots,n_r)$, then
  the $X_i$ are independent random variables and $\mathrm{Prob}(X_i=0) =
  \frac{1}{q}$.
\end{proposition}
\begin{proof}
  Pairs of polynomials $f,g$ are in bijection with the GCD-sequence
  $(R_r, Q_r, \ldots, Q_1)$ constituted by their GCD and the quotients
  of the GCD algorithm. To each such sequence is associated a degree
  sequence
  \begin{equation*}
    (n_0,n_1,\ldots,n_r) =
    \left(\deg R_r + \sum_{i=1}^r\deg Q_i, \ldots, \deg R_r + \sum_{i=1}^1\deg Q_i, \deg R_r\right)
    \;\text{,} 
  \end{equation*}
  thus for any given degree sequence there are
  \begin{equation*}
    (q-1)q^{n_0-n_1}\cdot(q-1)q^{n_1-n_2}\cdot\cdots\cdot(q-1)q^{n_r} =
    (q-1)^{r+1}q^{n_0}
  \end{equation*}
  GCD-sequences.
 
  Let $I$ and $O$ be two disjoints subsets of $\{X_i\}$, the number of
  GCD-sequences such that $X\in I \Rightarrow X=1$ and $X\in O
  \Rightarrow X=0$,
  \begin{equation*}
     \sum_{s=0}^{n_1-\card{I}-\card{O}}\binom{n_1-\card{I}-\card{O}}{s}(q-1)^{s+2+\card{I}}q^{n_0} =
    (q-1)^{2+\card{I}}q^{n_0}q^{n_1-\card{I}-\card{O}}
    \;\text{.}
  \end{equation*}
  There are $(q-1)^2q^{n_0}q^{n_1}$ pairs of polynomials of degrees
  $n_0,n_1$, thus
  \begin{equation}
   \label{th:normseq:prob}
    \mathrm{Prob}\bigl(\{X = 1 \mid X\in I\},
    \{X=0\mid X\in O\}\bigr) = \left(\frac{q-1}{q}\right)^{\card{I}}\left(\frac{1}{q}\right)^{\card{O}}
    \;\text{.}
  \end{equation}
  The claim follows.
\end{proof}

Degree sequences associated to isogenies are in general not normal, in
fact if $\ell\le\left\lfloor\frac{\deg T}{2}\right\rfloor-t$, equation
\eqref{eq:degseq} shows that there must be at least a gap of degree
$2c$ in the degree sequence. Heuristically, we can expect that if the
polynomial $A$ doesn't correspond to an isogeny, then $A$ and $T$ act
like random polynomials, thus, by the proposition above, the
probability that $A$ looks like an isogeny of degree
$\ell\le\left\lfloor\frac{\deg T}{2}\right\rfloor-t$ is less than $\frac{1}{q^{2t}}$.

Therefore, by choosing an appropriate $t\in O(\log_q p^k)$, C2SD can
find any isogeny of degree less than $\frac{p^k-1}{4}-t$ at the same
cost of one run of C2. Also notice that C2SD is not restricted to
isogenies of degree prime to $p$ as it was already mentioned in
Section \ref{sec:C2:non-prime}.

No other method for computing isogenies is known to have a similar
generalisation, this makes C2SD-AS-FI interesting for practical
applications.




% Local Variables:
% mode:flyspell
% ispell-local-dictionary:"british"
% mode:TeX-PDF
% TeX-master: "ec-isogeny"
% End:
%
% LocalWords:  Schreier Artin pseudotrace frobenius bivariate Joux Sirvent FFT
% LocalWords:  Couveignes isogenies Schoof isogeny cryptosystems Lercier
% LocalWords:  precomputation arithmetics polylogarithmic Karatsuba precomputes
% LocalWords:  endomorphisms  isogenous
