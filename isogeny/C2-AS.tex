\section{The algorithm \alg{C2-AS}}
\label{sec:C2-AS}

One of the most expensive steps of \ctwo{} is the resolution of an
Artin-Schreier equation in an extension field $\U_i$. We call \ctwoas{}
the variant of Couveignes' algorithm that uses the fast Artin-Schreier
towers of Chapter~\ref{cha:artin-schr-towers}; in this section we
analyze the complexity of \ctwoas{}

\subsection{Complexity analysis}
\label{sec:C2-AS:complexity}
We borrow the complexity notations $\Lift(i)$ (Theorem~\ref{theo:L})
and $\Ptr(i)$ (Theorem~\ref{th:b-pseudo}) from
Chapter~\ref{cha:artin-schr-towers}.

\paragraph{$p$-torsion}
The construction of $\F_q[c]$ may be done in many ways. The only
requirements of Theorem~\ref{th:cantor}
\begin{enumerate}
\item that its elements have a representation as elements of
  $F_p[X]/Q_1(X)$ for some irreducible polynomial $Q_1$,
\item that either $(d,p)=1$ or $\deg Q_1' + 2 = \deg Q_1$.
\end{enumerate}
\pdfmctwo{Yes, there are deterministic algorithms to generate
  irreducible polynomials, and it would be easier to compute with Q1,
  if it were sparse. But if I want to have a good chance of meeting
  condition 2 when p divides d, I must take a random polynomial and
  certainly not a sparse one! This was already said in the remark.}
Selecting a random polynomial $Q_1$ and testing for irreducibility is
usually enough to meet these conditions, as we saw in
Remark~\ref{rk:comp-minim-polyn}.  This costs $O\bigl(pd\Mult(pd)\log
(pd)\log(p^2d)\bigr)$ according to \cite[Th.  14.42]{vzGG}.

Now we need to compute the embedding $\F_q\subset\F_q[c]$. Supposing
$\F_q$ is represented as $\F_p[X]/Q_0(X)$, we factor $Q_0$ in
$\F_q[c]$, which costs $O\bigl(pd\Mult(pd^2)\log d\log p\bigr)$ using
\cite[Coro. 14.16]{vzGG}. Then the most naive technique to express the
embedding is linear algebra. This requires the computation of $pd$
elements of $\F_q[c]$ at the expense of $\Theta\bigl(pd\Mult(pd)\bigr)$
operations in $\F_p$, then the inversion of the matrix holding such
elements, at a cost of $\Theta\bigl((pd)^\omega\bigr)$ operations. This is
certainly not optimal, yet this phase will have negligible cost
compared to the rest of the algorithm.

Now we can compute $c$ and $c'$ by factoring the polynomials
$Y^{p-1}-H_E$ and $Y^{p-1}-H_{E'}$ in $\F_p[X]/Q_1(X)$. This costs
\[O\bigl((p\ModComp(pd) + \ModComp(p)\Mult(pd) + \Mult(p)\Mult(pd)\log
p)(\log^2 p+\log d)\bigr)\] using \cite[Section 3]{kaltofen+shoup97}.

Finally, computing the determinants needed by Gunji's formulas takes
$\Theta(p^2)$ multiplications in $\F_q[c]$, that is
$\Theta\bigl(p^2\Mult(pd)\bigr)$.

Letting out logarithmic factors, the overall cost of this phase is
\begin{equation}
  \label{eq:gunji-complexity}
  \tildO\bigl(p^2d^3 + p\ModComp(pd) + \ModComp(p)pd + (pd)^\omega \bigr)
\end{equation}


\paragraph{$p^k$-torsion}
Application of Voloch formulas requires at each of the levels
$\U_2,\ldots,\U_k$
\begin{enumerate}
\item to solve equation \eqref{th:voloch:cover} by factoring an
  Artin-Schreier polynomial,
\item to solve the system \eqref{th:voloch:isom}.
\end{enumerate}
If we assume the worst case $[\U_2:\U_1] = p$, according to
Theorem~\ref{theo:main}, at each level $i$ the first step costs
\begin{equation*}
  O\bigl((pd)^\omega i + {\Ptr}(i-1) + \Mult(p^{i+1}d)\log p\bigr)
\end{equation*}
while the second takes the GCD of two degree $p$ polynomials in
$\U_i[X]$ for each $i$ (see Section \ref{sec:implementation}), at a
cost of $O\bigl(\Mult(p^{i+1}d)\log p\bigr)$ operations using a
\hyperref[sec:eucl-algor-rati]{fast Euclidean algorithm}.

Summing up over $i$, the total cost of this phase up to logarithmic
factors is
\begin{equation}
  \label{eq:C2-AS:complexity:p^k}
  \tildO_{p,d,\log\ell}\left((pd)^\omega \log_p^2\ell + p^2\ell d\log_p^4\ell +
  \frac{\ell}{p}\ModComp(pd)\right)
  \;\text{.}  
\end{equation}
Also notice that there is no need to store a $p^{k-1}d\times p^{k-1}d$
matrix to solve the Artin-Schreier equation, thus the space
requirements are not anymore quadratic in $\ell$.


\paragraph{Interpolation}
The interpolation phase does not change in a significant way: one
needs first to interpolate a degree $p^k-1$ polynomial with
coefficients in $\U_k$, then use
\titleref{alg:push-down} to obtain the corresponding
polynomial in $\F_q[X]$ and finally do a rational fraction
reconstruction.

The first step costs $O\bigl(\Mult(p^{2k}d)\log p^k\bigr)$ using fast
techniques as in Section~\ref{sec:chin-rema-algor}, then converting to
$\F_q[c][X]$ takes $O\bigl(p^k\Lift(k-1)\bigr)$ and further
converting to $\F_q[X]$ takes $\Theta\bigl((pd)^2\bigr)$ by linear
algebra. The \hyperref[sec:eucl-algor-rati]{rational function
  reconstruction} then takes $O\bigl(\Mult(p^kd)\log p^k\bigr)$.

The overall complexity of one interpolation is then
\begin{equation}
  \label{eq:C2-AS:complexity:interp}
  O\bigl(\Mult(\ell^2d)\log_p\ell + \ell\Lift(k-1) + (pd)^2\bigr)
  \;\text{.}
\end{equation}
Remember that this step has to be repeated an average number of
$\euler(p^k)/4$ times, thus the dependency of \ctwoas{} in $\ell$ is still cubic.



% Local Variables:
% mode:flyspell
% ispell-local-dictionary:"american"
% mode:TeX-PDF
% TeX-master: "../these"
% mode:reftex
% End:
%
% LocalWords:  Schreier Artin pseudotrace Frobenius bivariate Joux Sirvent FFT
% LocalWords:  Couveignes isogenies Schoof isogeny cryptosystems Lercier
% LocalWords:  precomputation arithmetics polylogarithmic Karatsuba
% LocalWords:  irreducibility
