
\section{Couveignes' algorithm}
\label{sec:C2}

In this section we describe Couveignes' algorithm to compute isogenies
between ordinary elliptic curves in arbitrary characteristic, as it
was originally presented in~\cite{couveignes96}. In the next sections
we will discuss more efficient variants of this algorithm; to
distinguish between the variants, we call C2 this original version (it
is to be understood that C1 would be the code-name of the other
algorithm by Couveignes, appeared in\cite{couveignes94}, that we will
not present in this document).

C2 takes as input two elliptic curves $E, E'$ and an integer $\ell$
prime to $p$, and it returns, if it exists, an $\F_q$-rational isogeny
of degree $\ell$ between $E$ and $E'$. It only works in odd
characteristic.

\subsection{The original algorithm}
Suppose there exists an $\F_q$-rational isogeny
$\I:E\rightarrow E'$ of degree $\ell$. Since $\ell$ is prime to $p$
one has $\I(E[p^k]) = E'[p^k]$ for any $k$.

Recall that $E[p^k]$ and $E'[p^k]$ are cyclic groups. C2 iteratively
computes generators $P_k,P_k'$ of $E[p^k]$ and $E'[p^k]$
respectively. Now C2 makes the guess $\I(P_k) = P_k'$; then, if $\I$
is given by rational fractions as in~\eqref{eq:159},
\begin{equation}
  \label{eq:C2:I}
  \frac{g\bigl(x([i]P_k)\bigr)}{h\bigl(x([i]P_k)\bigr)} = x([i]P_k')
  \quad\text{for $i\in\Z/p^k\Z$,} 
\end{equation}
and \hyperref[sec:velu-formulas]{Vélu formulas} imply that
\begin{equation}
  \label{eq:velu-deg}
  \deg g = \deg h + 1 = \ell
  \text{.}
\end{equation}


Using \eqref{eq:C2:I} one can compute the rational fraction
$\frac{g(X)}{h(X)}$ through Cauchy interpolation over the points of
$E[p^k]$ for $k$ large enough. C2 takes $p^k > 4\ell - 2$,
interpolates the rational fraction and then checks that it corresponds
to the restriction of an isogeny to the $x$-axis. If this is the case,
the whole isogeny is computed through Vélu formulas and the algorithm
terminates. Otherwise the guess $\I(P_k) = P_k'$ was wrong, then C2
computes a new generator for $E'[p^k]$ and starts over again.

We now go through the details of the algorithm.

\paragraph{The $p$-torsion}
The computation of the $p$-torsion points follows from the work of
Gunji \cite{gunji76}. Here we suppose $p\ne2$.

\begin{definition}
  \label{def:hasse}
  Let $E$ have equation $y^2 = f(x)$. The \emph{Hasse invariant} of
  $E$, denoted by $H_E$, is the coefficient of $X^{p-1}$ in
  $f(X)^{\frac{p-1}{2}}$.
\end{definition}

Gunji shows the following proposition.

\begin{proposition}
  \label{th:gunji}
  Let $c_1,\ldots,c_{p-1}$ be the roots of $X^{p-1}-H_E$ in its
  splitting field. The abscissas of the abscissas of the $p$-torsion
  points of $E$ are given by
  \[X_i^p = \frac{\Delta_0^2 - a_6c_i^2\Delta_1^2}{4c_i^2}\text{,}\]
  where $\Delta_0$ and $\Delta_1$ are the following determinants
  \begin{equation}
    \begin{gathered}
      \Delta_0 = \begin{vmatrix}
        \beta_1 & \alpha_2 & 0 & 0 & \ldots & 0\\
        \delta_1 & \gamma_2 - c^2 & \beta_3 & \alpha_4 & \ddots & \vdots \\
        0 & \delta_2 & \gamma_3 - c^2 & \beta_4 & \ddots & 0 \\
        \vdots & \ddots & \delta_3 & \ddots & \ddots & \alpha_r \\
        \vdots & & \ddots & \ddots & \ddots & \beta_r \\
        0 & \ldots & \ldots & 0 & \delta_{r-1} & \gamma_r - c^2
      \end{vmatrix}\text{,}\\
      \Delta_1 = \begin{vmatrix}
        \gamma_2 - c^2 & \beta_3 & \alpha_4 & \ldots & 0 \\
        \delta_2 & \gamma_3 - c^2 & \beta_4 & \ddots & \vdots \\
        0 & \delta_3 & \ddots & \ddots & \alpha_r \\
        \vdots & \ddots & \ddots & \ddots & \beta_r \\
        0 & \ldots & 0 & \delta_{r-1} & \gamma_r - c^2
      \end{vmatrix}\text{,}
    \end{gathered}
  \end{equation}
  with $r = \frac{p-1}{2}$, $\alpha_\nu = \nu(\nu-1)a_6$, $\beta_\nu =
  \nu(\nu-\frac{1}{2})a_4$, $\gamma_\nu = \nu^2a_2$ and $\delta_\nu =
  \nu(\nu+\frac{1}{2})$ ($\Delta_1$ is set to $1$ when $r = 1$).
\end{proposition}

In what follows we let $c$ be any $(p-1)$-th root of $H_E$. Then
Gunji's formulas imply that the $p$-torsion points are defined in
$\F_q[c]$ and their abscissas are defined in $\F_q[c^2]$.


\paragraph{The $p^k$-torsion}
$p^k$-torsion points are iteratively computed via $p$-descent. The
basic idea is to split the multiplication map as $[p] = \frobisog\circ
V$ and invert each of the components. The purely inseparable isogeny
$\frobisog$ is just a Frobenius map and the separable isogeny $V$ can
be computed by \ref[sec:velu-formulas]{Vélu formulas} once the
$p$-torsion points are known. Although this is reasonably efficient,
pulling $V$ back may involve factoring polynomials of degree $p$ in
some extension field.

A finer way to do the $p$-descent, as suggested in the original paper
\cite{couveignes96}, is to use the work of Voloch
\cite{voloch90}. Suppose $p\ne2$, let $E$ and $\widetilde{E}$ have
equations respectively
\begin{align*}
  y^2&=f(x)=x^3+a_2x^2+a_4x+a_6 \;\text{,}\\
  \tilde{y}^2&=\tilde{f}(\tilde{x}) = \tilde{x}^3 +
  \sqrt[p]{a_2}\tilde{x}^2 + \sqrt[p]{a_4}\tilde{x} + \sqrt[p]{a_6}
  \;\text{,}
\end{align*}
set
 \begin{equation}
  \label{eq:voloch:cover}
  \tilde{f}(X)^{\frac{p-1}{2}} = \alpha(X) + H_{\widetilde{E}}X^{p-1} + X^p\beta(X)
\end{equation}
with $\deg \alpha < p-1$ and $H_{\widetilde{E}}$ the Hasse invariant
of $\widetilde{E}$. Voloch shows the following proposition.

\begin{proposition}
  \label{th:voloch}
  Let $\tilde{c} = \sqrt[p-1]{H_{\widetilde{E}}}$, the cover of
  $\widetilde{E}$ defined by
  \begin{equation}
    \label{th:voloch:cover}
    C:\; \tilde{z}^p - \tilde{z} = \frac{\tilde{y}\beta(\tilde{x})}{\tilde{c}^p}
  \end{equation}
  is an étale cover of degree $p$ and is isomorphic to $E$ over
  $\F_q[\tilde{c}]$; the isomorphism is given by
  \begin{equation}
    \label{th:voloch:isom}
    \left\{
      \begin{aligned}
        (\tilde{x}, \tilde{y}) &= V(x, y)\\
        \tilde{z} &= -\frac{y}{\tilde{c}^p}\sum_{i=1}^{p-1}\frac{1}{x - x([i]P_1)}
      \end{aligned}
    \right.
  \end{equation}
  where $P_1$ is a primitive $p$-torsion point of $E$.
\end{proposition}

The descent is then performed as follows: starting from a point $P$ on
$E$, first pull it back along $\frobisog$, then take one of its
pre-images in $C$ by solving equation \eqref{th:voloch:cover}, finally
use equation \eqref{th:voloch:isom} to land on a point $P'$ in $E$.
The proposition guarantees that $[p]P' = P$. The descent is pictured
in figure \ref{fig:voloch}.

\begin{figure}
  \centering
  \[
  \xymatrix{\widetilde{E}\ar@/^/[r]^{\frobisog} & E\ar@/^/[l]^{V}}
  %%
  \qquad
  %%
  \xymatrix{
    \widetilde{E}\ar@/^/[r]^{\frobisog} & E\ar@/^/@{-->}[l]^{V}\ar[d]_{\simeq}\\
    & C\ar@/^/[ul]
  }
  \]
  
  \caption{Two ways of doing the $p$-descent: standard on the left and via a degree $p$ cover on the right}
  \label{fig:voloch}
\end{figure}


The reason why this is more efficient than a standard descent is the
shape of equation \eqref{th:voloch:cover}: it is an Artin-Schreier
equation and it can be solved by the techniques of
Chapter~\ref{cha:artin-schr-towers} (or by linear algebra, as was
suggested in \cite{couveignes96}). Once a solution $\tilde{z}$ to
\eqref{th:voloch:cover} is known, solving in $x$ and $y$ the bivariate
polynomial system \eqref{th:voloch:isom} takes just a GCD computation
(explicit formulas were given by Lercier in
\cite[$\S$6.2]{lercier-algorithmique}, we give some slightly improved
ones in Section \ref{sec:implementation}). Compare this with a generic
factoring algorithm needed by standard descent.

In this section we assume that the Artin-Schreier equations are solved
using linear algebra. The impact of
Chapter~\ref{cha:artin-schr-towers} over Couveignes' algorithm is
discussed in the next section.


\paragraph{Cauchy interpolation}
Interpolation reconstructs a polynomial from the values it takes on
some points; Cauchy interpolation reconstructs a rational fraction. As
we saw in Section~\ref{todo}, the Cauchy interpolation algorithm is
divided in two phases: first find the polynomial $P$ interpolating the
evaluation points, then use rational fraction reconstruction to find a
rational fraction congruent to $P$ modulo the polynomial vanishing on
the points.

Cauchy interpolation needs $n+2$ points to reconstruct a degree
$(k,n-k)$ rational fraction. This, together with \eqref{eq:velu-deg},
justifies the choice of $k$ such that $p^k > 4\ell - 2$. Some of our
variants of C2 will interpolate only on the primitive $p^k$-torsion
points, thus requiring the slightly larger bound $\euler(p^k) \ge
4\ell - 2$. This is not very important to our asymptotical analysis
since in both cases $p^k \in O(\ell)$.

\paragraph{Recognising the isogeny}
Once the rational fraction $\frac{g(X)}{h(X)}$ has been computed, one
has to verify that it is indeed an isogeny. The first test is to check
that the degrees of $g$ and $h$ match equation \eqref{eq:velu-deg}, if
they don't, the equation can be discarded right away and the algorithm
can go on with the next trial. Next, one can check that $h$ is indeed
the square of a polynomial (or, if $\ell$ is even, the product of one
factor of the $2$-division polynomial and a square polynomial). This
two tests are usually enough to detect an isogeny. In case a higher
confidence is needed, one can evaluate the rational fraction on some
random points of $E$ and check that it is indeed a group
morphism. Finally, if a deterministic proof is needed, one can compute
the $\ell$-division polynomial modulo $h$ and verify that it is equal
to $0$.


\subsection{The case $p=2$}
\label{sec:p=2}
The algorithm, as we presented it, only works when $p\ne2$, it is
however an easy matter to generalise it. The only phase that doesn't
work is the computation of the $p^k$-torsion points. For curves in the
simplified Weierstrass form \eqref{eq:weierstrass=2} the only
$2$-torsion point is $(0,\sqrt{b})$.

Voloch formulas are hard to adapt, nevertheless a $2$-descent on the
\hyperref[def:kummer]{Kummer surface} of $E$ can easily be performed
since the doubling formula reads
\begin{equation}
  x([2]P) = \frac{b}{x(P)^2} + x(P)^2 =
  \frobisog\left(\frac{\sqrt{b} + x(P)^2}{x(P)} \right) = \frobisog\circ V
  \;\text{.}
\end{equation}
Given point $x_P$ on $K_E$, a pull-back along $\frobisog$ gives a
point $\tilde{x}_P$ on $K_{\widetilde{E}}$. Then pulling $V$ back
amounts to solve
\begin{equation}
  \label{eq:2-descent}
  x^2 + \tilde{x}_Px = \sqrt{b}
\end{equation}
and this can be turned in an Artin-Schreier equation through the
change of variables $x \rightarrow x'\tilde{x}_P$.

From the descent on the Kummer surfaces one could deduce a full
$2$-descent on the curves by solving a quadratic equation at each step
in order recover the $y$ coordinate, but this would be too
expensive. Fortunately, the $y$ coordinates are not needed by the
subsequent steps of the algorithm, thus one may simply ignore
them. Observe in fact that even if $K_E$ does not have a group law,
the restriction of scalar multiplication is well defined and can be
computed through Montgomery formulas as described in
Remark~\ref{rk:montgomery}. This is enough to compute all the
abscissas of the points in $E[p^k]$ once a generator is known.


\subsection{Complexity analysis}
\label{sec:C2:complexity}
Analyzing the complexity of C2 is a delicate matter since the
algorithm relies on some black-box computer algebra algorithms in
order to deal with finite extensions of $\F_q$. The choice of the
actual algorithms may strongly influence the overall complexity of C2.
In this section we will only give some lower bounds on the complexity
of C2, since a much more accurate complexity analysis will be carried
out in Section \ref{sec:C2-AS}.

\paragraph{$p$-Torsion}
Applying Gunji formulas first requires to find $c$ and $c'$, $p-1$-th
roots of $H_E$ and $H_{E'}$, and build the field extension $\F_q[c] =
\F_q[c']$. Independently of the actual algorithm used, observe that in
the worst case $\F_q[c]$ is a degree $p-1$ extension of $\F_q$, thus
simply representing one of its elements requires $\Theta(pd)$ elements
of $\F_p$.

Subsequently, the main cost in Gunji's formulas is the computation of
the determinant of a $\frac{p-1}{2}\times\frac{p-1}{2}$
quadri-diagonal matrix (see \cite{gunji76}). This takes $\Theta(p^2)$
operations in $\F_q[c]$ by Gauss elimination, that is no less than
$\Omega(p^3d)$ operations in $\F_p$.

\paragraph{$p^k$-Torsion}
During the $p$-descent, factoring of equations \eqref{th:voloch:cover}
or \eqref{eq:2-descent} may introduce some field extensions over
$\F_q[c]$. Recall that an Artin-Schreier polynomial is either
irreducible or totally split (see
Proposition~\ref{th:artin-schreier}), so at each step of the
$p$-descent we either stay in the same field or we take a degree $p$
extension. This shows that in the worst case we have to take an
extension of degree $p^{k-1}$ over $F_q[c]$. The following
proposition, which is a generalization of
\cite[Proposition~26]{lercier-algorithmique}, states precisely how
likely this case is.

\begin{proposition}
  \label{th:tower}
  Let $E$ be an elliptic curve over $\F_q$, we denote by $\U_i$ the
  smallest field extension of $\F_q$ such that $E[p^i]\subset
  E(\U_i)$. For any $i\ge1$, either $[\U_{i+1}:\U_i] = p$ or
  $\U_{i+1}=\U_i=\cdots=\U_1$.
\end{proposition}

Before proving the proposition, we shall state a lemma. Its proof is
elementary and can be found in \cite[$\S6.1$]{lercier-algorithmique}.

\begin{lemma}
  \label{th:lercier-p-adic}
  Let $p$ be a prime and let $c$ be prime to $p$. For any $k>1$, let
  $\ord_k(c)$ be the order of $c$ in $\Z/p^k\Z$. Then
  $\ord_{k+1}(c)=\ord_k(c)$ implies $\ord_k(c)=\ord_{k-1}(c)$.
\end{lemma}

\begin{proof}[Proof of Propositon~\ref{th:tower}]
  Observe that the action of the Frobenius $\frobisog$ on $E[p]$ is
  just multiplication by the trace $t$, in fact the equation
  \[\frobisog^2 - [t \bmod p]\circ\frobisog + [q \bmod p] = 0\]
  has two solutions, namely $[t \bmod p]$ and $[0 \bmod p]$, but the
  second can be discarded since it would imply that $\frobisog$ has
  non-trivial kernel.  By lifting this solution, one sees that the
  action of $\frobisog$ on the Tate module $\mathcal{T}_p(E)$ is equal
  to multiplication by some $\tau\in\Z_p$.

  Let $G$ be the absolute Galois group of $\F_q$, there is a well
  known action of $G$ on $\mathcal{T}_p(E)$. Since $G$ is generated by
  the Frobenius automorphism of $\F_q$, the restriction of this action
  to $E[p^k]$ is equal to the action (via multiplication) of the
  subgroup of $(\Z/p^k\Z)^\ast$ generated by $\tau_k = \tau \bmod
  p^k$. Hence $[\U_k:\F_q] = \ord(\tau_k)$.

  Then, for any $k>1$, Lemma~\ref{th:lercier-p-adic} applied to
  $\tau_{k+1}=\tau\bmod p^{k+1}$ shows that
  $\ord(\tau_{k+1})=\ord(\tau_k)$ implies
  $\ord(\tau_k)=\ord(\tau_{k-1})$ and this concludes the proof.
\end{proof}


Thus for any elliptic curve there is an $i_0$ such that $[\U_i:\U_1] =
p^{i-i_0}$ for any $i \ge i_0$. This shows that the worst and the
average case coincide since for any fixed curve $[\U_k:\U_1] \in
\Theta(p^k)$ asymptotically. In this situation, one needs
$\Theta(p^kd)$ elements of $\F_p$ to store an element of $\U_k$.

Now the last iteration of the $p$-descent needs to solve an
Artin-Schreier equation in $\U_k$. To do this C2 precomputes the
matrix of the $\F_q$-linear application $(x^q-x):\U_k\rightarrow\U_k$
and its inverse, plus the matrix of the $\F_p$-linear application
$(x^p-x):\F_q\rightarrow\F_q$ and its inverse. The former is the most
expensive one and takes $\Theta(p^{\omega k})$ operations in $\F_q$,
that is $\Omega(p^{\omega k}d) = \Omega(\ell^\omega d)$ operations in
$\F_p$, plus a storage of $\Theta(\ell^2d)$ elements of
$\F_p$. Observe that this precomputation may be used to compute any
other isogeny with domain $E$.

After the precomputation has been done, C2 successively applies the
two inverse matrices; details can be found in
\cite[$\S$2.4]{couveignes96}. This costs at least $\Omega(\ell^2d)$.


\paragraph{Interpolation}
The most expensive part of Cauchy interpolation is the polynomial
interpolation phase. In fact, simply representing a polynomial of
degree $p^k-1$ in $\U_k[X]$ takes $\Theta(p^{2k}d)$ elements of
$\F_p$, thus at least $\Omega(\ell^2d)$ operations are needed to
interpolate unless special care is taken\footnote{This contribution
  due to arithmetics in $\U_k$ had been underestimated in the
  complexity analysis of \cite{couveignes96}, where an estimate of
  $\Omega(\ell d\log\ell)$ operations was given.}. We will give more
details on interpolation in Section \ref{sec:C2-AS-FI}.


\paragraph{Recognising the isogeny}
The cost of testing for squareness of the denominator and of the other
probabilistic tests is negligible compared to the rest of the
algorithm. The cost of computing the $\ell$-division polynomial modulo
$h$ is $O(\Mult(\ell)\log\ell)$ operations in $\F_q$, thus, again,
negligible.

Nevertheless it is important to realize that, on average, half of the
$\euler(p^k)$ mappings from $E[p^k]$ to $E'[p^k]$ must be tried before
finding the isogeny, for only one of these mappings corresponds to
it. This implies that the Cauchy interpolation step must be repeated
an average of $\Theta(p^k)$ times, thus contributing a
$\Omega(\ell^3d)$ to the total complexity.

Summing up all the contributions one ends up with the following lower
bound
\begin{equation}
  \label{eq:C2:complexity}
  \Omega(\ell^3d + p^3d)
\end{equation}
plus a precomputation step whose cost is negligible compared to this
one and a space requirement of $\Theta(\ell^2d)$ elements. In the next
sections we will see how to make all these costs drop.


\subsection{The case $(p,\ell)\ne1$}
\label{sec:C2:non-prime}
If we are interested in finding a separable isogeny whose degree is
not prime to $p$, the best way is to compute the curve $\widetilde{E}$
such that $E = \widetilde{E}^{(p)}$, then compute an isogeny of degree
$\ell/p$ between $\widetilde{E}$ and $E'$ and finally compose it with
the separable $p$-isogeny $V$ from $E$ to $\widetilde{E}$.

Observe however that C2 can be easily adapted to directly compute such
an isogeny. In fact let $v=v_p(\ell)$, then $\I(E[p^k]) =
E'[p^{k-v}]$. All one needs to do in this case is to modify the Cauchy
interpolation so that it interpolates the rational function that sends
a generator of $E[p^k]$ over a generator of $E'[p^{k-v}]$ and the
other points accordingly. The maximum number of trials to do before
finding the isogeny is $\euler(p^{k-v})$, thus the overall complexity
is
\begin{equation}
  \label{eq:C2:complexity-non-prime}
  \Omega\left(\frac{\ell^3}{p^v}d + p^3d\right)
  \;\text{.}
\end{equation}

Although this method is less efficient than the first one, it will
come handy in Section \ref{sec:bounded}.



% Local Variables:
% mode:flyspell
% ispell-local-dictionary:"american"
% mode:TeX-PDF
% mode:reftex
% TeX-master: "../these"
% End:
%
% LocalWords:  Schreier Artin pseudotrace Frobenius bivariate Joux Sirvent FFT
% LocalWords:  Couveignes isogenies Schoof isogeny cryptosystems Lercier
% LocalWords:  precomputation arithmetics polylogarithmic Karatsuba precomputes
% LocalWords:  endomorphisms asymptotical
