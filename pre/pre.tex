\chapter*{Introduction}
\addcontentsline{toc}{chapter}{Introduction (English)}



% In this thesis we apply techniques from computer algebra and language
% theory to speed up the elementary operations in some specific towers
% of finite fields. We apply our construction to the problem of
% computing isogenies between elliptic curves and obtain faster (both
% asymptotically and in practice) variants of Couveignes' algorithm.

% After some preliminaries, we introduce in Part II one of our main
% tools: transposition of linear programs. Previously, Bostan, Schost
% and Lecerf had shown that transposition of programs is possible,
% without losses in time and space complexity, in a slightly generalized
% straight-line-program model. We generalize their construction and
% devise a fully featured functional language together with an algorithm
% to transpose programs written in it. Our new transposition is as
% efficient as theirs on programs that fit into their model, but it also
% permits to treat more generic programs with a minor loss in space
% complexity. We also describe the implementation of a compiler
% materializing our ideas.

% In Part III, we combine our transposition method with well known
% techniques coming from computer algebra and elimination theory, then
% we show some new results on towers of Artin-Schreier extensions over
% finite fields. This allows us to design a complete set of algorithms
% to perform the most common arithmetic operations on Artin-Schreier
% towers in quasi-optimal time. We implemented all our algorithms in a
% software package that we describe at the end of this part.

% Finally, Part IV discusses the problem of isogeny computation. We
% apply our techniques to a fast variant of Couveignes' algorithm that
% we had obtained in our Masters thesis, and make comparisons to other
% algorithms to compute isogenies.



% Dans cette thèse nous appliquons des techniques provenant du calcul
% formel et de la théorie des langages afin d'améliorer les operations
% élémentaires dans certaines tours de corps finis. Nous appliquons
% notre constructions au problème du calcul d'isogénies entre courbes
% elliptiques et obtenons une variante plus rapide (à la fois en théorie
% et en pratique) de l'algorithme de Couveignes.

% Après avoir introduit les notions de base, nous présentons dans la
% Partie II un de nos outils principaux: la transposition de programmes
% linéaires. Précédemment, Bostan, Schost et Lecerf avaient prouvé qu'il
% est possible transposer des programmes, sans pertes en complexité
% temps ou espace, dans un modèle qui est une modeste généralisation des
% programmes sans branchements. Nous généralisons leur construction en
% concevant un langage fonctionnel standard pour lequel nous donnons un
% algorithme de transposition. Notre nouvelle transposition est aussi
% efficace que la leur sur des programmes qui font partie de leur
% modèle, mais elle permet aussi de traiter des programmes plus généraux
% avec des pertes mineures en complexité espace. Nous décrivons
% l'implantation d'un compilateur réalisant nos idées.

% Dans la Partie III, nous combinons notre transposition avec des
% techniques classiques de calcul formel et théorie de l'élimination,
% puis nous montrons des nouveaux résultats sur les tours
% d'Artin-Schreier sur des corps finis. Ceci nous permet de concevoir
% une gamme d'algorithmes qui calculent en temps quasi-optimal les
% opérations arithmetiques les plus communes sur les tours
% d'Artin-Schreier. Nous avons implanté ces algorithmes dans une
% bibliothèque logicielle que nous décrivons à la fin de cette partie.

% Enfin, la Partie IV traite le problème du calcul d'isogénies. Nous
% appliques nos techniques à une variante rapide de l'algorithme de
% Couveignes que nous avions proposée dans notre mémoire de Master, puis
% nous comparons les résultats avec d'autres algorithmes pour le calcul
% d'isogénies.



It would not be far from the truth saying that all our electronic
devices spend their time doing: 1. nothing, 2. integer arithmetics, 3.
finite field arithmetics. Your favorite CD player certainly does and
your toaster maybe will one day. This is so true, that the last
generation of Intel Core processors supports a hardware instruction
(CLMUL) for multiplication in $\F_{2^m}$~\cite{intel-carryless}.

This is because finite fields appear everywhere in telecommunications
engineering, in particular in Error Correcting Codes and
Cryptography. This thesis applies advanced
algorithmic % and algebraic ?
techniques to computations in towers of extensions of finite fields,
in view of applications to elliptic curve cryptography.



\paragraph*{Isogeny computation}
In elliptic curve cryptography, in order to build a secure
cryptosystem, one must select a curve whose number of points contains
a large enough prime factor. The preferred method for doing this is to
randomly select a curve and then use a point-counting algorithm to
determine its cardinality. The first polynomial time point counting
algorithm for elliptic curves was due to Schoof~\cite{schoof85}, then
improved by Atkin and
Elkies~\cite{atkin88,elkies92,elkies98,schoof95}, henceforth named
SEA.

The SEA algorithm raised interest in explicit computations with
isogenies, i.e.\ algebraic group morphisms of elliptic curves. When
computing isogenies over finite fields one must distinguish between
the large and arbitrary characteristic. In the first case, one can use
algorithms that work for characteristic $0$, and then reduce the
result; the methods of Elkies~\cite{elkies98,morain95},
Atkin~\cite{schoof95} and Bostan, Morain, Salvy and
Schost~\cite{bostan+morain+salvy+schost08} belong to this family. When
the reduction modulo the characteristic introduces division by $0$,
these algorithms are not of help.

The first two algorithms to compute isogenies in arbitrary
characteristic are due to Couveignes~\cite{couveignes94,couveignes96}:
both have a polynomial dependency in the characteristic, which makes
them unpractical for values higher than $2$ or $3$. An algorithm
specific to characteristic $2$ was given by Lercier~\cite{lercier96};
in practice it performs faster than Couveignes' algorithms, but its
complexity is not well understood. More recently, Joux and
Lercier~\cite{joux+lercier06} and Lercier and
Sirvent~\cite{lercier+sirvent08} have shown that it is possible to
avoid division by $0$ by lifting the curves in the $p$-adics. The last
one is currently the algorithm for the arbitrary characteristic case
having the best asymptotic complexity; 
%% Or rather: the last algorithm is currently the one having the best
%% asymptoyic complexity for the arbitrary characteristic case
its complexity in the
characteristic is only logarithmic.

It is interesting to remark, however, that no algorithm to compute
isogenies has optimal or quasi-optimal complexity, with the only
exception of~\cite{bostan+morain+salvy+schost08} on a very special
case. 

The starting point of this work was Couveignes' second
algorithm~\cite{couveignes96}. It computes an isogeny by interpolating
it over the $p^k$-torsion points of the elliptic curves for a large
enough $k$; when those points are not defined on the base field, one
has to take towers of field extensions to find them. The field
extensions that naturally arise when doing this computation are
splitting fields of polynomials of the form
\[X^p - X -\alpha\text{;}\] such extension are called Artin-Schreier
extensions. 


\paragraph*{Towers of finite fields}
Besides addition, multiplication and inversion, the arithmetic
operations of interest in a tower of finite extensions arguably are
relative traces %, minimal polynomials
and embeddings. For finite fields one could add explicit Galois groups
to the list as these are relatively easy to compute with.

The arithmetic of towers of finite fields is a central question for
any computer algebra system, however it has received few attention, if
any. Magma is known for having had support for lattices of finite
fields for a long time~\cite{bosma+cannon+steel97}, but it is hard to
tell which algorithms it implements nowadays and what their
complexities are. All other results that can possibly apply to towers
of finite fields were derived in the more general context of
polynomial system solving and effective algebraic geometry, in
particular in the resolution of triangular
sets~\cite{diaz+gonzalez01,giusti+lecerf+salvy01,bostan+salvy+schost03,pascal+schost06,li+moreno+schost07,dahan+jin+moreno+schost08,boulier+lemaire+moreno01,FGLM,rouiller99,alonso+becker+roy+wormann}.

In the specific case of Artin-Schreier towers, the literature is not
extensive either.  Using ideas from~\cite{Conway:ONAG2000},
Cantor~\cite{cantor89} constructs a particular Artin-Schreier tower
% that he applies to FFT multiplication in $\F_2[X]$.
suited for FFT multiplication in characteristic
$2$. In~\cite{couveignes00}, Couveignes gives an algorithm to compute
isomorphisms between Artin-Schreier towers; however, his algorithm
needs as a prerequisite a fast multiplication algorithm in
% a tower of the same shape as in~\cite{cantor89}
some towers of a special kind, called ``Cantor towers''
in~\cite{couveignes00}. Such an algorithm is unfortunately not in the
literature, making the results of~\cite{couveignes00} non practical.


\paragraph*{Transposition principle}
One algorithmic tool that we shall study in depth and apply throughout
the whole document is the \emph{transposition principle}, which is the
language-theoretic counterpart to algebraic duality.

The transposition principle was discovered in electrical network
theory by Bordewijk~\cite{bordewijk57}, then proved in its general form
by Fiduccia~\cite{fiduccia:phd}; but it only became popular in
computer algebra much later through the works of Kaltofen, Yagati,
Shoup, von zur Gathen and
others~\cite{kaltofen+lakshman89,vzgathen+shoup92,shoup94,shoup95,shoup99,hanrot+quercia+zimmermann}. It
states that
\begin{quote}
Any \emph{$R$-algebraic} algorithm
  computing a linear function $f:M\ra N$, where $M,N$ are $R$-modules,
  can be transformed in an $R$-algebraic algorithm to compute
  $\dual{f}$. The two algorithms have the same space and time
  algebraic complexities.
\end{quote}

The transposition principle is important in computer algebra because
it permits to derive asymptotically good algorithms that were not
otherwise evident. One big step forward in the understanding of it was
done by Bostan, Lecerf and Schost~\cite{bostan+lecerf+schost:tellegen}
who, extending work of Shoup~\cite{shoup95}, remarked that
transposition can be systematically applied to a restricted
programming language. It is also remarkable that the transposition
principle has a strong connection with automatic
differentiation~\cite{baur+strassen83,kaltofen+lakshman89,Ka2K,gashkov+gashkov05,sergeev08}.

In this document we investigate more in depth the relationships
between the transposition principle and programming languages. We use
the theory of typed purely functional languages~\cite{pierce} as
framework, because its elegant mathematical structure permits to
reason at an algebraic level on programs.


\paragraph*{Outline, contributions and software packages}
This document is divided in four parts. Part~\ref{part:prerequisites}
recalls the basic notions from algebra and computer algebra that we
will use later.

Part~\ref{part:transp-princ} studies the transposition principle. In
Chapter~\ref{cha:algebr-compl-dual} we present the arithmetic circuit
model and the straight line program model, and prove the
% transposition theorem
principle in them. Then we discuss the relationships with automatic
differentiation. Chapter~\ref{cha:autom-transp-code} is a
collaboration with Schost. We study the relationships between the
arithmetic circuit model and functional programming languages, then we
show that transposition can be applied algorithmically to a generic
functional language. Some complementary observations on the
categorical semantics of arithmetic circuits and their application to
automatic transposition are given in
Appendix~\ref{cha:basic-categ-theory}, which is joint work with
Boespflug.

Part~\ref{part:fast-arithm-using} is devoted to arithmetics in towers
of extensions. We start by reviewing the general theory of
zero-dimensional ideals and rational univariate representations in
Chapter~\ref{cha:trace-computations}. Here, the results of
Part~\ref{part:transp-princ} are the key to obtain asymptotically fast
algorithms. The algorithms of this chapter are then applied in
Chapter~\ref{cha:artin-schr-towers}, where we provide asymptotically
good algorithms for Artin-Schreier towers (fruit of another
collaboration with Schost).

Finally Part~\ref{part:appl-isog-comp} applies the results of the
previous chapters to isogeny computation. After some general
references on elliptic curves in Chapter~\ref{cha:ellipt-curv-isog},
we review in Chapter~\ref{cha:algor-small-char} the most efficient
algorithms to compute isogenies over finite fields, and present new
asymptotically fast variants of Couveignes' algorithm, along with a
promising generalization in
Section~\ref{sec:bounded}. Chapter~\ref{cha:experimental-results} then
discusses implementation and experimental results for this part of the
document.

Theory would be meaningless without practice. Similarly, this
manuscript would make no sense if it was not accompanied by our
software packages. The great majority of the algorithms we present
here have been implemented, packaged and distributed under open source
licences. So, all the algorithms of
Chapter~\ref{cha:artin-schr-towers} can be found in the \texttt{C++}
library \texttt{FAAST}, available from
\url{http://www.lix.polytechnique.fr/Labo/Luca.De-Feo/FAAST/}. At the
moment we write, the compiler for the language \texttt{transalpyne} of
Chapter~\ref{cha:autom-transp-code} is not distributed yet; we are
currently working on the first stable release and hope to start
distributing it by the beginning of 2011. It will be available from
\url{ http://transalpyne.gforge.inria.fr/}.




%%% Local Variables: 
%%% mode:flyspell
%%% ispell-local-dictionary:"american"
%%% mode: TeX-PDF
%%% mode: reftex
%%% TeX-master: "../these"
%%% End: 


\part{Prerequisites}

\chapter{Algebra}
\label{cha:algebra}
Here we recall the basic concepts from abstract algebra that
constitute the background for all the chapters that follow. To the
reader interested in reading more about these topics, we recommend
\cite{lang} for general algebra, \cite{lidl+niederreiter:2} for finite
fields and \cite{silverman:elliptic,silverman:advanced} for elliptic
curves.

\section{Groups, Rings, Fields}
\label{sec:ring-fields}

\subsection{Objects}
\label{sec:ring-fields:objects}

\subsubsection{Groups}

A \index{group}\textbf{group} is a pair $(G,\cdot)$ such that $G$ is a
set and $\cdot:G\times G\ra G$ is an \emph{internal composition law}
satisfying:
\begin{itemize}
\item \index{associativity}\textbf{Associativity}: $(a\cdot b)\cdot c
  = a \cdot (b\cdot c)$ for any $a,b\in G$;
\item There is an element $e\in G$, called the
  \index{identity~element}\textbf{identity}, such that $a\cdot e =
  e\cdot a = a$ for any $a\in G$;
\item For any $a\in G$ there is an
  \index{inverse~element}\textbf{inverse element} $a^{-1}$ such that
  $a\cdot a^{-1} = a^{-1}\cdot a = e$.
\end{itemize}
If $\cdot$ also satisfies $a\cdot b=b\cdot a$
(\index{commutativity}\textbf{commutativity}), the group is said to be
\index{group!abelian}\textbf{abelian}. The group composed of one
single element with the obvious law is called the
\index{group!trivial}\textbf{trivial group}.

A \index{subgroup}\textbf{subgroup} of a group $(G,\cdot)$ is a group
$(H,\circ)$ such that $H\subset G$ and $\circ$ is the restriction of
$\cdot$ to $H$. Any group has two trivial subgroups: the trivial group
and itself. The \index{opposite!group}
\index{group!opposite~group}opposite group of a group $(G,\cdot)$ is
the group $(G,\cdot^\op)$ where $a\cdot^\op b=b\cdot a$ for any
$a,b\in G$.

The \index{center!of~a~group} \index{group!center}\textbf{center} of
$(G,\cdot)$ is the subgroup formed by the elements $a\in G$ such that
$a\cdot b=b\cdot a$ for any $a\in G$. The center is clearly a
commutative group.

Let $A$ be a subset of $G$, the group
\index{group!generator}\index{generator!of~a~group}\textbf{generated}
by $A$, denoted by $\langle A\rangle$, is the smallest subgroup of
$(G,\cdot)$ containing $A$. A finite group generated by a single
element is said to be
\index{cyclic!group}\index{group!cyclic}\textbf{cyclic}.

Let $(G,\cdot)$ be a group, $(H,\cdot)$ a subgroup and $g\in G$. The
subset $g\cdot H = \{g\cdot h | h \in H \}$ of $G$ is called a
\index{coset!left~coset}\textbf{left coset} of $H$, or simply
\index{coset}coset. A \index{coset!right~coset}\textbf{right coset},
denoted by $H\cdot g$, is a left coset for the opposite group; when
$G$ is abelian the two notions coincide. A subgroup $H$ is called
\index{subgroup!normal}normal if $g\cdot H=H\cdot g$ for any $g\in G$;
note that if $G$ is abelian, any subgroup is normal. Let $H$ be
normal, the cosets of $H$ form a group under the law $(g\cdot
H,g'\cdot H)\mapsto (g\cdot g')\cdot H$, this group is called the
\index{quotient!of~groups}\textbf{quotient} of $G$ by $H$ and is
denoted by $G/H$.

Let $S$ be a set and let $G$ be a group. A
\index{group~action!left~group~action}
\index{group~action}\textbf{(left) group action} of $G$ over $S$ is a
law $.:G\times S\ra S$ such that for any $g,g'\in G$ and $x\in S$,
\begin{itemize}
\item $g'.(g.x) = (g'\cdot g). x$,
\item $e. x = x$.
\end{itemize}
A \index{group~action!right~group~action}\textbf{right group action}
is a left group action for the opposite group.


\subsubsection{Rings, Fields}

A \index{ring}\textbf{ring} is a tuple $(R,+,\cdot)$ such that $(R,+)$
is an abelian group and $\cdot:R\times R\ra R$ is an internal
composition law satisfying associativity, existence of the identity
and \index{distributivity}\textbf{distributivity} over $+$
\[a \cdot (b + c) = (a\cdot b) + (a\cdot c) \quad\text{for any
  $a,b,c\in R$.}\] When $\cdot$ satisfies commutativity, the ring is
said to be \index{ring!commutative}\textbf{commutative}.  The law $+$
is called \emph{addition}, $\cdot$ is called \emph{multiplication},
the identity for $+$ is denoted by $0$ and the identity for $\cdot$ by
$1$.  A commutative ring such that $1\ne 0$ and where $\cdot$ also
satisfies the existence of the inverse, is called a
\index{field}\textbf{field}.

A \index{subring}\textbf{subring} of a ring $(R,+,\cdot)$ is a ring
$(S,\ast,\circ)$ such that $(S,\ast)$ is a subgroup of $(R,+)$ and
$\circ$ is the restriction of $\cdot$ to $S$.  The
\index{opposite!ring}\index{ring!opposite~ring}\textbf{opposite ring}
of a ring $(R,+,\cdot)$ is the ring $(R,+,\cdot^\op)$
where $a\cdot^\op b=b\cdot a$ for any $a,b\in R$.

The simplest example of ring is $\Z$, the set of integers; the
rational numbers $\Q$ are an example of field, it is the
\emph{smallest} field containing $\Z$ as a subring. The
\index{ring!trivial}\textbf{trivial ring} is the ring composed of one
unique element $r=0=1$ with the evident laws; note that by definition
this is not a field.

\subsubsection{Modules, ideals, vector spaces}

Given a ring $(R,+,\cdot)$ a \index{module!left~module}\textbf{left
  module}, or simply \textbf{module}, over $R$ is a tuple $(M, +_M,
\cdot_M)$ such that $(M,+_M)$ is an abelian group and $\cdot_M:R\times
M\ra M$ is an \emph{external law} such that for any $r,r'\in R$ and
$m,m'\in R$
\begin{itemize}
\item $(r + r')\cdot_M m = (r \cdot_M m) +_M (r'\cdot_M m)$,
\item $r\cdot_M(m +_M m') = (r\cdot_M m) +_M (r'\cdot_M m)$,
\item $r'\cdot_M(r\cdot_M m ) = (r'r)\cdot_M m$,
\item $1\cdot_M m = m$.
\end{itemize}
The law $+_M$ is called \emph{addition}, its identity is denoted by
$0$; the law $\cdot_M$ is called
\index{multiplication!scalar}\index{scalar~multiplication}\textbf{scalar}
or
\index{mutliplication!external}\index{external~multiplication}\textbf{external}
multiplication.

A \index{module!right~module}\textbf{right module} is a left module
for the opposite ring $R^\op$, a
\index{module!two-sided}\textbf{two-sided module} also called
\index{bimodule}\textbf{bimodule} is an object that is both a left and
a right module. When $R$ is commutative, the three notions coincide
and we simply speak of a \index{module}\textbf{module}.  $R$-module is
another way of saying ``module over $R$''. When $\K$ is a field, a
$\K$-module is called a \index{vector~space}\textbf{$\K$-vector
  space}.

A \index{module!submodule}\index{submodule!left~submodule}left
(\index{submoudle!right~submodule}right,
\index{submodule!two-sided}two-sided) \textbf{submodule} of a left
(right, two-sided) $R$-module $(M,+,\cdot)$ is a left (right, two-sided)
$R$-module $(N,\ast,\circ)$ such that $(N,\ast)$ is a subgroup of
$(M,+)$ and $\circ$ is the restriction of $\cdot$ to $R\times N$.

The module containing one unique element with the evident laws is
called the \index{module!zero~module}\textbf{zero module}; any
$R$-module contains a submodule that is isomorphic to the zero
module. Any group $(G,+)$ can be given a $\Z$-module structure by the
law
\[n\cdot g = \underbrace{g + \cdots + g}_{n\text{ times}} \text{.}\]
Any ring $R$ is trivially a two-sided module over itself; a
\index{ideal!left~ideal} (\index{ideal!right~ideal}right,
\index{ideal!two-sided}two-sided) \textbf{ideal} of a ring $R$ is a
submodule of the left (right, two-sided) $R$-module $R$.  When $R$ is
commutative one simply speaks of an \index{ideal}ideal.  

Any ring contains at least two submodules: the zero module and itself;
these are called the \index{ideal!trivial}\textbf{trivial ideals}. The
only non-trivial ideals of $\Z$ are the $n\Z$ for any $n\ne0,1$. A
field has no non-trivial ideals.

The \index{direct~sum}\textbf{direct sum} $M\oplus N$ of two
$R$-modules $(M,+_M,\cdot_M)$ and $(N,+_N,\cdot_N)$ is the module
$(M\times N,+,\cdot)$ where the laws $+$ and $\cdot$ are defined
component-wise. This generalizes to sums of an arbitrary number of
modules: let $(M_i)_{i\in I}$ be a sequence of $R$-modules, the direct
sum $\bigoplus_{i\in I}M_i$ is the $R$-module whose elements are the
sequences $(m_1,m_2,\ldots)$ where $m_i\in M_i$ and $m_i=0$ for all
but a finite number of them; the laws are defined component-wise.
Although less commonly used, there also exists a notion of
\index{direct~product}\textbf{direct product}: given $(M_i)_{i\in I}$,
the direct product $\prod_{i\in I}M_i$ is the $R$-module whose
elements are the sequences $(m_1,m_2,\ldots)$ where $m_i\in M_i$ with
the laws defined component-wise. Clearly, the two definition coincide
when $I$ is a finite set.

When $R$ is seen as an $R$-module over itself, we denote by $R^n$ the
direct sum $\bigoplus_{0<i\le n}R$ and by $R^\infty$ the direct sum
$\bigoplus_{i>0}R$. An $R$-module that is isomorphic to the direct sum
$\bigoplus_{I}R$ for some $I$ is called a
\index{module!free}\textbf{free module}. A \index{basis}\textbf{basis}
of a module $M$ is a family $(m_i)_{i\in I}$ of elements of $M$ such
that any $m\in M$ can be written as 
\begin{equation}
  \label{eq:module-basis}
  m = \sum_{i\in I} r_i\circ m_i
  \quad\text{with $r_i\in R$}
\end{equation}
in an unique way. Clearly, if we note by $e_i$ the element of
$\bigoplus_IR$ that has $1$ in the $i$-th position and $0$ elsewhere,
the family $(e_i)_{i\in I}$ forms a basis; hence, any free module has
a basis and, conversely, any module that has a basis is free. One
important statement about bases of modules is the following.

\begin{proposition}
  Any two bases for a free module $M$ over a commutative ring $R$ have
  the same cardinality.
\end{proposition}

For this reason, when $M$ is a free module over a commutative ring $R$
we call \index{dimension} \index{module!free!dimension~of}
\index{vector~space!dimension~of}\textbf{dimension} the cardinality of
any of its bases. It is a well known result in linear algebra that any
vector space has a basis, hence any $\K$-vector space is free as a
$\K$-module.

Given a module $M$ and a submodule $N$, the quotient group $M/N$ can
be given a module structure by the law $r\cdot(m+N)=(r\cdot m)+N$, it
is then called the \index{quotient!of~modules}\textbf{quotient
  module}. When $R$ is a ring and $I$ a module, the quotient $R/I$ can
also be given a ring structure by the law $(r+I)\cdot(r'+I)=(r\cdot
r')+I$, it is then called the \index{quotient!of~ring}\textbf{quotient
  ring}.


\subsection{Arrows}
\label{sec:ring-fields:arrows}

Morphisms, kernels, isomorphism theorems, 

\section{Linear algebra}
\label{sec:linear-algebra}

duality
matrices, trace, determinant, resultant

\section{Basic Galois theory}
\label{sec:basic-galois-theory}
Field extension, splitting field, Galois extensions, algebraic closure

roots of unity, cyclotomic polynomials

trace, norm

Artin-Schreier

finite fields


\section{Elliptic curves}
\label{sec:elliptic-curves}



%%% Local Variables: 
%%% mode:flyspell
%%% ispell-local-dictionary:"american"
%%% mode: TeX-PDF
%%% mode: reftex
%%% TeX-master: "../these"
%%% End: 




\chapter{Basic category theory}
\label{cha:basic-categ-theory}
\section{Basic concepts}
Objects, arrows, functors, natural transformations, adjoints

\section{Additive categories}
Ab-categories, additive categories

K-Vect, R-Mod

\section{Cartesian closed categories}
CCC

Applications to lambda calculus




%%% Local Variables: 
%%% mode:flyspell
%%% ispell-local-dictionary:"american"
%%% mode: TeX-PDF
%%% mode: reftex
%%% TeX-master: "../these"
%%% End: 


\chapter{Logics for computer science}
\label{cha:logics-comp-science}
\section{$\lambda$-calculus}
\label{sec:lambda-calculus}
\section{Simply typed $\lambda$-calculus}
\label{sec:simply-typed-lambda}
\section{System F}
\label{sec:system-f}
\section{Purely functional languages and polymorphic type checking}
\label{sec:purely-funct-lang}


\chapter{Algorithms and complexity}
\label{cha:algor-compl}
\section{Asymptotic complexity}
Many algorithms below rely on fast multiplication; thus, we let $\Mult
: \N \rightarrow \N$ be a {\em multiplication function}, such that
polynomials in $\F_p[X]$ of degree less than $n$ can be multiplied in
$\Mult(n)$ operations, under the conditions of~\cite[Ch.~8.3]{vzGG}.
Typical orders of magnitude for $\Mult(n)$ are $O(n^{\log_2(3)})$ for
Karatsuba multiplication or $O(n\log (n) \log\log (n))$ for FFT
multiplication. Using fast multiplication, fast algorithms are
available for Euclidean division or extended GCD~\cite[Ch.~9 \&
11]{vzGG}.

The cost of {\em modular composition}, that is, of computing $F(G)
\bmod H$, for $F,G,H\in\F_p[X]$ of degrees at most $n$, will be
written $\ModComp(n)$. We refer to~\cite[Ch.~12]{vzGG} for a
presentation of known results in an algebraic computational model: the
best known algorithms have subquadratic (but superlinear) cost in
$n$. Note that in a boolean RAM model, the algorithm of~\cite{KeUm08}
takes quasi-linear time.

\section{Fundamental algorithms}
Polynomial multiplication, Newton inversion, matrix multiplication,
modular composition, rational fraction reconstruction\ldots




%%% Local Variables: 
%%% mode:flyspell
%%% ispell-local-dictionary:"american"
%%% mode: TeX-PDF
%%% mode: reftex
%%% TeX-master: "../these"
%%% End: 
