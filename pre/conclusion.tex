\chapter*{Conclusion}
\addcontentsline{toc}{chapter}{Conclusion}

\pdfmctwo{Conclusion.}
We have presented our contributions to the study of efficient
algorithms for towers of finite fields and isogenies between elliptic
curves. In view of these applications, we have employed advanced
algebraic and algorithmic techniques, and developed new tools that
have an interest of their own.

Before this work, the transposition principle used to be regarded
merely as an existential result about algebraic algorithms.
Following~\cite{bostan+lecerf+schost:tellegen}, it could be applied in
an automatic fashion to a very special class of algebraic algorithms,
that we call \emph{algebraic transforms} in
Section~\ref{sec:stra-line-progr}. In this document we have shown
that, using partial evaluation and typed functional languages, it is
possible to automatically infer the linear algebraic structure of an
algebraic algorithm, and produce its transposition.  It would be
interesting to explore new ways of implementing the transposition
principle in higher order languages such as Haskell or Coq, as this
could have applications to the formal verification of computer algebra
systems by automated theorem provers.  We have sketched some relevant
ideas in Appendix~\ref{cha:basic-categ-theory}.

With the help of transposed algorithms, we have constructed a family
of Artin-Schreier towers of finite fields with quasi-optimal
arithmetic operations. Thanks to Couveignes'
algorithm~\cite{couveignes00}, such fast arithmetics generalize to any
Artin-Schreier tower. Since any separable extension of degree equal to
the characteristic is Artin-Schreier, our construction provides --at
least in theory-- fast arithmetics for all these tower of extensions.
This can be applied, for example, to the computation of torsion points
of Abelian varieties, as we did in this document. It would be
interesting to generalize this construction to the case of function
fields, as this could have applications to coding
theory~\cite{garcia+stichtenoth96,shum-et-al01}.

Using our construction for Artin-Schreier towers, we were able to
implement and improve Couveignes' second algorithm for isogeny
computation~\cite{couveignes96}. The comparison of our implementation
with Lercier and Sirvent's algorithm~\cite{lercier+sirvent08}
concludes in favor of the latter, however our improvements to
Couveignes' algorithm stay of theoretical interest for several
reasons. First, Couveignes' algorithm can be easily generalized to
Jacobians of hyperelliptic curves, although with a much worse
complexity. Improving such generalization, at least for the case of
genus $2$ hyperelliptic curves, would be of some relevance for point
counting~\cite{schoof95,pila90,gaudry+schost04}, although $p$-adic
methods are likely to remain the best algorithms for the small
characteristic case~\cite{kedlaya01,denef+vercauteren06}. Second, our
generalization of Couveignes' algorithm to compute isogenies of
unknown degree sheds new light on Couveignes' algorithm and on the
complexity of the isogeny computation problem; and could have
applications in cryptography~\cite{teske06,rostovtsev+stolbunov06}.
Looking for similar generalizations of other algorithms, such as
Couveignes' first algorithm~\cite{couveignes94}, is a first step
towards a better understanding of the problem, and could ultimately
lead to an optimal algorithm to compute isogenies of given degree
between elliptic curves, a result that is still out of reach today.


%%% Local Variables: 
%%% mode:flyspell
%%% ispell-local-dictionary:"american"
%%% mode:reftex
%%% TeX-master: "../these"
%%% mode: PDFLaTex
%%% End: 
