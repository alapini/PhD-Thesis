\subsection*{Abstract}
In this thesis we apply techniques from computer algebra and language
theory to speed up the elementary operations in some specific towers
of finite fields. We apply our construction to the problem of
computing isogenies between elliptic curves and obtain faster (both
asymptotically and in practice) variants of Couveignes' algorithm.

After some preliminaries, we introduce in Part II one of our main
tools: transposition of linear programs. Previously, Bostan, Schost
and Lecerf had shown that transposition of programs is possible,
without losses in time and space complexity, in a slightly generalized
straight-line-program model. We generalize their construction and
devise a fully featured functional language together with an algorithm
to transpose programs written in it. Our new transposition is as
efficient as theirs on programs that fit into their model, but it also
permits to treat more generic programs with a minor loss in space
complexity. We also describe the implementation of a compiler
materializing our ideas.

In Part III, we combine our transposition method with well known
techniques coming from computer algebra and elimination theory, then
we show some new results on towers of Artin-Schreier extensions over
finite fields. This allows us to design a complete set of algorithms
to perform the most common arithmetic operations on Artin-Schreier
towers in quasi-optimal time. We implemented all our algorithms in a
software package that we describe at the end of this part.

Finally, Part IV discusses the problem of isogeny computation. We
apply our techniques to a fast variant of Couveignes' algorithm that
we had obtained in our Masters thesis, and make comparisons to other
algorithms to compute isogenies.



\selectlanguage{french}
\subsection*{Résumé}

Dans cette thèse nous appliquons des techniques provenant du calcul
formel et de la théorie des langages afin d'améliorer les operations
élémentaires dans certaines tours de corps finis. Nous appliquons
notre constructions au problème du calcul d'isogénies entre courbes
elliptiques et obtenons une variante plus rapide (à la fois en théorie
et en pratique) de l'algorithme de Couveignes.

Après avoir introduit les notions de base, nous présentons dans la
Partie II un de nos outils principaux: la transposition de programmes
linéaires. Précédemment, Bostan, Schost et Lecerf avaient prouvé qu'il
est possible transposer des programmes, sans pertes en complexité
temps ou espace, dans un modèle qui est une modeste généralisation des
programmes sans branchements. Nous généralisons leur construction en
concevant un langage fonctionnel standard pour lequel nous donnons un
algorithme de transposition. Notre nouvelle transposition est aussi
efficace que la leur sur des programmes qui font partie de leur
modèle, mais elle permet aussi de traiter des programmes plus généraux
avec des pertes mineures en complexité espace. Nous décrivons
l'implantation d'un compilateur réalisant nos idées.

Dans la Partie III, nous combinons notre transposition avec des
techniques classiques de calcul formel et théorie de l'élimination,
puis nous montrons des nouveaux résultats sur les tours
d'Artin-Schreier sur des corps finis. Ceci nous permet de concevoir
une gamme d'algorithmes qui calculent en temps quasi-optimal les
opérations arithmetiques les plus communes sur les tours
d'Artin-Schreier. Nous avons implanté ces algorithmes dans une
bibliothèque logicielle que nous décrivons à la fin de cette partie.

Enfin, la Partie IV traite le problème du calcul d'isogénies. Nous
appliques nos techniques à une variante rapide de l'algorithme de
Couveignes que nous avions proposée dans notre mémoire de Master, puis
nous comparons les résultats avec d'autres algorithmes pour le calcul
d'isogénies.

\selectlanguage{american}

% Local Variables:
% mode:flyspell
% ispell-local-dictionary:"american"
% mode:TeX-PDF
% mode:reftex
% TeX-master: "../these"
% End:
%
% LocalWords:  Schreier Artin pseudotrace frobenius bivariate Joux Sirvent FFT
% LocalWords:  Couveignes isogenies Schoof isogeny cryptosystems Lercier
% LocalWords:  precomputation arithmetics polylogarithmic Karatsuba
