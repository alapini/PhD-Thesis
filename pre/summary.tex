%% these.tex
%% Copyright 2010 Luca De Feo
%% All rights reserved

\ifafive
\setlength{\absleftindent}{0pt}
\setlength{\absrightindent}{0pt}
\fi
~\vfill
\begin{abstract}
In this thesis we apply techniques from computer algebra and language
theory to speed up the elementary operations in some specific towers
of finite fields. We apply our construction to the problem of
computing isogenies between elliptic curves and obtain faster (both
asymptotically and in practice) variants of Couveignes' algorithm.

The document is divided in four parts. In
Part~\ref{part:prerequisites} we recall some basic notions from
algebra and complexity theory. Part~\ref{part:transp-princ} deals with
the transposition principle: in it we generalize ideas of Bostan,
Schost and Lecerf, and show that it is possible to automatically
transpose computer programs without losses in time complexity and with
a small loss in space complexity. Part~\ref{part:fast-arithm-using}
combines the results on the transposition principle with classical
techniques from elimination theory; we apply these ideas to obtain
asymptotically optimal algorithms for the arithmetic of Artin-Schreier
towers of finite fields. We also describe an implementations of these
algorithms. Finally, in Part~\ref{part:appl-isog-comp} we use the
previous results to speed up Couveignes' algorithm and compare the
result with the other state of the art algorithms for isogeny
computation. We also present a new generalization of Couveignes'
algorithm that computes isogenies of unknown degree.
\end{abstract}

\vfill

\selectlanguage{french}

\begin{abstract}
Dans cette thèse nous appliquons des techniques provenant du calcul
formel et de la théorie des langages afin d'améliorer les opérations
élémentaires dans certaines tours de corps finis. Nous appliquons
notre construction au problème du calcul d'isogénies entre courbes
elliptiques et obtenons une variante plus rapide (à la fois en théorie
et en pratique) de l'algorithme de Couveignes.

Le document est divisé en quatre parties. Dans la
partie~\ref{part:prerequisites} nous faisons des rappels d'algèbre et
de théorie de la complexité. La partie~\ref{part:transp-princ} traite
du principe de transposition: nous généralisons des idées de Bostan,
Schost et Lecerf et nous montrons qu'il est possible de transposer
automatiquement des programmes sans pertes en complexité-temps et avec
une petite perte en complexité-espace. La
partie~\ref{part:fast-arithm-using} combine les résultats sur le
principe de transposition avec des techniques classiques en théorie de
l'élimination; nous appliquons ces idées pour obtenir des algorithmes
asymptotiquement optimaux pour l'arithmétique des tours
d'Artin-Schreier de corps finis. Nous décrivons aussi une implantation
de ces algorithmes. Enfin, dans la partie~\ref{part:appl-isog-comp}
nous utilisons les résultats précédents afin d'accélérer l'algorithme
de Couveignes et de comparer le résultat avec les autres algorithmes
pour le calcul d'isogénies qui font l'état de l'art. Nous présentons
aussi une nouvelle généralisation de l'algorithme de Couveignes qui
calcule des isogénies de degré inconnu.
\end{abstract}
\selectlanguage{american}

\vfill

% Local Variables:
% mode:flyspell
% ispell-local-dictionary:"american"
% mode:TeX-PDF
% mode:reftex
% TeX-master: "../these"
% End:
%
% LocalWords:  Schreier Artin pseudotrace frobenius bivariate Joux Sirvent FFT
% LocalWords:  Couveignes isogenies Schoof isogeny cryptosystems Lercier
% LocalWords:  precomputation arithmetics polylogarithmic Karatsuba
