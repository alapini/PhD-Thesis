\section{Asymptotic complexity}
\label{sec:asympt-compl}
Many algorithms below rely on fast multiplication; thus, we let $\Mult
: \N \rightarrow \N$ be a {\em multiplication function}, such that
polynomials in $\F_p[X]$ of degree less than $n$ can be multiplied in
$\Mult(n)$ operations, under the conditions of~\cite[Ch.~8.3]{vzGG}.
Typical orders of magnitude for $\Mult(n)$ are $O(n^{\log_2(3)})$ for
Karatsuba multiplication or $O(n\log (n) \log\log (n))$ for FFT
multiplication. Using fast multiplication, fast algorithms are
available for Euclidean division or extended GCD~\cite[Chapter~9 \&
11]{vzGG}.

The cost of {\em modular composition}, that is, of computing $F(G)
\bmod H$, for $F,G,H\in\F_p[X]$ of degrees at most $n$, will be
written $\ModComp(n)$. We refer to~\cite[Chapter~12]{vzGG} for a
presentation of known results in an algebraic computational model: the
best known algorithms have subquadratic (but superlinear) cost in
$n$. Note that in a boolean RAM model, the algorithm of~\cite{KeUm08}
takes quasi-linear time.

\section{Fundamental algorithms}
\label{sec:fund-algor}
In this section we review some fundamental algorithms that we will
repeatedly use in the rest of the document. Most of the algorithms we
present are taken from~\cite{vzGG}; another source of inspiration
is~\cite{todo}.

\paragraph{Polynomial multiplication}
Multiplication of polynomials with coefficients in a ring is a
fundamental brick to which most of the algorithms in computer algebra
reduce. See \cite[Chapter~9]{vzGG} for a complete account.

In the previous section we introduced the notation $\Mult(n)$ to
denote the number of operations in $R$ required to multiply two
polynomials of degree at most $n$ in $R[X]$.  Using the school-book
algorithm, we have $\Mult(n) = O(n^2)$. The first major step forward
in the complexity of multiplication was done by Karatsuba.  While a
naive algorithm would make $4$ recursive calls, Karatsuba's makes only
$3$. It follows that its complexity is $O(n^{\log_23})$.

\begin{algorithm}
  \caption{Karatsuba multiplication}
  \begin{algorithmic}
    \STATE Let $f = f_1X^{n/2} + f_2$, $g = g_1X^{n/2} + g_2$;
    \STATE compute $p_1 = (f_1 + f_2)(g_1 + g_2)$;
    \STATE compute $p_2 = f_1g_1$;
    \STATE compute $p_3 = f_2g_2$;
    \STATE return $p_3X^n + (p_1-p_2-p_3)X^{n/2} + p_2$.
  \end{algorithmic}
\end{algorithm}

When the base ring $R$ is a field containing \emph{enough} roots of
unit, algorithms based on the Fast Discrete Fourier Transform achieve
a complexity of $O(n\log n)$.


Newton inversion (vzGG chap
11), matrix multiplication, modular composition, GCD
\cite[$\S$11.1]{vzGG}, rational fraction reconstruction
\cite[$\S$5.8]{vzGG} ,chinese remainder, subproduct tree
(~\cite[Algorithm~10.3]{vzGG}), interpolation
\cite[$\S$10.2]{vzGG},transposed mod,frobenius composition(vzGS92
Algorithm~5.2),Kronecker substitution (vzGS92)\ldots

Transposed mod (or maybe later)
Bivariate operations (some Pascal-Schost and some Li-Moreno-Schost) ?


%%% Local Variables: 
%%% mode:flyspell
%%% ispell-local-dictionary:"american"
%%% mode: TeX-PDF
%%% mode: reftex
%%% TeX-master: "../these"
%%% End: 
