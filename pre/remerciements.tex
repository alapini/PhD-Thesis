%% these.tex
%% Copyright 2010 Luca De Feo
%% All rights reserved


\selectlanguage{french}

\chapter*{Remerciements}

Cher lecteur, le document que vous avez entre les mains est le fruit
de trois années de travail, que, au contraire de ce que l'on a
tendance à imaginer d'un chercheur, je n'ai pas passées enfermé avec
mes bouquins et mon ordinateur, isolé du reste du monde. Ainsi,
j'aimerais abuser de votre temps de lecture pour remercier les
personnes qui, d'une façon ou de l'autre, ont participé à ces années
de thèse.

Tout d'abord, il y a mes mentors, François Morain et Éric Schost, a
qui vous devez quasiment tous les aspects scientifiques de cette
thèse: le sujet, les références et l'exactitude des théorèmes (s'il en
reste de faux, c'est entièrement ma faute).

D'autres collègues ont grandement contribué au contenu de ce document,
que ce soit en travaillant côte à côte, en me donnant des références
bibliographiques, ou bien en relisant et critiquant certaines parties:
Mathieu Boespflug, Alin Bostan, Alexandre Benoît, Jacques Carette,
Jean-Marc Couveignes, Léonard Gérard, Romain Lebreton et Benjamin
Smith.

La relecture et la correction du manuscrit n'auraient pas été
complètes, sans l'excellent travail des rapporteurs, Erich Kaltofen et
Christophe Ritzenthaler, dont les remarques ont grandement contribué à
son amélioration. 
% Les membres du jury, qui ont voyage des distances parfois
% considérables pour se rendre à ma soutenance, ont aussi critiqué ?

Mes recherches auraient peut-être été moins abouties si je n'avais pas
eu la chance de travailler dans des laboratoires de recherche animés
et stimulants. Au sein du LIX, ma pensée va à Daniel Augot, à
Jean-François Biasse et aux autres collègues de l'équipe TANC et du
séminaire Cal4doc avec qui j'ai eu le plaisir de travailler.  Une
partie considérable de ce travail a été développée dans le SCL à
l'University of Western Ontario, dont les membres ont toujours été
chaleureux et accueillants; j'aimerais remercier en particulier Marc
Moreno Maza et Stephen Watt. Edlyn Teske a été une charmante hôtesse
lors de ma visite à Waterloo.  On sait que les chercheurs sont de
piètres planificateurs: sans l'aide d'Evelyne Rayssac tous ces voyages
auraient été impossibles.

Si vous trouvez ce livre agréable à lire et plaisant aux yeux, c'est
en partie grâce aux talents artistiques de Rachel Deyts. Ses talents
linguistiques y sont aussi pour beaucoup, lorsque vous plongez dans
mon français. Parler de ses talents culinaires et de compagne nous
amènerait trop loin.

La beauté de la science réside dans son étendue. J'ai eu un immense
plaisir à passer des heures avec Simone De Liberato à décortiquer les
sujets scientifiques les plus variés, de la physique quantique à la
théorie de la complexité. Son influence sur mes recherches, bien que
difficile à localiser dans cette thèse, est indéniable.

Enfin, j'aimerais remercier les gens avec qui j'ai partagé les
\emph{autres} moments de ces trois années de thèse: ma famille, mes
amis, le COF, mes camarades de Xdoc et de ELSE.

\selectlanguage{american}


%%% Local Variables: 
%%% mode:flyspell
%%% ispell-local-dictionary:"francais"
%%% mode: TeX-PDF
%%% mode: reftex
%%% TeX-master: "../these"
%%% End: 
