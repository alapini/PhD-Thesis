\chapter[Intoduction (English)][Introduction]{Introduction}

\pdfmcone{Removed my beautiful incipit.}
Finite field arithmetics are at the hearth of modern technology; this
is so true, that the last generation of Intel Core processors supports
a hardware instruction (CLMUL) for multiplication in
$\F_{2^m}$~\cite{intel-carryless}. The reason is that finite fields
appear everywhere in telecommunications engineering, in particular in
Error Correcting Codes and Cryptography. This thesis applies advanced
algorithmic and algebraic techniques to computations in towers of
extensions of finite fields, in view of applications to elliptic curve
cryptography.


\paragraph*{Elliptic curves}
In elliptic curve cryptography, in order to build a secure
cryptosystem, one must select a curve whose number of points contains
a large enough prime factor. The preferred method for doing this is to
randomly select a curve and then use a point-counting algorithm to
determine its cardinality. The first polynomial time point counting
algorithm for elliptic curves was due to Schoof~\cite{schoof85}, then
improved by Atkin and Elkies~\cite{atkin88,elkies98,schoof95},
henceforth named SEA.

The SEA algorithm raised interest in explicit computations with
isogenies, i.e.\ algebraic group morphisms of elliptic curves. When
computing isogenies over finite fields one must distinguish between
the large and arbitrary characteristic. In the first case, one can use
algorithms that work for characteristic $0$, and then reduce the
result; the methods of Elkies~\cite{elkies98,morain95},
Atkin~\cite{schoof95} and Bostan, Morain, Salvy and
Schost~\cite{bostan+morain+salvy+schost08} belong to this family. When
the reduction modulo the characteristic introduces division by $0$,
these algorithms are not of help.

The first two algorithms to compute isogenies in arbitrary
characteristic are due to Couveignes~\cite{couveignes94,couveignes96}:
both have a polynomial dependency in the characteristic, which makes
them unpractical for values higher than $2$ or $3$. An algorithm
specific to characteristic $2$ was given by Lercier~\cite{lercier96};
in practice it performs faster than Couveignes' algorithms, but its
complexity is not well understood. 

\pdfmcone{Rapidly recall the existence of p-adic methods.}
After the discovery of $p$-adic alternatives to the SEA
algorithm~\cite{satoh00,fouquet+gaudry+harley00} interest in computing
isogenies in small characteristic was lost.  Nevertheless, two
$p$-adic algorithms were recently proposed by Joux and
Lercier~\cite{joux+lercier06} and Lercier and
Sirvent~\cite{lercier+sirvent08} to solve the isogeny problem in
arbitrary characteristic. They show that it is possible to avoid
divisions by $0$ by lifting the curves in the $p$-adics. The last
algorithm is currently the one having the best asymptotic complexity
for the arbitrary characteristic case; its complexity in the
characteristic is only logarithmic.

It is interesting to remark, however, that no algorithm to compute
isogenies has optimal or quasi-optimal complexity, with the only
exception of~\cite{bostan+morain+salvy+schost08} on a very special
case. 

The starting point of this work was Couveignes' second
algorithm~\cite{couveignes96}. It computes an isogeny by interpolating
it over the $p^k$-torsion points of the elliptic curves for a large
enough $k$; when those points are not defined on the base field, one
has to take towers of field extensions to find them. The field
extensions that naturally arise when doing this computation are
splitting fields of polynomials of the form
\[X^p - X -\alpha\text{;}\] such extension are called Artin-Schreier
extensions. 


\paragraph*{Towers of finite fields}
Besides addition, multiplication and inversion, the arithmetic
operations of interest in a tower of finite extensions arguably are
relative traces, minimal polynomials and embeddings. For finite fields
one could add explicit Galois groups to the list as these are
relatively easy to compute with.

The arithmetic of towers of finite fields is a central question for
any computer algebra system, however it has received few attention, if
any. Magma is known for having had support for lattices of finite
fields for a long time~\cite{bosma+cannon+steel97}, but it is hard to
tell which algorithms it implements nowadays and what their
complexities are. All other results that can possibly apply to towers
of finite fields were derived in the more general context of
polynomial system solving and effective algebraic geometry, in
particular in the resolution of triangular
sets~\cite{diaz+gonzalez01,giusti+lecerf+salvy01,bostan+salvy+schost03,pascal+schost06,li+moreno+schost07,dahan+jin+moreno+schost08,boulier+lemaire+moreno01,FGLM,rouiller99,alonso+becker+roy+wormann}.

In the specific case of Artin-Schreier towers, the literature is not
extensive either.  Using ideas from~\cite{Conway:ONAG2000},
Cantor~\cite{cantor89} constructs a particular Artin-Schreier tower
that he applies to FFT multiplication in $\F_2[X]$.
In~\cite{couveignes00}, Couveignes gives an algorithm to compute
isomorphisms between Artin-Schreier towers; however, his algorithm
needs as a prerequisite a fast multiplication algorithm in a tower,
called a ``Cantor tower'' in~\cite{couveignes00}, having the same
shape as the one in~\cite{cantor89}. Such an algorithm is
unfortunately not in the literature, making the results
of~\cite{couveignes00} non practical.


\paragraph*{Transposition principle}
One algorithmic tool that we shall study in depth and apply throughout
the whole document is the \emph{transposition principle}, which is the
language-theoretic counterpart to algebraic duality.

The transposition principle was discovered in electrical network
theory by Bordewijk~\cite{bordewijk57}, then proved in its general
form by Fiduccia~\cite{fiduccia:phd}; but it only became popular in
computer algebra much later through the works of Kaltofen, Yagati,
Shoup, von zur Gathen and
others~\cite{kaltofen+lakshman89,vzgathen+shoup92,shoup94,shoup95,shoup99,hanrot+quercia+zimmermann}. One
possible statement is:
\begin{quote}
  Let $\pspace$ be an arbitrary set. To any $R$-algebraic algorithm
  $A$ computing a family of linear functions $(f_p:M\ra
  N)_{p\in\pspace}$ corresponds an $R$-algebraic algorithm $\dual{A}$
  computing the \emph{dual family}
  $(\dual{f}_p:\dual{N}\ra\dual{M})_{p\in\pspace}$. The algebraic time
  and space complexities of $\dual{A}$ are bounded by the time
  complexity of $A$.
\end{quote}

The transposition principle is important in computer algebra because
it permits to derive asymptotically good algorithms that were not
otherwise evident. One big step forward in the understanding of it was
done by Bostan, Lecerf and Schost~\cite{bostan+lecerf+schost:tellegen}
who, extending work of Shoup~\cite{shoup95}, remarked that
transposition can be systematically applied to a restricted
programming language. It is also remarkable that the transposition
principle has a strong connection with automatic
differentiation~\cite{baur+strassen83,kaltofen+lakshman89,Ka2K,gashkov+gashkov05,sergeev08}.

In this document we investigate more in depth the relationships
between the transposition principle and programming languages. We use
the theory of typed purely functional languages~\cite{pierce} as
framework, because its elegant mathematical structure permits to
reason at an algebraic level on programs.


\paragraph*{Outline of our contributions}
This document is divided in four parts. Part~\ref{part:prerequisites}
recalls the basic notions from algebra and computer algebra that we
will use later.

Part~\ref{part:transp-princ} studies the transposition principle. In
Chapter~\ref{cha:algebr-compl-dual} we review the arithmetic circuit
model and the straight line program model, and prove the transposition
theorem in them. Then we discuss the relationships with automatic
differentiation. As a complement, in
Appendix~\ref{cha:basic-categ-theory} we also give a new proof of the
transposition theorem, using categorical semantics, and discuss its
consequences on the implementation of a DSL in Haskell; this is joint
work with Boespflug.

Chapter~\ref{cha:autom-transp-code} is a collaboration with Schost. We
study the relationships between the arithmetic circuit model and
functional programming languages, then we show that transposition can
be applied algorithmically to a generic functional language. 

Part~\ref{part:fast-arithm-using} is devoted to arithmetics in towers
of extensions. We start by reviewing the general theory of
zero-dimensional ideals and rational univariate representations in
Chapter~\ref{cha:trace-computations}. Here, the results of
Part~\ref{part:transp-princ} are the key to obtain asymptotically fast
algorithms. The algorithms of this chapter are then applied in
Chapter~\ref{cha:artin-schr-towers}, where we provide asymptotically
good algorithms for Artin-Schreier towers (fruit of another
collaboration with Schost).

\pdfmcone{More emphasis on C2-UD and its cryptographic interest.}
Finally Part~\ref{part:appl-isog-comp} applies the results of the
previous chapters to isogeny computation. After some general
references on elliptic curves in Chapter~\ref{cha:ellipt-curv-isog},
we review in Chapter~\ref{cha:algor-small-char} the asymptotically
fastest algorithms to compute isogenies over finite fields.  We start
by reviewing the BMSS algorithm for large
characteristic~\cite{bostan+morain+salvy+schost08} and its
generalization for arbitrary characteristic by Lercier and
Sirvent~\cite{lercier+sirvent08}; then we review Couveignes' original
algorithm~\cite{couveignes96}, and present some improved variants with
better asymptotic behavior: the key to this results are
Chapter~\ref{cha:artin-schr-towers} and new ideas on interpolation in
towers of extensions.  We also present in Section~\ref{sec:bounded} a
surprising generalization of Couveignes' algorithm that allows to
compute isogenies of unknown degree at the same cost of computing an
isogeny of a given degree.  This discovery sheds new light on the
(sub)optimality of Couveignes' algorithm and can possibly find
applications in
cryptology~\cite{gaudry+hess+smart02,GHS,hess03,teske06}.

Theory would be meaningless without practice. Similarly, this
manuscript would make no sense if it was not accompanied by our
software packages. The great majority of the algorithms we present
here have been implemented, packaged and distributed under open source
licences. So, all the algorithms of
Chapter~\ref{cha:artin-schr-towers} can be found in the \texttt{C++}
library \texttt{FAAST}, available from
\url{http://www.lix.polytechnique.fr/~defeo/FAAST/}. At the
moment we write, the compiler for the language \texttt{transalpyne} of
Chapter~\ref{cha:autom-transp-code} is not distributed yet; we are
currently working on the first stable release and hope to start
distributing it by the beginning of 2011. It will be available from
\url{ http://transalpyne.gforge.inria.fr/}.




%%% Local Variables: 
%%% mode:flyspell
%%% ispell-local-dictionary:"american"
%%% mode: TeX-PDF
%%% mode: reftex
%%% TeX-master: "../these"
%%% End: 
