Here we recall the basic concepts from abstract algebra that
constitute the background for all the chapters that follow. To the
reader interested in reading more about these topics, we recommend
\cite{lang} for general algebra, \cite{lidl+niederreiter:2} for finite
fields and \cite{silverman:elliptic,silverman:advanced} for elliptic
curves.

\section{Groups, Ring, Fields}
\label{sec:ring-fields}

\subsection{Objects}
\label{sec:ring-fields:objects}

A \emph{\index{group}group} is a pair $(G,\cdot)$ such that $G$ is a
set and $\cdot:G\times G\ra G$ is an \emph{internal composition law}
satisfying:
\begin{itemize}
\item \emph{\index{associativity}Associativity}: $(a\cdot b)\cdot c =
  a \cdot (b\cdot c)$ for any $a,b\in G$;
\item There is an element $e\in G$, called the \emph{\index{identity
      element}identity}, such that $a\cdot e = e\cdot a = a$ for any
  $a\in G$;
\item For any $a\in G$ there is an \emph{\index{inverse
      element}inverse element} $b$ such that $a\cdot b = b\cdot a =
  e$.
\end{itemize}
If $\cdot$ also satisfies \emph{\index{commutativity}commutativity}
$a\cdot b=b\cdot a$, the group is said to be
\emph{\index{group!abelian}abelian}. The group composed of one single
element with the obvious law is called the
\index{group!trivial}\emph{trivial group}.

A \emph{\index{subgroup}subgroup} of a group $(G,\cdot)$ is a group
$(H,\circ)$ such that $H\subset G$ and $\circ$ is the restriction of
$\cdot$ to $H$. Any group has two trivial subgroups: the trivial group
and itself.

A \emph{\index{ring}ring} is a tuple $(R,+,\cdot)$ such that $(R,+)$
is an abelian group and $\cdot:R\times R\ra R$ is an internal
composition law satisfying associativity, existence of the identity
and \emph{\index{distributivity}distributivity over $+$}
\[a \cdot (b + c) = (a\cdot b) + (a\cdot c) \quad\text{for any
  $a,b,c\in R$.}\] When $\cdot$ satisfies commutativity, the ring is
said to be \emph{\index{ring!commutative}commutative}.  The law $+$ is
called \emph{addition}, $\cdot$ is called \emph{multiplication}, the
identity for $+$ is denoted by $0$ and the identity for $\cdot$ by
$1$.  A commutative ring such that $1\ne 0$ and where $\cdot$ also
satisfies the existence of the inverse, is called a
\emph{\index{field}field}.

The simplest example of ring is $\Z$, the set of integers; the
rational numbers $\Q$ are an example of field, it is the ``smallest''
field containing $\Z$ as a subring. The \emph{\index{ring!trivial}trivial
  ring} is the ring composed of one unique element $r=0=1$ with the
evident laws; note that by definition this is not a field.

Subrings, ideals

Modules, free modules, vector spaces

Cosets, quotients

\subsection{Arrows}
\label{sec:ring-fields:arrows}

Morphisms, kernels, isomorphism theorems, 

\section{Linear algebra}
\label{sec:linear-algebra}

duality
matrices, trace, determinant, resultant

\section{Basic Galois theory}
\label{sec:basic-galois-theory}
Field extension, splitting field, Galois extensions, algebraic closure

roots of unity, cyclotomic polynomials

trace, norm

Artin-Schreier

finite fields


\section{Elliptic curves}
\label{sec:elliptic-curves}



%%% Local Variables: 
%%% mode:flyspell
%%% ispell-local-dictionary:"american"
%%% mode: TeX-PDF
%%% mode: reftex
%%% TeX-master: "../these"
%%% End: 


