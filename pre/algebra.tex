%% these.tex
%% Copyright 2010 Luca De Feo
%% All rights reserved


Here we recall the concepts from abstract algebra that will constitute
the background for all the chapters that follow. One chapter is
certainly not enough to present such a vast subject, hence we just
recall the few definitions and properties that will help the reader
understand the results presented in this document. The material of
this chapter is mainly drawn from
\cite{lang,lidl+niederreiter:2,silverman:elliptic}.

\section{Linear algebra}
\label{sec:linear-algebra}
In Part~\ref{part:transp-princ} we shall apply some classical linear
algebraic tools to free modules over non-commutative ring. We recall
here the fundamental concepts.


\subsection{Bra-ket notation}
\label{sec:linear-algebra:bra-ket}

It will be convenient to (ab)use Dirac's
\index{bra-ket~notation}bra-ket notation to represent elements of
modules. If $(M,+,\cdot)$ is a left $R$-module and $x\in (M,+)$ is an
element of its underlying group, by $\ket{x}_R$ we mean the element
obtained by lifting $x$ in $(M,+,\cdot)$. We call
$\ket{x}$\symb[braket-1]{$\ket{x}$}{Ket, element of a left module} a
\index{ket}\emph{ket} and read it as ``ket x''.

The external multiplication by an element $a\in R$ will be written
$a\ket{x}_R$; if $f:M\ra N$ is a left module morphism, we write
$f\ket{x}_R$ for $f(\ket{x}_R)$. By a slight abuse of notation we may
write $\ket{a x}_R$ and $\ket{f(x)}_R$ for $a\ket{x}_R$ and
$f\ket{x}_R$ respectively. When $R$ is clear from the context, a ket
can be simply written as $\ket{x}$.

\symb[braket-2]{$\bra{x}$}{Bra, element of a right module} In a
symmetric way, elements of right $R$-modules will be written
${}_R\!\bra{x}$, which we call a \index{bra}\emph{bra} and read as
``bra x''. External multiplication will be written as ${}_R\!\bra{x}a$
and application of a right module morphism as ${}_R\!\bra{x}f$.

Let $M$ be a right module and $N$ a left module. A
\index{bilinear~form}\emph{bilinear
  form}\symb[braket-3]{$\braket{x}{y}_f$}{Bilinear form} on
$M\times N$ is a map $f:M\times N\ra R$ such that for any $x\in M$,
the map
\[\ket{y}\mapsto f(x, y)\]
is a left module morphism, and for any $y\in N$, the map
\[\bra{x}\mapsto f(x, y)\]
is a right module morphism. If $f$ is a bilinear form, we write
$\braket{x}{y}_f$ for $f(x,y)$, or simply $\braket{x}{y}$ when $f$ is
clear from the context. Note that textbooks usually define bilinear
forms only when $R$ is commutative, in our more general setting some
common properties of bilinear forms fail to hold, for example
\pdfmcthree{Changed inequality in "not necessarily".}
$\braket{xa}{y}$ is not necessarily equal to $\braket{x}{ay}$.

If $M$ is a left (right) module, we denote by $\dual{M}=\hom(M,R)$ the
\index{dual~module}\emph{dual
  module}\symb[dual]{$\dual{M}$}{Dual of a module or vector
  space: $\dual{M}=\hom(M,R)$} of $M$, it is a right (left) module.
Any bilinear form $f$ gives rise to a morphism $ \phi_f : M \ra
\dual{N}$ of right modules where $\bra{x}\phi_f$ is the linear form $y
\mapsto \braket{x}{y}$. Similarly, $f$ gives rise to a morphism
$\phi^f:N\ra\dual{M}$ of left modules. The maps $f\mapsto\phi_f$,
$f\mapsto\phi^f$ and their obvious inverses induce group isomorphisms
between $\hom(M,\dual{N})$, $\hom(N,\dual{M})$ and the group of
bilinear forms on $M\times N$.  A bilinear form $f$ is said to be
\index{bilinear~form!degenerate}\emph{non-degenerate} if $\phi_f$
and $\phi^f$ are module isomorphisms.
\pdfmcone{The distinction
  non-singular/non-degenerate comes from Lang, but it is not very
  standard, indeed. I removed the distinction: now I call
  non-degenerate what I used to call non-singular before.}


\subsection{Matrices and morphisms}
\label{sec:linear-algebra:matrices}
$M=M_1\oplus\cdots\oplus M_n$ be a left module and
$N=N_1\oplus\cdots\oplus N_m$ be a right module.  Let $\iota_i$ be the
injections $M_i\ra M$ and let $\pi_j$ be the projections $N\ra N_i$,
then a linear map $f:M\ra N$ is uniquely determined by the maps
$\pi_j\circ f\circ\iota_i$. If we consider $m\times n$ matrices whose
$(j,i)$-th coefficient is in $\hom(M_i,N_j)$, then we verify that
there is a group isomorphism between $\hom(M,N)$ and this group of
matrices. Furthermore, let $f:M\ra N$ and $g:N\ra O$ and let $M_f$ and
$M_g$ be the matrices that are associated respectively, then the
matrix associated to $g\circ f$ is $M_gM_f$, where the product of two
entries is defined as composition of morphisms. This induces a ring
isomorphism between $\End(M)$ and the ring of square matrices with
entries in $\hom(M_i,M_j)$.

Consider $R$ as an $R$-module over itself, a linear map from $R$ to
itself is uniquely determined by the image of $1$, hence $\End(R)\isom
R^\op$. As a consequence, there is a group isomorphism
$\hom(R^n,R^m)\isom\Mat_{m\times n}(R^\op)$, and matrix multiplication
is equivalent to composition as above.  Hence, if $M$ is a free
module, for any fixed basis $\basis{B}$ of cardinality $n$ we have an
isomorphism of rings $\End_R(M)\isom\Mat_n(R^\op)$; in particular
$\Aut(M)\isom\GL_n(R^\op)$ as groups.

Let $R$ be commutative, then $R^\op = R$. We denote by
$M_{\basis{B}}(f)$ the matrix associated to $f\in\End_R(M)$ with
respect to the basis $\basis{B}$.  If $\basis{B'}$ is another basis,
it has the same cardinality as $\basis{B}$. Then, there is an
invertible matrix $B$ such that $A\mapsto B^{-1}AB$ is the
automorphism of $\Mat_n(R)$ that sends $M_{\basis{B}}(f)$ over
$M_{\basis{B'}}(f)$. Hence, any property of matrices that is invariant
by similarity, can be defined for linear operators. We define the
\index{trace!of~an~operator} \index{trace}\emph{trace} of a linear
operator as $\Tr f = \Tr M(f)$\symb[Tr]{$\Tr$}{Trace of a
  matrix, of a linear operator}, and its
\index{determinant}\emph{determinant} as $\det f = \det
M(f)$\symb[det]{$\det$}{Determinant of a matrix, of a linear
  operator}.


\subsection{Duality}
\label{sec:linear-algebra:duality}
We fix a non-degenerate bilinear form $f$ on $M\times N$. Let
$g\in\End(M)$, then the map
\[(x,y) \mapsto \braket{g(x)}{y}_f\] is a bilinear form. On the other
hand, let $h$ be a bilinear form on $M\times N$, for any $x\in M$ the
map $h_x : \ket{y} \mapsto \braket{x}{y}_h$ is a linear form on $N$,
thus $h_x\in\dual{N}$. From the non-degeneracy of $f$ we deduce that
there is an unique element $x'\in M$ such that
$\braket{x'}{y}_f=\braket{x}{y}_h$ and it is clear that the map
$\bra{x}\mapsto \bra{x'}$ is an endomorphism of $M$. It is evident
that the two maps are each other's inverse, thus we have a group
isomorphism between $\End(M)$ and the group of bilinear forms. An
analogous argument shows that $\End(N)$ is isomorphic to the group of
bilinear forms and ultimately $\End(M)\isom\End(N)$.

A consequence of this is that for any linear operator $g\in\End(M)$
there is an operator $\dual{g}\in\End(N)$ such that
\[\braket{g(x)}{y}_f = \braket{x}{\dual{g}(y)}_f\]
for any $x\in M$ and $y\in N$. We define similarly $\dual{h}$ when
$h\in\End(N)$, obviously $\dual{(\dual{g})} = g$. The operator
$\dual{g}$ is called the
\index{dual~operator}\emph{dual}\symb[dual]{$\dual{g}$}{Dual
  operator: $\braket{g(x)}{y}=\braket{x}{\dual{g}(y)}$} of $g$ with
respect to $f$. In general, whenever it is clear from the context that
$g$ belongs to $\End(M)$ (or to $\End(N)$), we simply
write\symb[braket-4]{$\braketop{x}{g}{y}$}{Bilinear form with linear operator}
\[\braketop{x}{g}{y} \eqdef \braket{g(x)}{y} = \braket{x}{\dual{g}(y)}\text{.}\]

More generally, Let $f:M\times M'\ra R$ and $g:N'\times N\ra R$ be two
non-degenerate bilinear forms, by the same technique as above we can
show that there is a group isomorphism between $\hom_R(N,M')$,
$\hom(M,N')$ and the bilinear forms on $M\times N$. Then, for any
$h:N\ra M'$ there is an unique $\dual{h}:M\ra N'$ such that
\[\braketop{x}{h}{y}\eqdef\braket{x}{h(y)}_f = \braket{\dual{h}(x)}{y}_g 
\text{.}\]
We also call $\dual{h}$ the \emph{dual} of $h$.

The canonical example of non-degenerate bilinear forms is obtained by
considering the family of forms on $\dual{M}\times M$ defined by
\[\braket{\ell}{x} = \ell(x) \text{.}\]
For any $f:M\ra N$, we define the dual map
$\dual{f}:\dual{N}\ra\dual{M}$ as the map that sends a form
$\ell\in\dual{N}$ over the form $\ell\circ f$ in $\dual{M}$; it is
easy to verify that
\[\braketop{\ell}{f}{x} =  \braket{\ell}{f(x)} = \braket{\dual{f}(\ell)}{x}
= \ell(f(x))\text{.}\]


\pdfmcone{Introduced definition of dual basis, clarified messy
  comments about columns and vectors.}  If $M$ is a free module and
$\basis{B}=\{\basis{e}_1,\ldots,\basis{e}_n\}$ a basis, the
\index{dual~basis}\emph{dual
  basis}\symb[dual]{$\dual{\basis{B}}$}{Dual basis}
$\dual{\basis{B}}$ is the unique basis
$\{\dual{\basis{e}_1},\ldots,\dual{\basis{e}_1}\}$ of $\dual{M}$ such
that
\begin{equation*}
  \braket{\dual{\basis{e}_i}}{\basis{e}_j} =
  \begin{cases}
    1 &\text{if $i=j$,}\\
    0 &\text{if $i\ne j$.}
  \end{cases}
\end{equation*}
If elements of $M$ and $\dual{M}$ are represented, respectively, as
vectors over $\basis{B}$ and $\dual{\basis{B}}$, then the bilinear
form $\braket{\ell}{x}=\ell(x)$ is given by the inner product
\begin{equation*}
  \left\langle\begin{matrix}
      \ell_1 &\cdots & \ell_n
    \end{matrix}\right\rvert
  \left\lvert\begin{matrix}
    x_1\\
    \vdots\\
    x_n
  \end{matrix}\right\rangle
  =
  \sum_i x_i\ell_i
\end{equation*}
(notice how the product is swapped, this is because $\End(R)\isom
R^\op$).  Now, if $M$ and $N$ are free modules with a fixed basis, a
linear map $f:M\ra N$ is isomorphic to a matrix with entries in
$R^\op$. Then the application $f\ket{x}$ is just matrix-vector
multiplication, while $\bra{\ell}\dual{f}$ is vector-matrix
multiplication by the same matrix. This justifies the notation
$\braketop{\ell}{A}{x}$ where $A$ is the matrix associated to
$f$.


\section{Basic Galois theory}
\label{sec:basic-galois-theory}
In Parts~\ref{part:fast-arithm-using} and~\ref{part:appl-isog-comp} we
shall need some basic Galois theory of finite fields. We recall here
the general concepts.

\subsection{Galois extensions}
\label{sec:basic-galois-theory:galois-extensions}
\pdfmcone{Removed pastiche about splitting fields} Let $\K$ be a
field. The \index{splitting~field}\emph{splitting field} of a family
of polynomials $(Q_i)_{i\in I}$ in $\K[X]$ is defined as an extension
$\LK$ of $\K$ where all the $Q_i$'s factor completely into linear
factors, and such that $\LK$ is generated over $\K$ by the roots of
the $Q_i$; the splitting field is unique up to isomorphism. An
algebraic field extension $\LK/\K$ such that $\LK$ is the splitting
field of a family of polynomials in $\K[X]$ is called a
\index{normal~field~extension}\emph{normal extension}.

Let $\LK/\K$ be an algebraic field extension, an element $x\in\LK$ is
said to be \index{separable!element}\emph{separable} over $\K$ if
its minimal polynomial over $\K$ has no multiple roots in $\LK$.
$\LK/\K$ is said to be \index{separable!field~extension}
\emph{separable} if every $x\in\LK$ is separable over $\K$. An
algebraic field extension is said to be a
\index{Galois~field~extension}\emph{Galois extension} if it is both
separable and normal.

\begin{theorem}
  Let $\LK/\K$ be a finite Galois extension, then there exists an
  element $x\in\LK$, called a
  \index{primitive~element}\emph{primitive element}, such that
  $\LK\isom\K[x]$.
\end{theorem}

Let $\LK/\K$ be a Galois extension, the group of automorphisms of
$\LK$ that fix $\K$ is called the \index{Galois~group}\emph{Galois
  group} of $\LK/\K$ and is denoted by
$\Gal(\LK/\K)$\symb[Gal]{$\Gal(\LK/\K)$}{Galois group}.  Let
$G$ be a group of automorphisms of a field $\K$, by
$\K^G$\symb[KG]{$\K^G$}{Fixed field, the subfield of $\K$
  fixed by the action of $G$} we denote the subfield of $\K$
consisting in the elements such that $\sigma(x)=x$ for any $\sigma\in
G$. Obviously, $\K=\LK^{\Gal(\LK/\K)}$.

\begin{theorem}
  Let $\LK/\K$ be a finite Galois extension. Let $H$ be a subgroup of
  $G=\Gal(\LK/\K)$, the map $H\mapsto \LK^H$ is a bijection between
  the subgroups of $G$ and the subfields of $\LK$ containing $\K$. The
  extension $\LK^H/\K$ is Galois if and only if $H$ is a normal
  subgroup of $G$; in this case its Galois group is isomorphic to
  $G/H$.
\end{theorem}

Let $\LK/\K$ be a Galois extension and let $x\in\LK$. The elements
$\sigma(x)$ for $\sigma\in\Gal(\LK/\K)$ are called the
\index{conjugate~element}\emph{conjugates} of $x$ under the action
of $\Gal(\LK/\K)$; they are the roots of the minimal polynomial of $x$
over $\K$.

Let $\K$ be a field, an element $x\in \K$ such that $x^n=1$ is called
an $n$-th \index{root~of~unity}\emph{root of unity}. If the
characteristic of $\K$ does not divide $n$, the polynomial $X^n-1$ has
$n$ distinct roots in $\clot{\K}$ and they form a multiplicative
group, denoted by $\mu_n$\symb[mu]{$\mu_n$}{Group of the $n$-th roots
  of unity}; it is a cyclic group, its generators are called the
\index{root~of~unity!primitive} \emph{primitive} roots of unity.  If
$\K$ has characteristic $p>0$, then $X^{p^m}-1$ has only one root,
namely $1$, thus $\mu_{p^m}$ is the trivial group.

The \index{Euler~function}\emph{Euler function}
$\euler:\N\ra\N$\symb[f]{$\euler$}{Euler totient function} is
defined as
\begin{align*}
  \euler(1) &= 1\text{,}\\
  \euler(p^r) &= p^{r-1}(p-1) &\text{for $p$ prime, $r\ge1$,}\\
  \euler(nm) &= \euler(n)\euler(nm) &\text{when $\gcd(n,m)=1$.}
\end{align*}
The Euler function counts the number of generators of the cyclic group
with $n$ elements, thus, when the characteristic of the field does not
divide $n$, the number of primitive roots of unity is equal to $\euler(n)$.

\begin{theorem}
  Let $x$ be a primitive $n$-th root of unity in an algebraic closure
  of $\Q$, then
  \[[\Q(x):\Q]=\euler(n)\text{.}\]
\end{theorem}

If $x$ is an $n$-th root of unity, its minimal polynomial over $\Q$ is
called the $n$-th \index{cyclotomic~polynomial}\emph{cyclotomic
  polynomial} and is denoted by
$\Cyclo_n$\symb[f]{$\Cyclo_n$}{$n$-th Cyclotomic polynomial};
it is a monic polynomial with coefficients in $\Z$. $\Cyclo_n$ is an
irreducible factor of $X^n-1$ over $\Q$, its roots are all the
primitive $n$-th roots of unity, hence $\deg\Cyclo_n=\euler(n)$.

\pdfmcone{Removed the Galois requirement.}  Let $\LK/\K$ be a finite
extension and let $x\in\LK$, the map $M_x:a\mapsto xa$ is an
automorphism of the $\K$-vector space $\LK$. The minimal polynomial of
its matrix with respect to any basis is equal to the minimal
polynomial of $x$ over $\K$.  The trace of $M_x$ is called the
\index{trace!of~a~field~extension}\emph{trace} of $x$ and is denoted
by $\Tr_{\LK/\K}(x)$\symb[Tr]{$\Tr_{\LK/\K}$}{Trace of a field
  extension}; its determinant is called the \index{norm}\emph{norm} of
$x$ and is denoted by
$\Norm_{\LK/\K}(x)$\symb[Norm]{$\Norm_{\LK/\K}$}{Norm of a field
  extension}.

\begin{proposition}
  \label{th:basic-galois-theory:trace}
  \pdfmcthree{"finite extension" is more precise than "field extension".}
  Let $\LK/\K$ and $\K/k$ be finite extensions and let
  $G=\Gal(\LK/\K)$. We have the following identities
  \begin{align*}
    \Tr_{\LK/\K}(x) &= \sum_{\sigma\in G}\sigma(x) \text{,}&
    \Tr_{\LK/k} &= \Tr_{\K/k}\circ\Tr_{\LK/\K}\text{,}\\
    \Norm_{\LK/\K}(x) &= \prod_{\sigma\in G}\sigma(x) \text{,}&
    \Norm_{\LK/k} &= \Norm_{\K/k}\circ\Norm_{\LK/\K}\text{.}
  \end{align*}
  The trace is a morphism of $\K$-vector spaces from $\LK$ to $\K$,
  the norm is a multiplicative morphism of groups from $\LK^\ast$ to
  $\K^\ast$.
\end{proposition}


\subsection{Finite fields}
\label{sec:basic-galois-theory:finite-fields}

Let $\K$ be a \index{finite~field} \emph{finite field}. It has
necessarily characteristic $p>0$, thus it must contain $\Z/p\Z$ as a
subfield. $\Z/p\Z$ is called the \index{prime~field}\emph{prime
  field} of $\K$ and is denoted by $\F_p$. Since $\K$ is a vector
space over $\F_p$, it must have cardinality $q=p^n$ for some $n$,
hence its multiplicative group has order $q-1$.

As a consequence, the elements of $\K^\ast$ must be roots of the
polynomial $X^{q-1}-1$. The fact that $p$ does not divide $q-1$
implies that $\K$ is isomorphic to $\F_p[\zeta]$, where $\zeta$ is a
primitive $(q-1)$-th root of unity in $\clot{\F}_p$. This implies
that, up to isomorphism, there is an unique finite field containing
$q$ elements, we denote by $\F_q$\symb[FiniteField]{$\F_q$}{Finite
  field of cardinality $q$} this field.

\pdfmcthree{Slightly reformulated.}
Using the same arguments, it is easy to show that for any $m\ge 1$,
$\F_{q^m}$ contains a subfield isomorphic to $\F_q$.  The map
$\frob_q:\F_{q^m}\ra\F_{q^m}$ sending $x\mapsto
x^q$\symb[f]{$\frob_q$, $\frob$}{Frobenius automorphism} is a morphism
of fields that fixes $\F_q$, it is called the
\index{Frobenius~automorphism}\emph{Frobenius automorphism} of
$\F_{q^m}/\F_q$. We now give the main result about the Galois theory
of finite fields.

\begin{proposition}
  The Galois group of $\F_{q^m}/\F_q$ is a cyclic group of order $m$;
  it is generated by the Frobenius automorphism $\frob_q$.
\end{proposition}


\section{Basic algebraic geometry}
\label{sec:basic-algebr-geom}

\subsection{Noetherian rings}
\label{sec:noetherian-rings}
A ring $R$ is called \index{Noetherian~ring}\emph{Noetherian} if any
ascending chain of ideals eventually terminates. Being Noetherian is a
very stable condition: fields and principal ideal domains, quotients
of Noetherian rings, rings of polynomials in finitely many variables
over a Noetherian ring, are all Noetherian. In particular, all the
rings we will work with in this document are Noetherian.

\pdfmcone{Changed definition of primary ideal.}
A proper ideal $I$ is \index{ideal!maximal}\emph{maximal} if it is
not strictly contained in any proper ideal, this is equivalent to
$R/I$ being a field. A proper ideal is
\index{ideal!prime}\emph{prime} if $R/I$ is an integral domain;
\index{ideal!primary}\emph{primary} if $ab\in I$ implies that $a\in
I$ or $b^n\in I$ for some $n$. The
\index{ideal!radical}\emph{radical} of an ideal $I$ is the
ideal\symb[I]{$\sqrt{I}$}{Radical of an ideal}
\begin{equation}
  \label{eq:212}
  \sqrt{I} = \{f \,|\, f^r\in I \text{ for some $r\ge0$.}\}
  \text{.}
\end{equation}
An ideal is said to be radical if $\sqrt{I}=I$. The radical of a
primary ideal is prime.

\pdfmcone{Forgot strict inclusion for reducibility.}
An ideal $I$ is said to be \index{ideal!reducible}\emph{reducible}
if it is strictly contained in two ideals $I_1,I_2$ such that
$I=I_1\cap I_2$, \index{ideal!irreducible}\emph{irreducible}
otherwise. Any primary ideal is irreducible; we have the following two
fundamental results about reducibility.

\begin{proposition}
  Let $R$ be Noetherian. Any radical ideal $I$ admits an unique
  decomposition
  \begin{equation}
    \label{eq:213}
    I = P_1\cap\cdots\cap P_n
  \end{equation}
  with $P_i$ prime and $P_i\not\subset P_j$ for $i\ne j$.
\end{proposition}

\begin{theorem}[Primary decomposition]
  Let $R$ be Noetherian. \index{primary~decomposition}Any ideal $I$
  admits a decomposition
  \begin{equation}
    \label{eq:214}
    I = Q_1\cap\cdots\cap Q_n
  \end{equation}
  into primary ideals. Furthermore, $\sqrt{Q_i}$ is uniquely
  determined.
\end{theorem}

Now we state a fundamental lemma that we will repeatedly use in the
next chapters.

\begin{lemma}[Chinese remainder theorem]
  \label{th:chinese-remainder}
  \index{Chinese~remainder~theorem}
  Let $I_1,\ldots,I_n$ be pairwise coprime ideals (i.e.\ $I_i+I_j=R$ if
  $i\ne j$).  Then the canonical morphism $A\ra\prod_j A/I_j$ gives an
  isomorphism of rings
  \begin{equation}
    \label{eq:211}
    A/I_1\cap\cdots\cap I_n \isom \prod_j A/I_j
    \text{;}
  \end{equation}
  and the intersection $I_1\cap\cdots\cap I_n$ equals the product
  $I_1\cdots I_n$.
\end{lemma}


\subsection{Algebraic varieties}
\label{sec:algebraic-varieties}
\pdfmcone{Substituted the ambiguous use of "variety" with "set of
  zeros".}  We now consider the polynomial ring $\K[x_1,\ldots,x_n]$,
where $\K$ is a perfect field with algebraic closure $\clot{\K}$. To
any ideal $I$, we associate its
\index{set~of~zeros}\symb[ideal]{$V(I)$}{Set of zeros of the ideal
  $I$}\emph{set of zeros}
\begin{equation}
  \label{eq:215}
  V(I) = \{x\in\clot{\K}^n \,|\, f(x) = 0 \text{ for any } f\in I\}
  \text{.}
\end{equation}
\pdfmcone{Detailed definition of I(V), so that it is compatible with
  the version of the Nullstellensatz given later.} Reciprocally, to
any $V\subset\clot{\K}^n$ we associate the ideal
\index{ideal!vanishing}\symb[ideal]{$I(V)$}{Ideal vanishing at the
  algebraic set $V$}\emph{vanishing at $V$}
\begin{equation}
  \label{eq:216}
  I(V) = \{f\in\clot{\K}[x_1,\ldots,x_n] \,|\, f(x) = 0 \text{ for any } x\in V\}
  \text{.}
\end{equation}

\pdfmcone{Defined the affine space (and the projective space three
  paragraphs later).} A subset of $\clot{\K}^n$ is called an
\index{algebraic~set!affine}\emph{affine algebraic set} if it is the
set of zeros of an ideal of $\clot{\K}[x_1,\cdots,x_n]$.  The
\index{affine~space}\emph{affine space} of dimension $n$, denoted by
\symb[affspace]{$\mathbb{A}^n$}{$n$-dimensional affine
  space}$\mathbb{A}^n$, is the affine algebraic set associated to the
zero ideal.

An algebraic set $V$ is
\index{algebraic~set!defined~over~a~field}\emph{defined over} $\K$ if
$I(V)$ has a set of generators in $\K[x_1,\ldots,x_n]$; in this case
we denote by \symb[ideal]{$V(\K)$}{$\K$-rational points of an
  algebraic set}$V(\K)$ the subset $V\cap\K^n$.

An algebraic set is
\index{algebraic~set!irreducible}\emph{irreducible} if it cannot be
written as the union of two proper algebraic sets; equivalently, it is
irreducible if $I(V)$ is prime. An irreducible affine algebraic set is
called an \index{variety!affine}\emph{affine variety}.

There is also an equivalent notion of
\index{algebraic~set!projective}\index{variety!projective}\emph{projective
  variety} for homogeneous ideals. The projective variety associated to
the zero ideal is called the \index{projective~space}\emph{projective
  space} of dimension $n$, and is denoted by
\symb[affspace]{$\Proj^n$}{$n$-dimensional projective space}$\Proj^n$.

In the sequel we shall drop the qualificatives ``affine'' or
``projective'', and simply speak of
\index{variety!algebraic}\emph{algebraic varieties} whenever
definitions/theorems are identical.


\begin{theorem}[Nullstellensatz]
  Let $I$ be an ideal and $V$ an algebraic set. We have the following
  identities
  \begin{equation}
    \label{eq:217}
    I(V(I)) = \sqrt{I}
    \text{,}\qquad
    V(I(V)) = V
    \text{.}
  \end{equation}
\end{theorem}

\pdfmcone{Made more explicit that V must be a variety.}
If $V$ is a variety defined over $\K$, its
\index{coordinate~ring}\emph{coordinate
  ring}\symb[KV]{$\K[V]$}{Coordinate ring of an algebraic
  variety} is
\begin{equation}
  \label{eq:218}
  \K[V] \eqdef \K[x_1,\ldots,x_n]/I(V)
  \text{;}
\end{equation}
the \index{function~field}\emph{function field}
$\K(V)$\symb[KV]{$\K(V)$}{Function field of an algebraic
  variety} is its field of fractions. 

The \index{dimension~of~a~variety}\emph{dimension} of a variety $V$
is the length $d$ of the longest chain of distinct non-empty
subvarieties of $V$
\begin{equation}
  \label{eq:219}
  V_d \subset \cdots \subset V_1 \subset V
  \text{.}
\end{equation}
Equivalently, it is the length of the longest strictly decreasing
chain of prime ideals in $\K[V]$. Yet another way of defining it, is
the degree of transcendence of $\K(V)$ over $\K$.

If $V_1$ and $V_2$ are two varieties, an
\index{rational~map}\emph{affine rational map} is a map
\begin{equation}
  \label{eq:220}
  \begin{aligned}
    \phi : V_1&\ra V_2\text{,}\\
    x &\mapsto (f_1(x),\ldots,f_n(x))\text{,}
  \end{aligned}
\end{equation}
with $f_1,\ldots,f_n\in\clot{\K}(V_1)$ and such for any point $P$ at
which $f_1,\ldots,f_n$ are defined, $\phi(P)\in V_2$. An equivalent
definition exists for \emph{projective rational maps}.

A rational map that is defined at any point of $V_1$ is called a
\index{morphism~of~varieties}\emph{morphism}. A rational map (a
morphism) is
\index{rational~map!defined~over~a~field}\index{morphism~of~varieties!defined~over~a~field}\emph{defined
  over} $\K$ if $f_1,\ldots,f_n\in\K(V)$.



%%% Local Variables: 
%%% mode:flyspell
%%% ispell-local-dictionary:"american"
%%% mode: TeX-PDF
%%% mode: reftex
%%% TeX-master: "../these"
%%% End: 


