%% these.tex
%% Copyright 2010 Luca De Feo
%% All rights reserved


\selectlanguage{french}
\chapter[Intoduction (Français)][Introduction]{Introduction}

Les corps finis sont au cœur de la technologie moderne, à tel point
que la dernière génération de processeurs Intel Core possède une
instruction matérielle (CLMUL) pour la multiplications dans
$\F_{2^m}$~\cite{intel-carryless}. Cela tient au fait que les corps
finis apparaissent partout dans le génie des télécommunications, en
particulier en Codes Correcteurs d'Erreurs et Cryptographie. Cette
thèse applique des techniques algorithmiques et algébriques avancées
aux calculs dans les tours d'extensions sur les corps finis, avec pour
but des applications à la cryptographie à base de courbes elliptiques.

\paragraph*{Courbes elliptiques}
En cryptographie à base de courbes elliptiques, afin de construire un
système de chiffrement sûr, il faut sélectionner une courbe au nombre
de points divisible par un grand nombre premier. La méthode préférée
consiste à sélectionner une courbe au hasard et à appliquer un
algorithme de comptage de points pour déterminer sa cardinalité. Le
premier algorithme de comptage de points de courbes elliptiques de
complexité polynomiale fut donné par Schoof~\cite{schoof85}, puis
amélioré par Atkin et Elkies~\cite{atkin88,elkies98,schoof95}, et par
la suite nommé SEA.

L'algorithme SEA suscita de l'intérêt pour l'utilisation effective des
isogénies: ce sont des morphismes de groupes algébriques entre courbes
elliptiques. Lorsqu'on calcule des isogénies sur des corps finis, il
faut distinguer entre la caractéristique grande et la caractéristique
quelconque. Dans le premier cas, on peut utiliser des algorithmes
conçus pour la caractéristique $0$, et ensuite réduire le résultat;
les méthodes de Elkies\cite{elkies98,morain95}, Atkin~\cite{schoof95}
et Bostan, Morain, Salvy et Schost~\cite{bostan+morain+salvy+schost08}
appartiennent à cette famille. Quand la réduction modulo la
caractéristique introduit des divisions par $0$, ces algorithmes ne
s'appliquent plus.

Les deux premiers algorithmes pour calculer des isogénies en
caractéristique quelconque furent donnés par
Couveignes~\cite{couveignes94,couveignes96}; les deux ont complexité
polynomiale en la caractéristique, ce qui les rend peu pratiques pour
des valeurs supérieures à $2$ ou $3$. Un algorithme spécifique pour la
caractéristique $2$ fut donné par Lercier~\cite{lercier96}: en
pratique il est plus rapide que l'algorithme de Couveignes, mais sa
complexité n'est pas bien comprise.

Après la découverte de méthodes $p$-adiques alternatives à
SEA~\cite{satoh00,fouquet+gaudry+harley00}, l'intérêt pour le calcul
d'isogénies s'est estompé. Pourtant, deux algorithmes $p$-adiques pour
le calcul d'isogénies en caractéristique quelconque ont récemment été
proposés par Joux et Lercier~\cite{joux+lercier06} et Lercier et
Sirvent~\cite{lercier+sirvent08}; ils montrent qu'il est possible
d'éviter les divisions par $0$ en liftant les courbes dans les
$p$-adiques. Le second algorithme est actuellement celui qui a la
meilleure complexité dans le cas de la caractéristique quelconque, sa
dépendance en la caractéristique est seulement logarithmique.

Il est tout de même intéressant de remarquer qu'aucun algorithme pour
le calcul d'isogénies n'a une complexité optimale ou quasi-optimale,
avec la seule exception de~\cite{bostan+morain+salvy+schost08} dans un
cas très spécifique.

Le point de départ de ce travail a été le deuxième algorithme de
Couveignes~\cite{couveignes96}. Il calcule une isogénie par
interpolation sur les points de $p^k$-torsion de la courbe, pour $k$
assez grand; quand ces points ne sont pas définis sur le corps de
base, il faut travailler dans des extensions de corps pour les
trouver. Les extensions qui apparaissent naturellement dans ce calcul sont des corps de rupture de polynômes de la forme
\[X^p-X-\alpha\text{;}\] de telles extensions sont appelées
d'Artin-Schreier.

\paragraph*{Tours de corps finis}
\label{sec:tours-de-corps}
Mis à part l'addition, la multiplication et l'inversion, les
opérations arithmétiques importantes dans une tour d'extensions finies
sont sans aucun doute les traces relatives, les polynômes minimaux et
les inclusions de corps. Pour les corps finis, il est possible
d'ajouter des groupes de Galois effectifs à la liste, puisqu'il est
relativement facile de calculer avec ces objets.

L'arithmétique des tours de corps finis est une question de première
importance pour tout système de calcul formel, pourtant elle a reçu
peu ou pas d'attention. On sait que Magma permet de gérer des
diagrammes quelconques de corps finis depuis
longtemps~\cite{bosma+cannon+steel97}, mais il est difficile de dire
quels algorithmes y sont implantés de nos jours et avec quelles
complexités. Tous les autres résultats qui peuvent éventuellement
s'appliquer aux tours de corps finis ont été obtenus dans le contexte
plus général de la résolution de systèmes polynomiaux et de la
géométrie algébrique effective, en particulier pour la résolution des
ensembles
triangulaires~\cite{diaz+gonzalez01,giusti+lecerf+salvy01,bostan+salvy+schost03,pascal+schost06,li+moreno+schost07,dahan+jin+moreno+schost08,boulier+lemaire+moreno01,FGLM,rouiller99,alonso+becker+roy+wormann}.

Dans le cas spécifique des tours d'Artin-Schreier, il n'y a pas
énormément de littérature non plus. En s'appuyant sur des idées
contenues dans~\cite{Conway:ONAG2000}, Cantor~\cite{cantor89}
construisit une tour d'Artin-Schreier avec des propriétés spécifiques,
qu'il appliqua à la multiplication par FFT dans
$\F_2[X]$. Dans~\cite{couveignes00}, Couveignes donna un algorithme pour
le calcul d'isomorphismes entre tours d'Artin-Schreier; néanmoins, son
algorithme nécessite une multiplication rapide dans une tour, appelée
\og{}tour de Cantor\fg{} dans~\cite{couveignes00}, ayant la même forme que
celle de~\cite{cantor89}. Un tel algorithme n'est malheureusement pas
dans la littérature, ce qui rend les résultats de~\cite{couveignes00}
difficiles à exploiter en pratique.

\paragraph*{Le principe de transposition}
Un des outils algorithmiques que nous allons étudier en détail et
appliquer tout le long du document est le \emph{principe de
  transposition}, qui est à la théorie des langages ce que la dualité
est à l'algèbre.

Le principe de transposition fut découvert dans la théorie des
circuits électri\-ques par Bordewijk~\cite{bordewijk57}, puis prouvé
dans sa forme générale par Fiduccia~\cite{fiduccia:phd}; mais ce n'est
que bien plus tard, à travers les travaux de Kaltofen, Yagati, Shoup,
von zur Gathen et
autres~\cite{kaltofen+lakshman89,vzgathen+shoup92,shoup94,shoup95,shoup99,hanrot+quercia+zimmermann},
qu'il est devenu populaire en calcul formel. L'un des énoncés
possibles est le suivant:
\begin{quote}
  Soit $\pspace$ un ensemble quelconque. À tout algorithme
  $R$-algébrique, qui calcule une famille de fonctions linéaires
  $(f_p:M\ra N)_{p\in\pspace}$, correspond un algorithme
  $R$-algébrique $\dual{A}$ qui calcule la \emph{famille duale}
  $(\dual{f}_p:\dual{N}\ra\dual{M})_{p\in\pspace}$. Les complexités
  algébriques en temps et espace de $\dual{A}$ sont bornées par la
  complexité en temps de $A$.
\end{quote}

Le principe de transposition est important en calcul formel car il
permet d'obtenir des algorithmes asymptotiquement bons qui n'auraient
pas paru évi\-dents autrement. Un grand pas en avant dans sa
compréhension fut fait par Bostan, Lecerf et
Schost~\cite{bostan+lecerf+schost:tellegen} qui, en généralisant un
travail de Shoup~\cite{shoup95}, remarquèrent que la transposition
peut être appliquée de façon systématique à un langage de
programmation restreint. Il est aussi intéressant de remarquer que le
principe de transposition a des liens importants avec la
différentiation
automatique~\cite{baur+strassen83,kaltofen+lakshman89,Ka2K,gashkov+gashkov05,sergeev08}.

Dans ce document nous enquêtons plus en détail sur les rapports entre
la transposition et les langages de programmation. Nous travaillons
dans le cadre de la théorie des langages purement fonctionnels
typés~\cite{pierce}, car sa structure mathématique élégante nous
permet de raisonner sur les programmes à un niveau algébrique.

\paragraph*{Organisation du document, résultats}
Ce document est divisé en quatre parties. Dans la
partie~\ref{part:prerequisites} nous revenons sur les notions
fondamentales d'algèbre et calcul formel dont nous allons nous servir
par la suite.

La partie~\ref{part:transp-princ} a pour objet le principe de
transposition. Au Chapitre~\ref{cha:algebr-compl-dual} nous rappelons
le modèle des circuits arithmétiques et le modèle des programmes sans
branchements, puis prouvons le théorème de transposition pour
chacun. Ensuite nous évoquons les liens avec la différentiation
automatique. En complément, dans
l'Annexe~\ref{cha:basic-categ-theory}, nous donnons une nouvelle
preuve du théorème de transposition, à base de sémantique catégorique,
et étudions ses implications pour l'implantation d'un DSL en Haskell;
il s'agit un travail commun avec Mathieu Boespflug.

Le Chapitre~\ref{cha:autom-transp-code} est une collaboration avec
Éric Schost. Nous étudions les liens entre les circuits arithmétiques
et les langages fonctionnels, puis montrons que la transposition peut
être appliquée algorithmiquement à un langage fonctionnel générique.

La Partie~\ref{part:fast-arithm-using} est dédiée à l'arithmétique
dans les tours d'extensions. Nous commençons par rappeler la théorie
des idéaux zéro-dimensionnels et la représen\-tation univariée
rationnelle au Chapitre~\ref{cha:trace-computations}. Ici, les
résultats de la Partie~\ref{part:transp-princ} sont la clef pour
obtenir des algorithmes asymptotiquement rapides. Les algorithmes de
ce chapitre sont ensuite appliqués au
Chapitre~\ref{cha:artin-schr-towers}, où nous fournissons des
algorithmes asymptotiquement bons pour les tours d'Artin-Schreier
(fruit d'une autre collaboration avec Éric Schost).

Enfin, la Partie~\ref{part:appl-isog-comp} applique les résultats des
chapitres précédents au calcul d'isogénies. Après quelques rappels sur
les courbes elliptiques au Chapitre~\ref{cha:ellipt-curv-isog}, nous
passons en revue les algorithmes asymptotiquement meilleurs pour le
calcul d'isogénies sur les corps finis. Nous commençons par rappeler
l'algorithme BMSS pour le cas de la grande
caractéristique~\cite{bostan+morain+salvy+schost08} et sa
généralisation à la caractéristique quelconque de Lercier et
Sirvent~\cite{lercier+sirvent08}; puis nous rappelons l'algorithme
original de Couveignes~\cite{couveignes96} et présentons des variantes
améliorées avec un meilleur comportement asymptotique: les clefs pour
ces résultats sont le Chapitre~\ref{cha:artin-schr-towers} et de
nouvelles idées algorithmiques pour l'interpolation dans les tours
d'extensions. Nous présentons aussi en Section~\ref{sec:bounded} une
généralisation surprenante de l'algorithme de Couveignes, qui permet
le calcul d'isogénies de degré inconnu au même prix que le calcul
d'isogénies de degré prescrit. Cette découverte éclaire davantage la
(sous)optimalité de l'algorithme de Couveignes et pourrait avoir des
applications en
cryptologie~\cite{gaudry+hess+smart02,GHS,hess03,teske06}.

\pdfmctwo{Rachel trouvait que "ce manuscrit n'aurait pas d'intérêt"
  était une figure de style qui n'a pas sa place dans l'introduction
  d'une thèse, et que "nos paquets logiciels" faisait trop
  commercial.}  La théorie ne suffirait pas sans pratique. De la même
façon, ce manuscrit ne serait pas complet s'il n'était accompagné par
les paquets logiciels que nous avons développés. La grande majorité
des algorithmes présentés ici a été implantée, paquetée et distribuée
avec des licences \emph{open source}. Ainsi, tous les algorithmes du
Chapitre~\ref{cha:artin-schr-towers} sont disponibles dans la
bibliothèque \texttt{FAAST}, écrite en \texttt{C++} et disponible à
l'adresse \url{http://www.lix.polytechnique.fr/~defeo/FAAST/}.  Au
moment où nous écrivons, le compilateur pour le langage
\texttt{transalpyne} du Chapitre~\ref{cha:autom-transp-code} n'est pas
encore distribué; nous travaillons en ce moment à la première
\emph{stable release} et espérons commencer la distribution au début
de 2011. Il sera disponible à l'adresse
\url{ http://transalpyne.gforge.inria.fr/}.



\selectlanguage{american}

%%% Local Variables: 
%%% mode:flyspell
%%% ispell-local-dictionary:"francais"
%%% mode: TeX-PDF
%%% mode: reftex
%%% TeX-master: "../these"
%%% End: 
